% Remove number from Zero chapter
% 0. Introduction -> Introduction
\makeatletter
\titleformat{\chapter}%
  [display]% shape
  {\relax\ifthenelse{\NOT\boolean{@tufte@symmetric}}{\begin{fullwidth}}{}}% format applied to label+text
%  {\itshape\huge\thechapter}% label
  {}% label
  {0pt}% horizontal separation between label and title body
  {\huge\rmfamily\itshape}%\thechapter. }% before the title body
  [\ifthenelse{\NOT\boolean{@tufte@symmetric}}{\end{fullwidth}}{}]% after the title body
\makeatother


\chapter*{Предисловие}
\label{ch:intro}
\addcontentsline{toc}{chapter}{Предисловие}
%\addcontentsline{toc}{part}{Предисловие}


% Return back a number before chapter
% Chapter -> 1. Chapter
\makeatletter
\titleformat{\chapter}%
  [display]% shape
  {\relax\ifthenelse{\NOT\boolean{@tufte@symmetric}}{\begin{fullwidth}}{}}% format applied to label+text
%  {\itshape\huge\thechapter}% label
  {}% label
  {0pt}% horizontal separation between label and title body
  {\huge\rmfamily\itshape\thechapter. }% before the title body
  [\ifthenelse{\NOT\boolean{@tufte@symmetric}}{\end{fullwidth}}{}]% after the title body
\makeatother


Это издание адресовано не только студентам, но и школьникам, изучающим информатику. 
Оно поможет тем, кто хочет развить свои аналитические навыки, 
научиться программировать 
и глубже разбираться в принципах работы самого компьютерного вики-проекта.
Вы познакомитесь с работой базы Викиданных и языком запросов SPARQL. 
Прочитав пособие, вы сможете извлекать из Викиданных информацию с~помощью SPARQL-скриптов, 
затем обрабатывать её и строить по ней таблицы, графики и карты.

Викиданные~--- это искусно сделанная база данных, 
которая, как огромный кит, лежит в основе громадной планеты 
Википедии. Впечатляет скорость роста этого кита, во многих главах мы будем обращать на это внимание.
Если изначально проект Викиданные создавался для обслуживания нужд Википедии, 
то сейчас он применяется крайне широко и в~самых разных целях.

В первой части пособия (\hyperref[part:foundation]{Основы Викиданных}) объясняется, 
что такое запросы и листинги программ, как ими пользоваться, даны базовые понятия об объектах Викиданных. 
Глава <<Обзор Викиданных>> включает историческую справку, 
вопрос качества данных и связи с Википедией. 
Описан сервис WDQS, к~которому мы будем обращаться на каждой странице. 
Также включён небольшой обзор научных исследований, связанных с Викиданными. 
Глава <<Корзины и мячи>> с помощью образов и аналогий позволяет подступиться к Викиданным тем, 
кто хочет научиться программировать. 

%включает в себя 10 уроков, описывающих язык программирования и протокол SPARQL.
%\newthought{Вторая часть} содержит рецепты решения самых разных практических задач, 
%возникающих при работе с объектами Викиданных.

Во второй и основной части издания (\hyperref[part:research]{Исследуем объекты Викиданных}) каждая 
глава~--- это небольшое приключение, 
где один из объектов Викиданных является главным героем. 
С помощью SPARQL-запросов мы пытаемся расспросить Викиданные об этом герое, 
с помощью графиков, таблиц, диаграмм~--- проанализировать и нарисовать его образ. 



\newpage
В третьей части пособия (\hyperref[part:advanced]
                                  {Специальные возможности вики-проектов и Викиданных}) 
рассмотрен вопрос защиты страниц в вики-проектах, 
рассказано о~сервисе балансировки ProWD и о том, 
как можно последовательно двигаться от~SPARQL-запросов 
к~языку Python при~создании компьютерных программ (ботов), редактирующих Викиданные. 


В заключительной части 
даны ответы на вопросы, рассеянные по всему пособию, приведены список литературы 
и индекс ключевых слов для удобного постраничного поиска.

В исследовании объектов Викиданных приняли участие студенты ПетрГУ в рамках курса <<Программирование Викиданных>>. 
Материалы этого онлайн-курса доступны на~сайте Викиверситета\sidenote{%
% 
URL: \href{https://ru.wikiversity.org/?curid=23388}
          {https://ru.wikiversity.org/wiki/Программирование Викиданных}.%
}. 
Если у~вас будут вопросы, замечания или пожелания к~этой книге, 
то~пишите их на~странице обсуждения курса\sidenote{%
% 
URL: \href{https://ru.wikiversity.org/?curid=28956}
          {https://ru.wikiversity.org/wiki/Обсуждение:Программирование Викиданных}.%
}. 

Курс и книга разрабатывались и писались с 2017 года. 
За семь лет часть сайтов, на~которые мы ссылались, канула в лету. 
Копии таких утраченных сайтов вы можете найти в~Архиве Интернета (\href{https://archive.org/}{archive.org}).

% todo to restore
%Учебное пособие распространяется на правах свободной лицензии 
%\href{https://creativecommons.org/licenses/by-sa/4.0/deed.ru}{Creative Commons Attribution-ShareAlike}.

Потребовалось несколько лет работы со студентами в Википедии, прежде чем мы пришли с~ними к~таким проектам, 
как Викиверситет и Викиданные. 
Результатом предыдущей работы в Википедии стало учебное пособие для тех, 
кто хочет научиться редактировать мировую энциклопедию\sidenote{\fullcite{Krizhanovsky2015}.}. 
Это пособие доступно онлайн\sidenote{%
% 
URL: \href{https://commons.wikimedia.org/?curid=86907665}
          {https://commons.wikimedia.org/?curid=86907665}.%
}. 


% \newthought{Книга научит} вас делать ... todo
%\vspace{5mm}


Будет ли в этом издании рассказано, 
как редактировать и пополнять Викиданные новой информацией?~--- Нет, 
хотя это не сложнее, чем писать текст в SMS или статью в Википедии. 
Здесь мы покажем вам, как увлекательно формулировать вопросы на языке Викиданных. 


Почему мы занялись и увлеклись Викиданными?~--- Потому что это самая большая, сложная 
и быстрорастущая база данных на Земле. 
Потому что эта база лежит в основе Википедии, которую может редактировать каждый.
И Викиданные тоже может редактировать каждый. 
Часто редакторы родственных проектов (Википедия, Викисклад, Викисловарь) 
заглядывают в Викиданные, чтобы там что-то поправить, изменить. 
Работу серверов Википедии и Викиданных обеспечивают сотрудники организации Викимедиа. 



Почему мы адресуем издание не только студентам, но и школьникам?~--- Потому что 
программирование и программирование Викиданных в частности~--- это 
увлекательная и, если увлечься, простая вещь. 
Если с помощью учителя или самостоятельно школьник увлечётся, 
то сможет программировать Викиданные, 
многие ребята плодотворно редактируют статьи Википедии. 
Надеемся, что благодаря этому пособию одним из первых компьютерных языков, 
изучаемых в~школе, станет язык Викиданных.



\newpage
\begin{marginfigure}%[baselineskip]
{%
\setlength{\fboxsep}{0pt}%
\setlength{\fboxrule}{1pt}%
\fcolorbox{gray}{gray}{\includegraphics[width=0.7\linewidth]{./intro/WDQS_play_button.png}}
}
%\setlength{\abovecaptionskip}{15pt} fail
%\vspace*{7pt} ++
\caption{Выполнение скрипта в сервисе Wikidata Query Service}
%Кнопка Play запуска скрипта в сервисе Wikidata Query Service. Также скрипт можно выполнить при одновременном нажатии кнопок Ctrl и Enter на~клавиатуре}%
\label{fig:WDQS_play_button}%
\end{marginfigure}%
Покажем прямо сейчас, что Викиданные находятся от нас на расстоянии одного клика.
Откройте на компьютере или на телефоне ссылку: 
\href{https://w.wiki/4cXU}{https://w.wiki/4cXU}. 
Вы увидите главное окно, в котором мы с вами будем писать наши небольшие программы. 
Если вы нажмёте большую синюю кнопку с белым треугольником (рис.~\ref{fig:WDQS_play_button}), 
то запустите эту программу из~семи строк. 
Программа обратится к базе данных 
и спросит, какие столицы есть в Викиданных. Результатом будет список столиц. %\sidenote[][1\baselineskip]{%
%
Подробнее о городах в Викиданных читайте в~разделе <<\nameref{ch:city}>> на с.~\pageref{ch:city}. 
%}. % 
Вот так просто можно запускать SPARQL-программы: достаточно браузера, 
Интернета и гиперссылки с запросом к~Викиданным.

Дадим читателю подсказку относительно тех изображений, которые кажутся слишком мелкими. Поскольку это электронное издание, то вы можете воспользоваться зумом и увеличить изображение. Как раз для этого мы старались включать сюда изображения с максимально большим разрешением. 

