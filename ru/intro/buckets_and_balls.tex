\chapter{Корзины и мячи}
\label{ch:BucketsAndBalls}
\marginnote[-6\baselineskip]{Глава основана на статье~--- \cite{bucketsAndBalls}~--- с~разрешения автора.}

Понятие <<связанные данные>> (Linked Data) по-прежнему недопонято и недооценено. 
Возможно, эти данные кажутся слишком сложными. 
Попробуем разобраться. Начнём с~переменных. 
Отметим, что Викиданные являются примером связанных данных (рис.~\ref{fig:Wikidata_in_linked_open_data}).

\begin{marginfigure}[-5\baselineskip]
	{
		\setlength{\fboxsep}{0pt}%
		\setlength{\fboxrule}{1pt}%
		\fcolorbox{gray}{gray}{\includegraphics[width=0.86\linewidth]{./intro/bucketsAndBalls/Wikidata_in_linked_open_data.png}}
	}
    \caption[Викиданные в связанном облаке открытых данных.]{Викиданные в связанном облаке открытых данных. 

\noindent Базы данных обозначены кружками (Викиданные обозначены как \textit{WD}) с~серыми линиями, связывающими базы данных в сети, если их данные выровнены. См. статью в Английской Википедии: \href{https://en.wikipedia.org/wiki/Linked_data}{Linked data}. Wikimedia Commons / \href{https://commons.wikimedia.org/wiki/File:Wikidata_in_the_Linked_Open_Data_cloud_2020-08-20.svg}{Thomas Shafee}}
	\label{fig:Wikidata_in_linked_open_data}
\end{marginfigure}

Что такое переменная в SPARQL? Пусть это будет то, что нужно чем-то заполнить. 
Но как себе представить это <<то>>? Что-то абстрактное легче представить, 
если связать с~чем-то физическим и~конкретным. 
Трудно представить себе время, но как только мы представим себе часы со стрелками, становится легче. Мы не можем представить себе мебель вообще, но представить стул может каждый.

Другая трудность заключается в том, 
чтобы сформулировать запрос на языке SPARQL. 
Хотя работа со SPARQL помогает понять, как работает граф знаний\footnote[][12pt]{%
%
\index{Информатика!Граф знаний}%
Граф знаний~--- это база знаний, которая использует графовую структурированную модель для интеграции данных. 
Графы знаний часто используются для хранения взаимосвязанных описаний сущностей~--- объектов, 
событий, ситуаций или абстрактных понятий. 
%Далее будет построен граф знаний (рис.~\ref{fig:Query_in_basket_and_balls_notation_with_ids}).%
}, 
запрос SPARQL на него не похож. Это как с~символами в математике: <<5 не похоже на пять, в то время как ||||| равно пяти>>.

Итак, как решить проблему с определением переменной и формулировкой SPARQL-запроса?

Представим себе каждый SPARQL-запрос в виде графа связанных между собой корзин и~мячей.

Пусть переменные являются чем-то, что нужно заполнить, но сейчас переменная~--- это абстрактное понятие. 
Нам нужен физический контейнер, чтобы наполнить его вещами. Нам нужны корзины  
и вещи, похожие на мячи. Давайте представим выполнение запроса как заполнение корзин мячами.
Тогда процесс заполнения графа мячами будет выглядеть так, как представлено 
на~рис.~\ref{fig:Query_as_filling_buckets_with_balls}. 
Корзина \textbf{?A} должна быть заполнена теми мячами, которые имеют отношение \textbf{R} к мячу \textbf{B}.

\newpage
\begin{marginfigure}[0cm]
	{
		\setlength{\fboxsep}{0pt}%
		\setlength{\fboxrule}{1pt}%
		\fcolorbox{gray}{gray}{\includegraphics[width=0.82\linewidth]{./intro/bucketsAndBalls/Query_as_filling_buckets_with_balls.PNG}}
	}
    \caption{Образец графа заполнения корзин мячами}
	\label{fig:Query_as_filling_buckets_with_balls}
\end{marginfigure}

Рисунок будет понятнее, если мы его упростим и нарисуем прямую линию (рис.~\ref{fig:Graph_pattern_in_basket_and_balls_notation}). Это шаблон графа в нотации <<Корзины и мячи>>. Направление отношения R не показано, но оно всегда слева направо.
Тогда процесс написания и выполнения SPARQL-запроса будет состоять из следующих шагов:
\begin{enumerate}
    \item Выберите свои корзины (в них вы будете собирать нужные вам мячи).
    \item Составьте свои условия в виде графа корзин и мячей.
    \item Запустите свой запрос, чтобы наполнить корзины мячами.
\end{enumerate}


\begin{marginfigure}[2\baselineskip]
	{
		\setlength{\fboxsep}{0pt}%
		\setlength{\fboxrule}{1pt}%
		\fcolorbox{gray}{gray}{\includegraphics[width=0.82\linewidth]{./intro/bucketsAndBalls/Graph_pattern_in_basket_and_balls_notation.PNG}}
	}
    \caption{Шаблон графа в нотации <<Корзины и мячи>>}
	\label{fig:Graph_pattern_in_basket_and_balls_notation}
\end{marginfigure}

Теперь напишем запрос, чтобы, например, получить список всех руководителей областей России, выполняя такие шаги:

\begin{enumerate}
    \item Возьмём две корзины: одна для регионов, другая для руководителей.
    \item Корзину для регионов привяжем к мячу <<область России>> отношением <<экземпляр>> (instance~of). 
        Тогда из множества мячей~--- объектов Викиданных~--- в эту корзину попадут только те мячи, которые являются областью России. 
        Корзина для регионов связана с~корзиной <<руководитель>> (head) отношением <<имеет руководителя>>, 
        в нашем случае это губернатор или глава области.
    \item 
          Сделаем запрос более интересным и добавим ещё одну корзину для~фотографий губернаторов. 
\end{enumerate}
Теперь запрос в~нотации <<Корзины и мячи>> будет выглядеть как на~рис.~\ref{fig:Query_in_basket_and_balls_notation}.

\begin{marginfigure}
    \includegraphics[width=0.95\linewidth]{./intro/bucketsAndBalls/Query_in_basket_and_balls_notation.PNG}
    \caption[Запрос в нотации <<Корзины и мячи>> для заполнения корзин мячами.]{Запрос в нотации <<Корзины и мячи>> для заполнения корзин <<регион>> мячами <<область России>>, корзины <<руководитель>>~--- губернаторами или главами области, корзины <<изображение>>~--- их фотографиями}
	\label{fig:Query_in_basket_and_balls_notation}
\end{marginfigure}




\newpage
После выполнения запроса (рис.~\ref{fig:Query_in_basket_and_balls_notation}) 
наши корзины будут выглядеть как на рис.~\ref{fig:3_buckets_region_head_image}.

\begin{marginfigure}%
{%
		\setlength{\fboxsep}{0pt}%
		\setlength{\fboxrule}{1pt}%
		\fcolorbox{gray}{gray}{\includegraphics[width=0.65\linewidth]{./intro/bucketsAndBalls/3_buckets_region_head_image.PNG}}
}
    \caption[Заполненные корзины после выполнении запроса.]{Заполненные корзины после выполнении запроса:\\ 
    %на~рис.~\ref{fig:Query_in_basket_and_balls_notation}.\\ 
    \textit{?region}~--- области России, \textit{?head}~--- руководители,\\
    \textit{?image}~--- фотографии руководства}
	\label{fig:3_buckets_region_head_image}
\end{marginfigure}

Теперь перейдём в сервис \href{https://query.wikidata.org/}{Wikidata Query Service} 
(далее~--- WDQS\footnote[][15pt]{См. о службе WDQS на c.~\pageref{sect:WDQS}.}) 
и напишем этот запрос на~языке SPARQL. Сначала выберем и назовём корзины для заполнения так:

\begin{lstlisting}[ language=SPARQL ]
SELECT DISTINCT ?region ?head ?image
\end{lstlisting}

Далее напишем условия соответствия мячей корзинам. 
Все условия в SPARQL должны заключаться в фигурные скобки (см.~рис.~\ref{fig:Query_in_basket_and_balls_notation}).

Когда нам необходимо определённое отношение или мяч, нам нужно использовать их идентификаторы. 
Викиданные облегчают поиск идентификатора и подсказывают его, 
когда вы выбираете отношение (свойство) или мяч (объект) по его имени (метке). 
Чтобы указать отношения в графе знаний Викиданных, 
мы используем префикс \textit{wdt:}, а для объектов (наших мячей)~--- префикс \textit{wd:}.

Следуя нашей нотации (рис.~\ref{fig:Query_in_basket_and_balls_notation}), 
напишем условия для заполнения первой корзины \textit{?region}, затем напишем первое отношение. 
Поскольку общей частью идентификаторов прямых отношений является wdt, мы пишем \textit{wdt}:, 
а затем в сервисе WDQS нажимаем Ctrl~+~пробел, чтобы запустить службу подсказок или  автозаполнения Викиданных (рис.~\ref{fig:WDQS_popup_instance_of}).

\begin{marginfigure}[0cm]
	{
		\setlength{\fboxsep}{0pt}%
		\setlength{\fboxrule}{1pt}%
		\fcolorbox{gray}{gray}{\includegraphics[width=0.8\linewidth]{./intro/bucketsAndBalls/WDQS_popup_instance_of.PNG}}
	}
    \caption[Меню автозаполнения свойства Викиданых в сервисе WDQS.]
            {Контекстное меню автозаполнения свойства Викиданых в сервисе WDQS, 
             открываемое с~помощью команды Ctrl~+~пробел}
	\label{fig:WDQS_popup_instance_of}
\end{marginfigure}

После этого мы воспроизведём модель из рис.~\ref{fig:Query_in_basket_and_balls_notation} в реальном запросе SPARQL.

Обычно при написании SPARQL-запроса его представляют в виде таблицы с~тремя столбцами: 
субъект, предикат, объект; 
или на языке Викиданных~--- объект, свойство, значение.

Возможно, начинающему программисту осваивать язык SPARQL будет проще, 
если первые запросы будут напоминать граф. 
Попробуем взять граф на рис.~\ref{fig:Query_in_basket_and_balls_notation} 
и написать запрос~\ref{lst:linkRegionsOfHeads} на языке SPARQL максимально похожим образом:

\begin{lstlisting}[ language=SPARQL, caption={\href{https://w.wiki/4NVc}{Список руководителей областей России с фотографиями}\protect\footnotemark},label=lst:linkRegionsOfHeads, ]
SELECT DISTINCT ?region ?head ?image
{
    ?region wdt:P31 wd:Q835714; # oblast of Russia
            wdt:P6  ?head. # heads of government
    ?head  wdt:P18 ?image. # images of heads of government
}
\end{lstlisting}
\footnotetext{Получено 44 ссылки на области России и их руководителей. Ссылка на SPARQL-запрос: \href{https://w.wiki/4NVc}{https://w.wiki/4NVc}.}




\newpage 
Теперь посмотрите, как этот запрос~\ref{lst:linkRegionsOfHeads} выглядит в сервисе WDQS, и запустите его. 
Затем нажмите на значок глаза слева и выберите \textit{image grid} 
(рис.~\ref{fig:WDQS_drop_down_result_type}), чтобы просмотреть результаты в~виде сетки изображений.

\begin{marginfigure}
	{
		\setlength{\fboxsep}{0pt}%
		\setlength{\fboxrule}{1pt}%
		\fcolorbox{gray}{gray}{\includegraphics[width=0.3\linewidth]{./intro/bucketsAndBalls/WDQS_drop_down_result_type.PNG}}
	}
    \caption[Выбор отображения результатов запроса в виде сетки изображений.]{Выбор отображения результатов запроса в виде \textit{image grid} (сетки изображений)}
	\label{fig:WDQS_drop_down_result_type}
\end{marginfigure}

В этой сетке результатов запроса 
под каждой фотографией мы видим только идентификаторы людей и~областей. 
И если щёлкнуть по гиперссылке идентификатора, то мы получим много информации об~этом объекте. 
Но нагляднее было бы в этой сетке, кроме идентификаторов, указывать имена людей и названия областей. 
Это похоже на наклеивание этикеток на наши корзины (рис.~\ref{fig:Query_in_basket_and_balls_notation_with_ids}).

\begin{marginfigure}
\includegraphics[width=0.99\linewidth]{./intro/bucketsAndBalls/Query_in_basket_and_balls_notation_with_ids.png}
\caption{Запрос в нотации <<Корзины и мячи>> с~номерами свойств и~объектов Викиданных}
\label{fig:Query_in_basket_and_balls_notation_with_ids}
\end{marginfigure}

Метка (\textit{Label})~--- это то, что есть у каждого мяча, то есть объекта Викиданных. 
Метка~--- это имя, позволяющее различать объекты между собой. 
В Викиданных есть сервис, упрощающий вывод меток, 
для этого достаточно добавить в запросе в конец имени переменной слово \textit{Label}.
Чтобы активировать этот сервис, на новой строке внутри фигурных скобок наберите Ctrl~+~пробел; 
когда вы начнёте писать слово \textit{Label}, тогда будет добавлена строка с~этим сервисом\footnote[][12pt]{%
%
Пример вызова этого сервиса представлен в запросе~\ref{lst:regionsOfHeads}, в~строке~8. 
В~этой строке языком для меток указан русский язык (код ru).}. По умолчанию 
вы получаете подсказку с~языком интерфейса 
и английским языком в качестве альтернативы, если метка недоступна на языке выбранного интерфейса Викиданных.


Викиданные полны таких замечательных сервисов, и для окончательного запроса мы воспользуемся ещё одним. 
Чтобы получить результат сразу в виде набора фотографий глав областей, 
без дополнительного нажатия на значок глаза, 
поместите где-нибудь в~своём запросе следующую конструкцию для~сервиса WDQS:
\begin{lstlisting}[ language=SPARQL ]
#defaultView:ImageGrid
\end{lstlisting}

На самом деле не всё нужно писать вручную. Когда вы начнёте набирать текст, сервис автозаполнения предложит варианты.


\newpage
Итоговый запрос~\ref{lst:regionsOfHeads} позволил нам получить требуемую сетку фотографий 
с подписью в виде названия файла иллюстрации, имени руководителя и названия области (рис.~\ref{fig:Result_of_the_request}).

%\lstset{numbers=left, firstnumber=1, frame=single}
\begin{lstlisting}[ 
        language=SPARQL, 
        xleftmargin=18pt, 
        numbers=left,
        caption={\href{https://w.wiki/4ENR}{Список руководителей областей России с фотографиями}\protect\footnotemark}, 
        label=lst:regionsOfHeads, 
        ]
# List of regions of the Russia and images of heads of government
#defaultView:ImageGrid
SELECT DISTINCT ?region ?regionLabel ?head ?headLabel ?image
{
  ?region wdt:P31 wd:Q835714; # ?region is Oblast of Russia
          wdt:P6  ?head.      #         has head of government
  ?head  wdt:P18 ?image.      # head has image
  SERVICE wikibase:label {bd:serviceParam wikibase:language "ru"} 
}
\end{lstlisting}
\footnotetext{Получено: 44 области России и список их руководителей. Ссылка на SPARQL-запрос: \href{https://w.wiki/4ENR}{https://w.wiki/4ENR}.}

%\begin{fullwidth}
\begin{figure*}[h!]
\includegraphics[width=0.9\linewidth]{./intro/bucketsAndBalls/result_of_request_for_photos_of_heads_of_government.png}
    \caption[Сетка фотографий руководителей областей.]
    {Сетка фотографий руководителей областей 
        с~названиями файлов с~фотографиями, 
        с именами руководителей 
        и названиями областей}
        %Руководители областей: их имена и фотографии, названия областей. Результат запроса~\protect\ref{lst:regionsOfHeads} в виде сетки изображений}
    \label{fig:Result_of_the_request}
\end{figure*}
%\end{fullwidth}

Такое представление о запросах SPARQL, как о связанных корзинах и мячах, может быть полезным, 
по крайней мере, в начале освоения Викиданных. 
И, конечно, у каждой метафоры есть свои ограничения. 
Например, нельзя поместить один и тот же мяч в две разные настоящие корзины, 
но~в~эти виртуальные~--- можно. 
<<Корзины и мячи>> могут быть полезны, чтобы вскарабкаться на~высоту абстракции Викиданных.
