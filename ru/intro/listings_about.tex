


%\chapter{Глава для учителей: технология написания книги со~студентами}
%\label{ch:howto-write-book}



\chapter{Программы и программки}
\label{ch:listing_about}

В книге приводится \emph{программный код}\footnote[][0cm]{%
    Программный код также называют  \emph{исходным кодом} или 
    \index{Программирование!Определения / Листинг} 
    \emph{листингом}.
%   
} на языке \ruwiki{vGq}{SPARQL}. 
Именно на этом языке пишут запросы к Викиданным.


Вот пример SPARQL-скрипта (листинг~\ref{lst:cities}), 
с помощью которого можно получить из Викиданных список городов, 
точнее экземляров объекта\footnote{\label{question:instance-in-OOP-vs-Wikidata}Что такое экземпляр объекта? 
    Какая разница между экземпляром объекта 
    в объектно-ориентированном программировании и в Викиданных?
    См. ответ~\ref{answer:instance-in-OOP-vs-Wikidata} на с.~\pageref{answer:instance-in-OOP-vs-Wikidata}.
    }
\wdqName{city}{515}.

%       escapebegin=ы,escapeend=я>
%       escapechar=ы
% # после знака # (то есть в комментариях) можно ставить \footnote в lstlisting
\begin{lstlisting}[ language=SPARQL, 
                    caption={\href{https://w.wiki/jcE}{Список городов.}\protect\footnotemark},
                    label=lst:cities,
                    texcl 
                    ]
SELECT ?city ?cityLabel WHERE { 
  ?city wdt:P31 wd:Q515.       # instance of city
  SERVICE wikibase:label { bd:serviceParam wikibase:language "ru" }
}
\end{lstlisting}%
\footnotetext{Получено \num{20800} городов в 2017 году, \num{9260} городов в 2020 году. Ссылка на SPARQL-запрос: \href{https://w.wiki/jcE}{https://w.wiki/jcE}}

Мы будем регулярно ссылаться на объекты Викиданных. 
Например, \wdqName{city}{515}\footnote[][0cm]{%
%    
    В электронной версии книги имена объектов Викиданных включают гиперссылки на соответствующие страницы Викиданных.
} 
--- это имя объекта Викиданных. 
Здесь \emph{city}~--- имя метки объекта (Label), 
a \emph{Q515}~--- это уникальный идентификатор объекта 
и одновременно название страницы с описанием этого объекта на сайте Викиданных: \href{https://www.wikidata.org/wiki/Q515}{https://www.wikidata.org/wiki/Q515}.

\label{WDObjectsNumbering}
Объекты Викиданных нумеруются последовательно\marginnote[0cm]{Нумерация объектов в Викиданных.%
}%
, чем раньше объект создан в Викиданных, тем меньше его номер. 
Так сложилось, что обычно объекты, имеющие меньший номер, являются более значимыми. 
Например, \emph{Солнце} имеет идентификатор \wdq{525}, а менее значимый объект \emph{мушка дрозофилы}~--- \wdq{312154}.
