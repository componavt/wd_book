\chapter{Анализ академий}
\label{ch:academy-analysis}

В этом разделе мы будем ранжировать академии, 
в которые входят отечественные математики, 
по тематикам и географическому расположению. 
В первую очередь, с помощью SPARQL-запроса получаем  список всех организаций, 
в которые входят исследуемые персоны (листинг~\ref{lst:academy-types}). 
В результате было выявлено 30 типов организаций.

\begin{lstlisting}[ language=SPARQL, 
                    caption={\href{https://w.wiki/t4o}{Типы академий, членами которых являются математики}\protect\footnotemark},
                    label=lst:academy-types
                    ]
# types of organizations (academies) which includes mathematicians, "member of" (P463)
SELECT ?academy_class ?academy_classLabel
WHERE {
  ?mathematician wdt:P106 wd:Q170790. # occupation is mathematician
  #country of citizenship
  { ?mathematician wdt:P27 wd:Q34266 } UNION  # Russian Empire
  { ?mathematician wdt:P27 wd:Q15180 } UNION  # Soviet Union
  { ?mathematician wdt:P27 wd:Q159 }.         # Russia
  ?mathematician wdt:P463 ?academy. # member of academy
  ?academy wdt:P31 ?academy_class.  # which types of academies includes mathematicians
  SERVICE wikibase:label { bd:serviceParam wikibase:language "en" }
}
GROUP BY ?academy_class ?academy_classLabel
\end{lstlisting}
\footnotetext{Получено \num{53} типа академий на 2021 год. Ссылка на SPARQL-запрос: \href{https://w.wiki/33XL}{https://w.wiki/33XL}}


...

%%%%%%%%%%%%%%%%%%%%%%%%%%%%%%%%%%%%%%%%%%%%%%%%%%%%%%%
\section{Название следующего подраздела \ldots}




