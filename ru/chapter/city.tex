\chapter{От малых городов до городов-миллионеров}
\label{ch:city}

Статья посвящена исследованию различных типов городов, соответствующих четырем объектам Викиданных — «Малый город», «Город», «Большой город» и «Города-миллионеры». В ходе исследования с использованием SPARQL-запросов получены данные о количестве экземпляров исследуемых объектов, а также рассмотрены вопросы, связанные со свойствами population (численность населения) и sister city (город-побратим) этих объектов Викиданных. В том числе были решены следующие задачи: подсчет и анализ численности населения разных типов городов; определение числа городов, не имеющих побратимов; построение списка городов, упорядоченного по числу побратимов; нахождение числа городов с определённым числом побратимов; определение страны с наибольшим числом побратимов; нахождение ближайших соседей России. В заключении работы дана оценка полноты данных, представленных в Википедии и Викиданных, и перечислены проблемы и сложности, возникшие при изучении объектов разных типов городов.