\chapter{Военные корабли и их операторы}
\label{ch:ships-chapter}

Глава посвящена отечественным и зарубежным кораблям. Они могут иметь как военное, так и гражданское назначение. Гражданские суда используются в грузоперевозках, рыболовстве, туризме, разведке полезных ископаемых, спасательных работах, а также в спортивных, культурных и других целях. Для хранения информации о судах и других объектах ведутся базы знаний. Одной из таких баз знаний являются Викиданные. В этой главе изучены хранимые в Викиданных объекты кораблей и проведена оценка качества и полноты их описания.


\begin{marginfigure}[0.0cm]
  \includegraphics[width=0.9\linewidth]{chapter/ship/Russian_ships_topic_imbalance.png}
  \caption[График равномерности заполнения свойств объектов Викиданных.]{График равномерности заполнения по числу свойств объекта Викиданных \href{https://www.wikidata.org/wiki/Q11446}{корабль (Q11446)}, коэффициент Джини равен 0.239. Данные получены с~помощью сервиса ProWD.id, 2020 год. График и коэффициент Джини показывают низкую равномерность заполнения свойств}%
  \label{fig:prowd_ships-unbalanced}%
\end{marginfigure}


\section{Список кораблей}

Нам потребуются объект \wdqName{корабль}{11446} и свойства:
   \wdProperty{31}{экземпляр}, 
   \wdProperty{137}{оператор}, 
   \wdProperty{17}{государство} и  
   \wdProperty{607}{конфликт}.
Построим список всех кораблей с помощью запроса~\ref{lst:ships_ru}.


\begin{lstlisting}[ language=SPARQL, 
            caption={{\href{https://w.wiki/wX6}{Список кораблей}}\protect\footnotemark}, 
            label=lst:ships_ru, 
            numbers=none
            ]
# List of ships
SELECT ?ship ?shipLabel WHERE {
  ?ship wdt:P31 wd:Q11446. # instance of ship
  SERVICE wikibase:label {bd:serviceParam wikibase:language "ru, en"}
}
\end{lstlisting}
\footnotetext{Получено: \num{19820} кораблей в 2017 году, 
                        \num{50681}~--- в 2020-м, 
                        \num{71203}~--- в 2021 году. 
        Ссылка на~SPARQL-запрос: \href{https://w.wiki/wX6}{https://w.wiki/wX6}.}


\newpage
По данным ProWD, больше всего свойств (34 свойства) имеет 
\href{https://www.wikidata.org/wiki/Q281147}{ледокол Красин (Q281147)}, 
а~меньше всего, по~пять свойств, у~кораблей 
\href{https://www.wikidata.org/wiki/Q99198666}{Ливень (Q99198666)} и 
\href{https://www.wikidata.org/wiki/Q28155282}{Dispatch (Q28155282)}~\autocite{ProWD_ru_ships}.

В Викиданных, как правило, записывается не прямая принадлежность корабля стране, 
а~принадлежность некоторому оператору. 
Чаще всего это какая-либо организация, 
например 
\href{https://www.wikidata.org/wiki/Q465283}{Военно-морской Флот Российской Федерации (Q465283)}. 
Составим список кораблей, 
операторы которых находятся или находились в России, СССР или Российской империи 
(см. запрос~\ref{lst:ships_with_ru_op}). 

\marginnote{%
    \label{question:ship_1}
    \MarginQuestion
    Найдите <<корабль Гиннесса>> по какому-либо параметру. 
    Например, напишите скрипт для поиска самого большого, самого длинного 
    или самого вместительного корабля.

    Ответ на с.~\pageref{answer:ship_Guinness}.
}


\begin{lstlisting}[ 
    language=SPARQL, 
    caption={\href{https://w.wiki/6gHD}{Cписок кораблей, операторы которых находятся или находились в России, 
            СССР или Российской империи}\protect\footnotemark}, 
    label=lst:ships_with_ru_op 
                  ]
# List of ships from Russia, Soviet Union and Russian Empire
SELECT ?ship ?shipLabel WHERE {
  VALUES ?country {wd:Q34266 # Russian Empire
                   wd:Q15180 # Soviet Union
                   wd:Q159}  # Russia
  ?ship wdt:P31 wd:Q11446;         # instance of ship
        wdt:P137/wdt:P17 ?country; # operator in country
  SERVICE wikibase:label {bd:serviceParam wikibase:language "ru, en"}
}
\end{lstlisting}
\footnotetext{Получено: 107 кораблей в 2017 году, 578 кораблей в 2021 году. Ссылка на~SPARQL-запрос: \href{https://w.wiki/6gHD}{https://w.wiki/6gHD}.}

Обратите внимание на~строку~7 в~запросе~\ref{lst:ships_with_ru_op}. 
В~этой строке записаны последовательно два свойста \texttt{wdt:P137/wdt:P17}:
\begin{itemize}
	\item \texttt{wdt:P137}~--- свойство \wdProperty{137}{operator} связывает корабль и оператора судов;
	\item \texttt{wdt:P17}~--- свойство \wdProperty{17}{country} связывает оператора судов и страну.
\end{itemize}
Таким образом, эта строка позволяет кратко и напрямую задать страну для корабля 
посредством свойства <<оператор>>. 
Такое последовательное указание свойств (\texttt{wdt:P137/wdt:P17}) 
является аналогом безымянных переменных (\lstinline|[]|),
\marginnote[-2\baselineskip]{
    \MarginInternalLink
    Пример использования безымянной переменной (\lstinline|[]|) 
    см.~в~запросе~\ref{lst:human-settlement-noname-count} на~с.~\pageref{lst:human-settlement-noname-count}.%
} %
поскольку в текущем запросе~\ref{lst:ships_with_ru_op} 
мы не~получаем в~явном виде никаких операторов судов. 




\newpage
\section{Полнота Викиданных}

Поиск точного количества кораблей в мире~--- трудная задача. 
Ведь данные о~некоторых из~них являются совершенно секретными, 
другие~--- частные судна, информации о~них тоже нет. 
Предположим, что общее число кораблей примерно равно \num{1600000}, 
как указано в базе данных судов~\autocite{FleetMon}. 
На 2021 год Викиданные содержали только 71206 кораблей, 
что составило 4,5\,\% от~их~общего числа.


Что касается российских кораблей, 
то~в~состав российский военных и гражданских флотов входит \num{17657} кораблей~\autocite{RussianShips}. 
В это же время запрос~\ref{lst:ships_with_ru_op} возвращает лишь 578 кораблей, 
что составляет 3,27\,\% от общего числа российских кораблей. 

В обоих случаях разница между фактическим количеством кораблей и результатом запросов огромная, что говорит о неполноте Викиданных.

\marginnote{%
    \label{question:ship_2}
    \MarginQuestion
    На марке, выпущенной в 1982 году, изображён самый известный советский эсминец \ruwiki{vgC}{проекта~7}, 
    удостоенный звания <<Гвардейский>>. Назовите его.
    
    \vspace{4pt}
    \includegraphics[width=0.93\linewidth]{chapter/ship/Secret_Grem_ship.jpg}

    Ответ на с.~\pageref{answer:ship_2}.\\
}





\section{Полнота свойств объектов военных кораблей}

Составим список кораблей, участвовавших в каких-либо конфликтах, 
с~помощью запроса~\ref{lst:ships_in_conflict_ru}.

\begin{lstlisting}[ 
        language=SPARQL, 
        caption={{\href{https://w.wiki/vum}{Список кораблей, участвовавших в каких-либо конфликтах}}\protect\footnotemark}, 
        label=lst:ships_in_conflict_ru, 
        numbers=none,
        ]
# List of ships with countries and war conflicts
SELECT ?ship ?shipLabel ?countryLabel ?conflict ?conflictLabel
WHERE
{
  ?ship wdt:P31 wd:Q11446;        # instance of ship
        wdt:P137/wdt:P17 ?country;# belongs to country
        wdt:P607 ?conflict.       # engaged in some conflict
  SERVICE wikibase:label { bd:serviceParam wikibase:language "ru, en" }
}
\end{lstlisting}
\footnotetext{Получено: 1400 кораблей в 2017 году, 3567~--- в 2021 году. Ссылка на~SPARQL-запрос: \href{https://w.wiki/vum}{https://w.wiki/vum}.}



\newpage
У военных кораблей, участвовавших в сражениях, 
указывается свойство 
\href{https://www.wikidata.org/wiki/Property:P607}{conflict (P607)} (война/сражение). 
В~то~же время военные конфликты и военные операции, 
которые являются частью войн,~--- это разные понятия. 
Корабли на Викиданных можно условно поделить на два типа:

\begin{enumerate}
  \item Корабли, у которых военные операции объединены с военными конфликтами. 
      Например, у~\href{https://www.wikidata.org/wiki/Q4148613}{эсминца Гремящий (Q4148613)} 10 войн/сражений, 
        см. запрос~\ref{lst:grem_wars}. 
        Такое большое число связано с тем, 
        что корабль принял участие во многих 
        \href{https://ru.wikipedia.org/wiki/Арктические_конвои}{арктических конвоях}, 
        которые являются военными операциями.

  \item Корабли, у которых военные операции отделены от военных конфликтов. Например, у британского крейсера \href{https://www.wikidata.org/wiki/Q1565575}{HMS Trinidad (Q1565575)} участие в военной кампании и арктическом конвое указаны как часть Второй мировой войны с помощью квалификатора \href{https://www.wikidata.org/wiki/Property:P1012}{including (P1012)}. Таким образом, в Викиданных у этого крейсера указана одна война/сражение.
\end{enumerate}

У кораблей первого типа при поиске по свойству~\wdProperty{607}{conflict} будет отображаться больше войн/сражений, чем у кораблей второго типа. 
%Но в этом случае операция \href{https://ru.wikipedia.org/wiki/Одесская_оборона_(1941)}{Одесская оборона} будет стоять наряду с \href{https://ru.wikipedia.org/wiki/Великая_Отечественная_война}{Великой Отечественной войной}, хотя она является частью этой войны. В такой ситуации выводимые данные будут не точными.

\marginnote{
    \MarginQuestion
    \label{question:ship_3}
    Найдите изображения кораблей, которые были сняты в кинофильмах 
    или описаны в книгах. 

    Ответ на с.~\pageref{answer:ship_3}
}


\begin{lstlisting}[ 
        language=SPARQL, 
        caption={{\href{https://w.wiki/vuo}{Военные конфликты, в которых участвовали 
            \wdqName{эсминец Гремящий}{4148613} и \wdqName{HMS Trinidad}{1565575}}}\protect\footnotemark}, 
            label=lst:grem_wars, 
            numbers=none,
            ]
# List of military conflicts of the two ships 
SELECT ?ship ?shipLabel ?conflict ?conflictLabel
WHERE
{
  VALUES ?ship {wd:Q4148613   # Soviet destroyer Gremyashchiy
                wd:Q1565575}  # United Kingdom's HMS Trinidad
  ?ship wdt:P607 ?conflict.   # conflict
  SERVICE wikibase:label { bd:serviceParam wikibase:language "ru, en" }
}
\end{lstlisting}
\footnotetext{Получено: 10 конфликтов у \href{https://www.wikidata.org/wiki/Q4148613}{эсминца Гремящий (Q4148613)} и один конфликт у крейсера \href{https://www.wikidata.org/wiki/Q1565575}{HMS Trinidad (Q1565575)}, 2021 год. Ссылка на SPARQL-запрос: \href{https://w.wiki/vuo}{https://w.wiki/vuo}.}




\newpage
Составим список отечественных кораблей, 
участвовавших в каких-либо военных конфликтах, 
с~помощью скрипта~\ref{lst:ships_in_war_ru}.

\begin{lstlisting}[ language=SPARQL, 
        caption={{\href{https://w.wiki/wXA}{Список отечественных кораблей, участвовавших в каких-либо военных конфликтах}}\protect\footnotemark}, 
        label=lst:ships_in_war_ru ]
# List of ship with countries and war conflicts
SELECT ?ship ?shipLabel ?countryLabel ?conflict ?conflictLabel
WHERE
{
  VALUES ?country {wd:Q34266 # Russian Empire
                   wd:Q15180 # Soviet Union
                   wd:Q159}  # Russia
  ?ship wdt:P31 wd:Q11446;        # instance of ship
        wdt:P137/wdt:P17 ?country;# belongs to operator
        wdt:P607 ?conflict.       # engaged in some conflict

  SERVICE wikibase:label { bd:serviceParam wikibase:language "ru, en" }
}
\end{lstlisting}
\footnotetext{Получено: 105 кораблей в 2017 году, 82 корабля в 2021 году. 
            Ссылка на SPARQL-запрос: \href{https://w.wiki/wXA}{https://w.wiki/wXA}.}

Особенность результатов скрипта~\ref{lst:ships_in_war_ru} в том, 
что полученные корабли не обязательно связаны только с Российской империей, 
СССР или Россией. 
Например, корабль \href{https://www.wikidata.org/wiki/Q653477}{Kasato Maru (Q653477)}~--- японский. 
Дело в том, что этот корабль принадлежал России в 1900--1905 годах, а Японии~--- с 1906 года.



\section{Корабли-музеи в странах мира}

\wdqName{Корабль-музей}{575727}~--- это корабль, на котором размещена музейная экспозиция, посвященная истории корабля. Такие корабли используются в общеобразовательных и мемориальных целях. Участие корабля в \href{https://www.wikidata.org/wiki/Q180684}{военном конфликте (Q180684)} может послужить поводом для создания корабля-музея в память о прошедших событиях. 

Построим граф кораблей-музеев и стран, в которых эти корабли находятся. 
Вершинами графа будут \href{https://www.wikidata.org/wiki/Q6256}{страны (Q6256)} и 
\href{https://www.wikidata.org/wiki/Q575727}{корабли-музеи (Q575727)}. 
Ребро между кораблём и страной означает, что корабль находится в этой стране. 
А ребро между двумя странами означает, 
что эти страны участвовали в одних и тех же конфликтах, число которых равно весу ребра. 
Запрос~\ref{lst:museum_graph} строит граф по описанным выше правилам.

%\begin{minipage}{\linewidth}
\begin{lstlisting}[ language=SPARQL, 
        caption={{\href{https://w.wiki/wz8}{Граф кораблей-музеев и стран, в которых они находятся (фрагмент скрипта)}}\protect\footnotemark}, 
        label=lst:museum_graph ]
#defaultView:Graph    
SELECT ?v1 ?v1Label ?v2 ?v2Label ?edgeLabel ?img 
WHERE {
  {SELECT ?c ?cLabel ?v1 ?v1Label ?v2 ?v2Label (STR(COUNT(?c)) as ?edgeLabel) 
   WHERE
   {
     VALUES ?cTypes 
            {wd:Q180684 # conflict
             wd:Q831663 # military campaign
             wd:Q645883 # military operation
             wd:Q198    # war
            } 
     ?c wdt:P31 ?cTypes.
     ?v1 wdt:P31 wd:Q6256. ?v2 wdt:P31 wd:Q6256. # country
     ?c wdt:P710 ?v1, ?v2. # in war
     FILTER (?v1 != ?v2 && STR(?v1) < STR(?v2)) 
     SERVICE wikibase:label {bd:serviceParam wikibase:language "ru, en"}
  }
  GROUP BY ?c ?cLabel ?v1 ?v1Label ?v2 ?v2Label
  }
 UNION
  {SELECT DISTINCT ?v1 ?v1Label ?v2 ?v2Label ?img
   WHERE
   {
     ?v2 wdt:P31 wd:Q575727. # museum ship
     {?v2 p:P17 [ps:P17 ?v1]} UNION # ?v2 has country ?v1
     {
       ?v2 wdt:P131 ?loc.      # in ?loc
       ?loc p:P17 [ps:P17 ?v1].# ?loc in country ?v1
     } 
     OPTIONAL {?v2 wdt:P18 ?img}
     SERVICE wikibase:label { bd:serviceParam wikibase:language "ru, en"}
   }
 }
}
\end{lstlisting}
%\end{minipage}
\footnotetext{Получено: 117 вершин графа в 2021 году. Ссылка на SPARQL-запрос: \href{https://w.wiki/wz8}{https://w.wiki/wz8}.}


\newpage
%\marginnote[-7.3cm]{%
    При определении \wdqName{страны}{6256} для переменных $v1$ и $v2$ 
    в~строках 26 и 29 запроса~\ref{lst:museum_graph} 
    используется конструкция \texttt{p:/ps:} (точнее: \texttt{\{?v2 p:P17 [ps:P17 ?v1]\}}). 
    Это нужно потому, что у стран в поле 
    \href{https://www.wikidata.org/wiki/Property:P31}{экземпляр (P31)}, 
    кроме значения <<страна>>, могут быть и другие значения. 
    Эта конструкция позволяет перебрать все значения поля P31 и найти значение <<страна>>.
    См. подробности в главе~\ref{ch:RussiaNotCountryPPS} на с.~\pageref{ch:RussiaNotCountryPPS}.

%}\marginnote[-1.7cm]

Из фрагмента графа на рис.~\ref{fig:museum_graph} видно, 
что корабли-музеи по большей части принадлежат Германии, США и Японии. 
Такая <<корреляция>> вполне логична, так как данные страны имеют длительную историю, 
за которую они поучаствовали во многих военных конфликтах. 
Также эти страны имеют выход к морю, что исторически обуславливает наличие у них флота, 
в котором могут найтись корабли для создания музея.

\begin{figure}[h]
  \includegraphics[width=\linewidth]{chapter/ship/museum-graph-Russia-USA.png}
  \caption[Граф стран и кораблей-музеев, 2021 год.]{Фрагмент графа стран, участвовавших в войнах, включает ряд кораблей-музеев России и США. Граф построен по скрипту~\protect\ref{lst:museum_graph} в 2021 году}%
  \label{fig:museum_graph}%
\end{figure}

При просмотре полного графа (и даже на фрагменте, рис.~\ref{fig:museum_graph}) 
видна уникальность научно-исследовательского судна \wdqName{Витязь}{1516653}. 
По Викиданным (запрос~\ref{lst:museum_graph}), 
это единственный музей-корабль России, связывающий Россию с другими странами. 
Изначально этот теплоход был построен и находился на службе в~Германии, 
что отражено в свойстве Витязя 
\href{https://www.wikidata.org/wiki/Property:P137}{operator (P137)}=``Кригсма\-рине'' 
(это название военно-морских сил в Третьем рейхе).

Вершины графа на рис.~\ref{fig:museum_graph} кликабельны. 
Например, можно кликнуть на \href{https://www.wikidata.org/wiki/Q168713}{крейсер Аврора (Q168713)}, 
тогда появятся дополнительные вершины, соответствующие свойствам этого крейсера, см. рис.~\ref{fig:aurora_graph}.

\newpage
\begin{figure}[h]
  \includegraphics[width=\linewidth]{chapter/ship/aurora_graph.jpg}
  \caption[Граф свойств Авроры, 2021 год.]{Граф свойств \href{https://www.wikidata.org/wiki/Q168713}{крейсера Аврора (Q168713)} на 2021 год}%
  \label{fig:aurora_graph}%
\end{figure}






% last figure
%\begin{figure*}[ht]
%  \includegraphics[width=0.25\linewidth]{chapter/ship/red-green-cells_ships_by_country_and_conflict.png}
%  \caption[Список кораблей и конфликтов, в которых они участвовали]{Фрагмент cписка кораблей, связанных с Россией и участвовавших в военных конфликтах, 2017 год. Из списка видно, что больше большая часть кораблей связаны с Россией и СССР, а также со Второй мировой или Великой Отечественной войнами.}%
%  \label{fig:ships_by_country_and_conflict}%
%\end{figure*}
