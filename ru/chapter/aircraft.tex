\chapter{Воздушные суда от мала до велика}
\label{ch:aircraft-chapter}

Глава посвящена исследованию различных свойств воздушных судов на основе базы знаний международного проекта Викиданные. В ходе исследования с помощью SPARQL-запросов, вычисляемых на объектах типа ”Воздушные суда” в Викиданных, получены сведения о всех воздушных судах и их произведённом количестве, также получена диаграмма соотношения количества производителей воздушных судов по странам.В заключении работы дана оценка полноты данных, представленных в Википедии и Викиданных.

Авиационная промышленность является одной из самой крупнейшей отраслью машиностроения в мире. В её задачи входит как разработка так и производство различной воздушной техники. Для того чтобы оценить какие модели воздушных судов являются самыми массовыми мы построим диаграмму по количеству выпущенных судов различных моделей.

На рисунке рис. ~\ref{fig:Number_of_aircraft_produced_ru_2020} видно, что на 2020 год больше всего было выпущено воздушных судов следующих моделей: Piper PA-32 (7842 штук), Piper PA-24 Comanche (4857), Junkers W 34 (3000), Piper J-4 (1251).

%\begin{figure}[h]
%\centering
%	\includegraphics[width=0.7\textwidth]{./chapter/aircraft/Number_of_aircraft_produced_ru.jpg}
%	\caption{Количество выпущенных воздушных судов по моделям, 2020.}
%	\label{fig:Number_of_aircraft_produced_ru_2020}
%\end{figure}


\begin{figure*}[h!]
	\includegraphics{./chapter/aircraft/Number_of_aircraft_produced_ru.jpg}
	\caption{Количество выпущенных воздушных судов по моделям, 2020.}
	\label{fig:Number_of_aircraft_produced_ru_2020}
\end{figure*}

Исходя из полученных данных можно подсчитать, что всего было произведено 24927 единиц воздушной техники.
