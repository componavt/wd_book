\chapter{На каких языках программирования пишутся операционные системы}
\label{ch:operating-sysmets}

В главе исследуется объект Викиданных "операционная система" (operating system) и его свойства. В каждом из разделов представлены задачи, решённые с помощью SPARQL-запросов. В их числе: нахождение экземпляров объекта "операционная система", построение списка операционных систем (ОС) по предку, по времени создания, по языку, на котором написана ОС. Также построена гистограмма, показывающая количество программ, написанных на том или ином языке программирования, и долю того, сколько из них работает под той или иной ОС. У многого программного обеспечения не указан язык программирования, на котором оно разрабатывалось. Для улучшения результатов решения вышеописанных задач отдельные объекты Викиданных были дополнены свойством "язык программирования".