\chapter{Где учатся и кем работают изобретатели языков программирования}
\label{ch:programming languages}

%\footnotetext{
\marginnote{Используемые в SPARQL-запросах объекты:
\begin{itemize}
	\item\href{https://www.wikidata.org/wiki/Q9143}{programming language} - язык программирования
	\item\href{https://www.wikidata.org/wiki/Q899523}{object-based language} - объектно ориентированный язык программирования
\end{itemize}
Используемые в SPARQL-запросах свойства:
\begin{itemize}
	\item\href{https://www.wikidata.org/wiki/Property:P737}{influenced by} - создан под влиянием каких языков программирования
	\item\href{https://www.wikidata.org/wiki/Property:P178}{developer} - кто разработал язык программирования
	\item\href{https://www.wikidata.org/wiki/Property:P275}{copyright license} - какая лицензия распространения и использования
	\item\href{https://www.wikidata.org/wiki/Property:P31}{instance of} - сущность какого объекта
	\item\href{https://www.wikidata.org/wiki/Property:P1195}{file extension} - расширение файлов
	\item\href{https://www.wikidata.org/wiki/Property:P159}{headquarters location} - место расположения штаб-квартиры
	\item\href{https://www.wikidata.org/wiki/Property:P625}{coordinate location} - геокоординаты объекта
	\item\href{https://www.wikidata.org/wiki/Property:P69}{educated at} - где учился объект
	\item\href{https://www.wikidata.org/wiki/Property:P19}{place of birth} - где родился объект
	\item\href{https://www.wikidata.org/wiki/Property:P106}{occupation} - род занятий объекта (профессия)
\end{itemize}
}

В статье исследуются свойства языков программирования на основе базы знаний международного проекта Викиданные. С помощью SPARQL-запросов, вычисляемых на объектах типа "язык программирования"\  в Викиданных, решен ряд задач. Получены перечени всех языков программирования под пермиссивными лицензиями и языков с закрытыми лицензиями и рассчитано их процентное соотношение. Построена пузырьковая диаграмма по количеству форматов файлов исходного кода. Получены карты, отображающие месторасположение учебных заведений и компаний, в которых учились или работали люди, связанные с созданием языков программирования. Построена пузырьковая диаграмма, отображающая профессии людей, причастных к созданию и разработке языков программирования. Получен список всех объектно-ориентированных языков программирования и сделан вывод об исчерпывающей полноте Викиданных относительно них. Проведено сравнение и анализ результатов SPARQL-запросов 2017 года и 2020 года, отмечены основные изменения. 

%%
% Постановка задачи
%%
\section{Постановка задачи}
Исследуем языки программирования, а именно информацию о них в Русской Википедии, Английской Википедии и Викиданных.
Задачи:
\begin{enumerate} 
  \itemПостроить упорядоченный список языков программирования по числу интервик.
  \itemПостроить список языков по числу посещений статей в Русской Википедии.
  \itemПостроить \href{https://en.wikipedia.org/wiki/Directed_acyclic_graph}{направленный ациклический граф} зависимостей языков программирования друг от друга (или найти циклы в зависимостях, если такой граф нельзя построить). См. свойство "\href{https://www.wikidata.org/wiki/Property:P737}{influenced by}
"\  в \href{https://www.wikidata.org/wiki/Q251}{Java}.
\end{enumerate}

Экземпляры объекта "Язык программирования"

\begin{lstlisting}[
	language=SPARQL,
	label=lst:prog_langs,
	caption={\href{https://w.wiki/kCe}{Список языков программирования}\protect\footnotemark},
	texcl 
]
#List of `instances of` "programming language" 
SELECT ?lang ?langLabel
WHERE
{
    ?lang wdt:P31 wd:Q9143. # instances of programming language
    SERVICE wikibase:label { bd:serviceParam wikibase:language "ru"}
}
\end{lstlisting}
\footnotetext{Результатом запроса~\ref{lst:prog_langs} является список всех языков прогрмаммирования. На 2017 год список содержал 732 записи, а на 2020 год число языков программирования увеличилось до \num{1422} записи.}

Наиболее полными и проработанными языками программирования на Викиданных на 2017 год являются: \href{https://www.wikidata.org/wiki/Q251}{Java}, \href{https://www.wikidata.org/wiki/Q28865}{Python}, \href{https://www.wikidata.org/wiki/Q15777}{C}. На 2020 год наиболее проработанными на Викиданных языками программирования являются: \href{https://www.wikidata.org/wiki/Q2407}{C++} (26 свойств), \href{https://www.wikidata.org/wiki/Q251}{Java} (26 свойств), \href{https://www.wikidata.org/wiki/Q2005}{JavaScript} (25 свойств), \href{https://www.wikidata.org/wiki/Q206904}{R} (25 свойств).
Почти пустыми и малоинформативными языками на 2017 год оказались: \href{https://www.wikidata.org/wiki/Q165372}{CLIPS}, \href{https://www.wikidata.org/wiki/Q1268744}{Dylan}, \href{https://www.wikidata.org/wiki/Q3109515}{Go!}.
Недостаток полученного списка в том, что ряд объектов получился безымянным на Викиданных (No label defined). Попробуем получить список языков, у которых поле "label"\  будет непустым.

\label{question:prog_lang_1}
\marginnote{
Соотнесите язык программирования и его разработчика.
	\begin{tabular}{ll}
		Разработчик & Язык\\
		\hline
		J. Ichbiah & \href{https://www.wikidata.org/wiki/Q154755}{Ада}\\
		C. Moore & \href{https://www.wikidata.org/wiki/Q275472}{Форт}\\
		J. Armstrong & \href{https://www.wikidata.org/wiki/Q334879}{Erlang}\\
	\end{tabular}
\newline
См. ответ~\ref{answer:prog_lang_1} на с.~\pageref{answer:prog_lang_1}.
}

\begin{lstlisting}[
	language=SPARQL,
	label=lst:labeled_languages,
	caption={\href{https://w.wiki/kCf}{Список языков программирования с заполненным свойством label}\protect\footnotemark},
	texcl
]
#List of `instances of` "programming language"\  only with a label.
SELECT ?item ?item_label
WHERE
{
    ?item wdt:P31 wd:Q9143 # instances of programming language
    ; rdfs:label ?item_label . 

    FILTER (LANG(?item_label) = "ru") . 
}
\end{lstlisting}
\footnotetext{Результатом запроса~\ref{lst:labeled_languages} также будет список языков программирования, но из неё исключены те, для которых не указан параметр label. На 2017 год список содержит 709 записей, на 2020 год в списке 1422 записи. Если в 2017 году на два десятка записей стало меньше, то в 2020 год все языки в списке имеют заполненное поле label.}

%%
% Демонстрация работы с операциями над множества в SPARQL
%%
Вывести все языки программирования, являющиеся открытым программным обеспечением (free software) и/или испытавшие на себе влияние, хотя бы одного из следующих языков программирования: \href{https://en.wikipedia.org/wiki/C_(programming_language)}{Си}, \href{https://ru.wikipedia.org/wiki/Python}{Python}, \href{https://ru.wikipedia.org/wiki/Java}{Java}. При этом разработанные какой-либо из фирм, кроме: \href{https://ru.wikipedia.org/wiki/Sun_Microsystems}{Sun Microsystems}, \href{https://en.wikipedia.org/wiki/Johnson_Space_Center}{Космический центр имени Линдона Джонсона}.

\pagebreak

\begin{lstlisting}[
	language=SPARQL,
	label=lst:free_license_languages,
	caption={\href{https://w.wiki/kCh}{Пример более сложного SPARQL-запроса}\protect\footnotemark},
	texcl
]
SELECT DISTINCT ?item ?item_label
WHERE
{
    ?item wdt:P31 wd:Q9143 # instances of programming language
    ; rdfs:label ?item_label . 

    FILTER (LANG(?item_label) = "ru") . 

    {
      { ?item wdt:P737 wd:Q15777 } UNION # influenced by C
      { ?item wdt:P737 wd:Q28865 } UNION # influenced by Python
      { ?item wdt:P737 wd:Q251   } UNION # influenced by Java
      { ?item wdt:P31  wd:Q341   }
    } MINUS 
  	{ 
      { ?item wdt:P178 wd:Q14647  } UNION # developer Sun Microsystems
      { ?item wdt:P178 wd:Q208371 }       # developer Lyndon Johnson
    }  										    # Space Center
}
\end{lstlisting}
\footnotetext{Рeзультатом запроса~\ref{lst:free_license_languages} является список языков программирования, удовлетворяющих указанным выше требованиям. На 2017 год список содержил 115 записей, на 2020 год - 112 записей.}


%%
% Пермессивные лицензии
%%
Выведем все языки программирования находящиеся под \href{https://en.wikipedia.org/wiki/Permissive_software_license}{пермиссивными лицензиями} (практически не ограничивают свободу действий пользователей ПО и разработчиков).
\footnotetext{Результатом работы запроса~\ref{lst:permessive_license} является список языков программирования, находящихся под пермиссивными лицензиями. На 2017 год список содержал 37 записей, на 2020 год в списке 82 записи. В этот список из 82 "свободных"\  языков попали, например \href{https://ru.wikipedia.org/wiki/CoffeeScript}{CoffeeScript}, \href{https://ru.wikipedia.org/wiki/Go}{Go}, \href{https://ru.wikipedia.org/wiki/Haml}{Haml}.}
\begin{lstlisting}[
	language=SPARQL,
	label=lst:permessive_license,
	caption={\href{https://w.wiki/kD6}{Языки программирования с пермессивными лицензиями}\protect\footnotemark},
	texcl
]
SELECT DISTINCT ?item ?item_label
WHERE
{
    ?item wdt:P31 wd:Q9143 # instances of programming language
    ; rdfs:label ?item_label . 

    FILTER (LANG(?item_label) = "ru") . 
  
      { ?item wdt:P275 wd:Q308915  }  UNION  # license Mozzila Public
      { ?item wdt:P275 wd:Q334661  }  UNION  # license MIT
      { ?item wdt:P275 wd:Q191307  }  UNION  # license BSD
      { ?item wdt:P275 wd:Q6905323 }         # license CC
}
\end{lstlisting}


Рассмотрим соотношение языков с пермиссивной лицензией и языков с проприетарной или закрытыми лицензиями.

\pagebreak

\footnotetext{Рeзультатом запроса~\ref{lst:license_compare} будет значение отношения числа языков программирования со свободной лицензией к числу языков с закрытой лицензией. На 2020 год это значение равно 20\%.}
\label{question:prog_lang_2}
\marginnote[50pt]{
Выберите логотип языка программирования \href{https://www.wikidata.org/wiki/Q513238}{LOLCODE}:
	\begin{tabular}{c}
		\includegraphics[width=2cm]{./chapter/programming_language/task_2_logo_1.PNG}\\
		\includegraphics[width=2cm]{./chapter/programming_language/task_2_logo_2.PNG}\\
		\includegraphics[width=2cm]{./chapter/programming_language/task_2_logo_3.PNG}\\
		\includegraphics[width=2cm]{./chapter/programming_language/task_2_logo_4.PNG}\\
	\end{tabular}
\newline
См. ответ~\ref{answer:prog_lang_2} на с.~\pageref{answer:prog_lang_2}.
}
\begin{lstlisting}[
	language=SPARQL,
	label=lst:license_compare,
	caption={\href{https://w.wiki/kD6}{Рассчёт отношения "свободных"\ языков к "закрытым".}\protect\footnotemark},
	texcl
]
#The script calculates the percentage of programming languages 
#with a free license in relation to languages with a closed license
SELECT (COUNT(?not_free)* 100 / (COUNT(?free)) as ?total) WHERE
{ 
{
    SELECT ?free WHERE 
    {
         ?free wdt:P31 wd:Q9143 # instances of programming language
         ; rdfs:label ?item_label . 

         FILTER (LANG(?item_label) = "ru") . 
  
         { ?free wdt:P275 wd:Q308915  }  UNION  # license Mozzila Public
         { ?free wdt:P275 wd:Q334661  }  UNION  # license MIT
         { ?free wdt:P275 wd:Q191307  }  UNION  # license BSD
         { ?free wdt:P275 wd:Q6905323 }         # license CC
    }
}
UNION
{
    SELECT ?not_free WHERE 
    {
      ?not_free wdt:P31 wd:Q9143 # instances of programming language
      ; rdfs:label ?lang_label . 
      FILTER (LANG(?lang_label) = "ru") .
  
      { ?not_free wdt:P275 wd:Q6165015 } UNION # Java Research License
      { ?not_free wdt:P275 wd:Q218616 } UNION # proprietary software
      { ?not_free wdt:P275 wd:Q3238057 } UNION # proprietary license 
      { ?not_free wdt:P275 wd:Q31202214 } UNION # proprietary software 
      { ?not_free wdt:P275 wd:Q979794 } # Aladdin Free Public License
    }
}
}
\end{lstlisting}

%%
% Количество форматов файлов исходного кода
%%
\section{Количество форматов файлов исходного кода}
В зависимости от языка программирования, файлы с исходным кодом программ могут иметь разные расширения. Построим пузырьковую диаграмму по количеству допустимых форматов файлов исходного кода и сравним с аналогичной диаграммой, построенноый в 2017 году.

\pagebreak

\begin{lstlisting}[
	language=SPARQL,
	label=lst:source_formats,
	caption={\href{https://w.wiki/kf7}{Количество форматов файлов исходного кода.}\protect\footnotemark},
	texcl
]
#defaultView:BubbleChart
SELECT ?lang_name (count(*) as ?count)
WHERE
{
    ?lang wdt:P31 wd:Q9143. # instance of programming language
  	?lang wdt:P1195 ?count. # file extension
  	?lang rdfs:label ?lang_name.
    filter (lang(?lang_name) = "ru").
}

GROUP BY ?lang_name 
ORDER BY DESC(?count)
\end{lstlisting}

На рисунке рис.~\ref{fig:source_format_2017} видно, что на 2017 год самыми исторически богатыми на форматы и расширения файлов являлись такие языки программирования, как: \href{https://en.wikipedia.org/wiki/C++}{C++} (10 форматов), \href{https://en.wikipedia.org/wiki/Geometric_Description_Language}{Geometric Description Language} (8 форматов), \href{https://en.wikipedia.org/wiki/Racket_(programming_language)}{Racket} (7 форматов). Причем стоит заметить, что соотношение один язык программирования - один формат файлов исходного кода не является верным. Например, файлы с программой на языке \href{https://en.wikipedia.org/wiki/Racket_(programming_language)}{Racket} могут иметь расширения rkt, rktl, rktd, scrbl, plt, ss или scm.

\begin{figure}
\centering
	\includegraphics[width=0.8\linewidth]{./chapter/programming_language/File_extensions_quantity_of_source_code_2017.png}
	\label{fig:source_format_2017}
	\caption{Пузырьковая диаграмма по количеству форматов файлов исходного кода на 2017 год.}
\end{figure}
К 2020 году (рис.~\ref{fig:source_format_2020}) \href{https://en.wikipedia.org/wiki/C++}{C++} и \href{https://en.wikipedia.org/wiki/Geometric_Description_Language}{Geometric Description Language (GDL)} остались на лидирующем месте (10 и 8 форматов-расширений). За три года подтянулись и вошли в первую восьмёрку также такие языки, как \href{https://en.wikipedia.org/wiki/Raku_(programming_language)}{Raku} (9 форматов), \href{https://en.wikipedia.org/wiki/Rexx}{REXX} и \href{https://en.wikipedia.org/wiki/Scratch_(programming_language)}{Scratch} (по 6 форматов), \href{https://ru.wikipedia.org/wiki/Java}{Java} и \href{https://en.wikipedia.org/wiki/Wolfram_Language}{Wolfram Language} (по 5 форматов).
\begin{figure}
\centering
	\includegraphics[width=0.8\linewidth]{./chapter/programming_language/File_extensions_quantity_of_source_code_2020.png}
	\label{fig:source_format_2020}
	\caption{Пузырьковая диаграмма по количеству форматов файлов исходного кода на 2020 год.}
\end{figure}

\footnotetext{Рисунки~\ref{fig:source_format_2017} и ~\ref{fig:source_format_2020} построены при при помощи запроса~\ref{lst:source_formats}.}
Из этого можно сделать вывод, что развитие языков программирвоания продолжается непрерывно, постоянно возникают новые форматы файлов исходного кода, но лидеры в этой области не спешат сдавать свои позиции.

%%
% Страны, в которых живут люди и располагаются организации, связанные с созданием языков программирования
%%
\section{Страны, в которых живут люди и располагаются организации, связанные с созданием языков программирования}

Отобразим на карте страны, в которых живут люди и располагаются организации, связанные с созданием языков программирования. Заметим, что разработчиком языка может выступать как организация, так и отдельных человек. Для определения месторасположения (свойство: \href{https://www.wikidata.org/wiki/Property:P625}{coordinate location}) организации будем использовать координаты её штаб-квартиры (свойство: \href{https://www.wikidata.org/wiki/Property:P159}{headquarters location}), для человека - координаты места его рождения (свойство: \href{https://www.wikidata.org/wiki/Property:P19}{place of birth}).

\footnotetext{Результатом запроса~\ref{lst:countries_map} является карта, где красными точками указаны места проживания людей, причастных к разработке языков программирования. На рисунке рис.~\ref{fig:countries_2017} изображен результат запроса на 2017 год, а на рис. ~\ref{fig:countries_2020} показаны аналогичные данные на 2020 год.}
\begin{lstlisting}[
	language=SPARQL,
	label=lst:countries_map,
	caption={\href{https://w.wiki/kD9}{Карта с указанием места жительства или работы разработчиков языков программирования}\protect\footnotemark},
	texcl
]
#defaultView:Map
SELECT ?item_label ?developer_label ?location_label ?coord
WHERE
{
    ?item wdt:P31 wd:Q9143 # instances of programming language
    ; rdfs:label ?item_label.     
    FILTER (LANG(?item_label) = "ru"). 
  
    ?item wdt:P178 ?developer. # developer
    ?developer rdfs:label ?developer_label. 
    FILTER (LANG(?developer_label) = "ru"). 
      		
    { ?developer wdt:P159 ?location. } UNION # headquarters location
    { ?developer wdt:P19  ?location  }       # place of birth
    ?location rdfs:label ?location_label. 
    FILTER (LANG(?location_label) = "ru").
    
    ?location wdt:P625 ?coord. # coordinate location

    SERVICE wikibase:label {
        bd:serviceParam wikibase:language "ru".
    }   	
}
\end{lstlisting}

\begin{figure}[h]
\centering
	\includegraphics[width=1\textwidth]{./chapter/programming_language/Map_showing_contries_2017.png}
	\caption{Страны, в которых живут люди и организации, связанные с созданием языков программирования (2017).}
	\label{fig:countries_2017}
\end{figure}
\begin{figure}
\centering
	\includegraphics[width=1\textwidth]{./chapter/programming_language/Map_showing_contries_2020.png}
	\caption{Страны, в которых живут люди и организации, связанные с созданием языков программирования (2020).}
	\label{fig:countries_2020}
\end{figure}

\pagebreak

\begin{marginfigure}
{
\setlength{\fboxsep}{0pt}%
\setlength{\fboxrule}{1pt}%
\fcolorbox{white}{white}{\includegraphics[width=\linewidth]{./chapter/programming_language/The_most_favorable_countries_for_the_emergence_of_people_capable_of_developing_programming_languages_2020_RU.png}}
}
  \caption{Наиболее благоприятные страны для появления людей, способных к разработке языков программирования на 2020 год.}%
  \label{fig:countries_2_2020}%
\end{marginfigure}
По  рис. ~\ref{fig:countries_2017} и рис. ~\ref{fig:countries_2020} можно сделать вывод, что наиболее благоприятными местами жительства для людей, разрабатывающих языки программирования являются восточное побережье \href{https://en.wikipedia.org/wiki/USA}{США}, центральная европа и Великобритания.

Построим также пузырьковую диаграмму, чтобы выявить наиболее благоприятные страны для появления людей, способных к разработке языков программирования и размещению в этих странах штаб-квартир. Видим на рисунке, что наиболее благоприятными странами оказались \href{https://en.wikipedia.org/wiki/USA}{США} (159 человек и штаб квартир) и \href{https://ru.wikipedia.org/wiki/Великобритания}{Великобритания} (15). В \href{https://en.wikipedia.org/wiki/Russia}{России} было разработано только два языка программирования: \href{https://www.wikidata.org/wiki/Q2626418}{РЕФАЛ} и \href{https://www.wikidata.org/wiki/Q65065977}{Встроенный язык программирования 1С:Предприятие}.

На 2020 год (рис. ~\ref{fig:countries_2_2020}) число штаб-квартир в США равно 241, в Великобритании — 24, в Франции — 18, а в России — 5.

%%
% Университеты, в которых учились люди, разрабатывавшие языки программирования
%%
\section{Университеты, в которых учились люди, разрабатывавшие языки программирования}
Отобразим на карте учебные заведения, в которых учились студенты, впоследствии разработавшие языки программирования.

\footnotetext{Результатом запроса~\ref{lst:developer_university} будет карта, на которой красными точками отмечены места расположения университетов, в которых учились люди, создавшие языки программирования. На 2017 год получено 142 записи, к 2020 число записей увеличилось до 282.}
\begin{lstlisting}[
	language=SPARQL,
	label=lst:developer_university,
	caption={\href{https://w.wiki/kDY}{Карта с указанием университетов, в которых учились разработчики языков программирования}\protect\footnotemark},
	texcl
]
#defaultView:Map
SELECT ?item_label ?developer_label ?educational_institution_label ?coord
WHERE
{
    ?item wdt:P31 wd:Q9143 # instances of programming language
    ; rdfs:label ?item_label. 
    FILTER (LANG(?item_label) = "ru"). 
    
    ?item wdt:P178 ?developer. # developer
    ?developer rdfs:label ?developer_label. 
    FILTER (LANG(?developer_label) = "ru"). 
    	
    ?developer wdt:P69 ?educational_institution. # educated at
    ?educational_institution rdfs:label ?educational_institution_label. 
    FILTER (LANG(?educational_institution_label) = "ru").
    
    ?educational_institution wdt:P625 ?coord. # coordinate location
    
    SERVICE wikibase:label {
        bd:serviceParam wikibase:language "ru".
    } 	
}
\end{lstlisting}

\begin{figure}[h]
\centering
	\includegraphics[width=1\textwidth]{./chapter/programming_language/Map_showing_educational_institutes_2017.png}
	\caption{Учебные заведения, в которых учились люди, создававшие языки программирования (2017).}
	\label{fig:universities_2017}
\end{figure}
\begin{figure}[h]
\centering
	\includegraphics[width=1\textwidth]{./chapter/programming_language/Map_showing_educational_institutes_2020.png}
	\caption{Учебные заведения, в которых учились люди, создававшие языки программирования (2020).}
	\label{fig:universities_2020}
\end{figure}


По картам~\ref{fig:universities_2017} и~\ref{fig:universities_2020} видно, что большая часть людей, причастных к созданию языков программирования, учились в Европе или в США и динамика не сильно изменилась за 3 года.

Построим также пузырьковую-диаграмму (запрос~\ref{lst:developer_universities_bubble}) по самым популярным учебным заведениям, среди будущих создателей языков программирования. На первых местах оказались: \href{https://www.wikidata.org/wiki/Q21578}{Принстонский университет} (8 студентов) и \href{https://www.wikidata.org/wiki/Q41506}{Стэнфордский университет} (8). МГУ оказался в конце списка, в нем учился \href{https://www.wikidata.org/wiki/Q92602}{Энтони Ричард Хоар}, разработавший \href{https://www.wikidata.org/wiki/Q188436}{ALGOL60}, и \href{https://www.wikidata.org/wiki/Q4466506}{Валентин Фёдорович Турчин}, разработавший \href{https://www.wikidata.org/wiki/Q2626418}{РЕФАЛ}. \href{https://www.wikidata.org/wiki/Q13164}{МГУ} попал в этот список, включающий 142 вуза мира.

\begin{lstlisting}[
	language=SPARQL,
	label=lst:developer_universities_bubble,
	caption={\href{https://w.wiki/kDb}{Университеты, у которых учились разработчики языков программирования}\protect\footnotemark},
	texcl
]
#defaultView:BubbleChart
SELECT ?educational_institution_label (count(*) as ?count)
WHERE
{
 ?item wdt:P31 wd:Q9143 # instances of programming language
 ; rdfs:label ?item_label. 
 FILTER (LANG(?item_label) = "en"). 
 
 ?item wdt:P178 ?developer. # developer
 ?developer rdfs:label ?developer_label. 
 FILTER (LANG(?developer_label) = "en"). 
 	
 ?developer wdt:P69 ?educational_institution. # educated at
 ?educational_institution rdfs:label ?educational_institution_label. 
 FILTER (LANG(?educational_institution_label) = "en").
 
 ?educational_institution wdt:P625 ?coord. # coordinate location
 
 SERVICE wikibase:label {
 bd:serviceParam wikibase:language "en".
 } 	
}
GROUP BY ?educational_institution_label
ORDER BY DESC(?count)
\end{lstlisting}

%%
% Профессии создателей языков программирования
%%
\section{Профессии создателей языков программирования}

\begin{marginfigure}[-302pt]
{
\setlength{\fboxsep}{0pt}
\setlength{\fboxrule}{1pt}
\fcolorbox{white}{white}{\includegraphics[width=\linewidth]{./chapter/programming_language/Bubble_chart_showing_the_quantity_of_professions_people_,creating_programming_languages,_have_2017.png}}
}
  \caption{Профессии людей, которые разрабатывают языки программирования (2017).}
  \label{fig:2017_profession}
\end{marginfigure}
\begin{marginfigure}[-52pt]
{
\setlength{\fboxsep}{0pt}
\setlength{\fboxrule}{1pt}
\fcolorbox{white}{white}{\includegraphics[width=\linewidth]{./chapter/programming_language/Bubble_chart_showing_the_quantity_of_professions_people,_creating_programming_languages_RU_2020.png}}
}
  \caption{Профессии людей, которые разрабатывают языки программирования (2020).}
  \label{fig:2020_profession}
\end{marginfigure}
Построим пузырьковую диаграмму (запрос~\ref{lst:developer_profession}), отображающую какие профессии преобладают среди людей, разрабатывающих языки программирования.

\begin{lstlisting}[
	language=SPARQL,
	label=lst:developer_profession,
	caption={\href{https://w.wiki/kDc}{Профессии создателей языков программирования}\protect\footnotemark},
	texcl
]
#defaultView:BubbleChart
SELECT ?occupation_label (count(*) as ?occupation)
WHERE
{
    ?item wdt:P31 wd:Q9143. # instances of programming language 
    ?item wdt:P178 ?developer. # developer
    ?developer wdt:P106 ?occupation. # occupation
    ?occupation rdfs:label ?occupation_label. 
    FILTER (LANG(?occupation_label) = "ru"). 
}
GROUP BY ?occupation_label 
ORDER BY DESC(?count)
\end{lstlisting}


\footnotetext{Результатом запроса~\ref{lst:developer_profession} в 2017 году является 48 записей, к 2020 году их число увеличилось до 74.}

Результаты запросов можно выдеть на рисунках~\ref{fig:2017_profession} и~\ref{fig:2020_profession}.
Наиболее распространенными профессиями оказались: специалист в области \href{https://www.wikidata.org/wiki/Q21198}{компьютерных наук}, \href{https://www.wikidata.org/wiki/Q81096}{инженер}, \href{https://www.wikidata.org/wiki/Q37226}{учитель}. Интересно заметить, что встречаются такие профессии как: джазовый музыкант, политик (\href{https://www.wikidata.org/wiki/Q181529}{Герберт Александер Саймон}). На 2020 год среди разработчиков языков программирования оказалось больше всего специалистов в области компьютерных наук (172 человека), а также 96 инженеров, 57 учителей, 56 программистов и 43 математика.

%%
% Объектно-ориентированные языки программирования
%%
\section{Объектно-ориентированные языки программирования}
Вывести список всех объектно-ориентированных языков программирования.

\begin{lstlisting}[
	language=SPARQL,
	label=lst:oopl,
	caption={\href{https://w.wiki/kDe}{Список объектно-ориентированных языков программирования}\protect\footnotemark},
	texcl
]
SELECT DISTINCT ?item ?item_label
WHERE
{
    ?item wdt:P31 wd:Q899523 # instances of object-oriented programming language
    ; rdfs:label ?item_label . 

    FILTER (LANG(?item_label) = "ru") . 
}
\end{lstlisting}

\footnotetext{Результатом SPARQL-запроса~\ref{lst:oopl} является список объектно-ориентированных языков программирования. На 2017 год список содержал 116 записей, к 2020 году число записей увеличилось на 2 и составляет 118 языков программирования.}

Таким образом 8\% языков программирования на 2020 год являются объектно-ориентированными.

%%
% Полнота викиданных
%%
\section{Полнота Викиданных}
По данным Боровского исследовательского университета\cite{link_1} существует как минимум 26 языков программирования, которые поддерживают объектно-ориентированную парадигму. В статьях посвященных объектно-ориентированному программированию к этому списку добавляются ещё 4\cite{link_2} и 3\cite{link_3} языка программирования. 

\begin{lstlisting}[
	language=SPARQL,
	label=lst:fullness,
	caption={\href{https://w.wiki/kDf}{Смотрим на полноту викиданных}\protect\footnotemark},
	texcl
]
#2017-4
SELECT DISTINCT ?item ?item_label
WHERE
{
 ?item wdt:P31 wd:Q899523 # instances of object-oriented programming language
 ; rdfs:label ?item_label . 

 FILTER (LANG(?item_label) = "en")
}
\end{lstlisting}
\footnotetext{SPARQL-запрос~\ref{lst:fullness} вернул 119 результатов.}

Судить о полноте данных в трех приведенных выше источниках сложно, так как большое количество малоизвестных, устаревших и узконаправленных языков, которые не освещаются в авторитетных источниках. Из этого можно сделать вывод, Викиданные предоставляют достаточно полный список объектно-ориентированных языков программирования.

%%
% Степень заполненности объектов
%%
\section{Степень заполненности объектов}

\label{question:prog_lang_3}
\marginnote{
Заполните пропуски.\newline
\href{https://www.wikidata.org/wiki/Q83303}{Фортран} находится на первом месте по количеству своих диалектов. Их число достигает порядка \underline{\hspace{1cm}}. На втором месте \href{https://www.wikidata.org/wiki/Q132874}{Лисп} - \underline{\hspace{1cm}} диалектов. Третье место делят между собой \href{https://www.wikidata.org/wiki/Q597330}{Standard ML} и \href{https://www.wikidata.org/wiki/Q633894}{Object Pascal} с \underline{\hspace{1cm}} диалектами.\newline
См. ответ~\ref{answer:prog_lang_3} на с.~\pageref{answer:prog_lang_3}.
}

Выведем список всех людей, которые связаны с разработкой языков программирования и у объектов которых заполнено поле "label"\  на английском языке:

\begin{lstlisting}[
	language=SPARQL,
	label=lst:filling,
	caption={\href{https://w.wiki/kDg}{Список разработчиков языков программирования с заполненным полем label}\protect\footnotemark},
	texcl
]
#2017-05
SELECT ?item_label ?item ?developer_label ?developer
WHERE
{
    ?item wdt:P31 wd:Q9143 # instances of programming language
    ; rdfs:label ?item_label. 
    FILTER (LANG(?item_label) = "ru"). 

    ?item wdt:P178 ?developer. # developer 
    ?developer wdt:P31 wd:Q5.  # instances of human
    ?developer rdfs:label ?developer_label. 
    FILTER (LANG(?developer_label) = "ru").  
}
\end{lstlisting}
\footnotetext{По результату запроса~\ref{lst:filling} в на 21 мая 2017 года было получено 133 записи, в 2020 году получено 223 записи.}
Выведем аналогичный список, но с заполненным полем "label"\  на русском языке. Таких записей 88. Заполним поля "label"\  и "description"\  на русском языке у этих объектов и выведем результат:

\begin{lstlisting}[
	language=SPARQL,
	label=lst:filling_label,
	caption={\href{https://w.wiki/kDj}{Список разработчиков языков программирования с заполненным полем label и description}\protect\footnotemark},
	texcl
]
#2017-05
SELECT ?item_label ?item ?developer_label ?developer
WHERE
{
    ?item wdt:P31 wd:Q9143 # instances of programming language
    ; rdfs:label ?item_label. 
    FILTER (LANG(?item_label) = "ru"). 

    ?item wdt:P178 ?developer. # developer 
    ?developer wdt:P31 wd:Q5.  # instances of human
    ?developer rdfs:label ?developer_label. 
    FILTER (LANG(?developer_label) = "ru").  
}
\end{lstlisting}
\footnotetext{В 2017 году запрос~\ref{lst:filling_label} вернул 133 значения, в 2020 году выполнение запроса дало 183 записи.}

%%
% Упражнения
%%
\section{Упражнения}
\label{prog_lang_test}
\begin{enumerate}
	\itemВывести все языки программирования с свойством "\href{https://www.wikidata.org/wiki/Property:P822}{персонаж-талисман}".
	\itemПосчитать количество языков программирования основанных раньше 1992 года (свойство: "\href{https://www.wikidata.org/wiki/Property:P571}{дата-основания/создания}").
	\itemПостроить столбчатую диаграмму, отражающую количество известных хештегов в Твиттере для каждого языка программирования (свойство: "\href{https://www.wikidata.org/wiki/Property:P2572}{хештег Твиттера}").
\end{enumerate}

Ответы можно найти на странице~\pageref{answer:prog_langs_4}.