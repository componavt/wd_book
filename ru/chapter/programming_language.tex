\chapter{Где учатся и кем работают изобретатели языков программирования}
\label{ch:programming languages}

В главе исследуются свойства языков программирования на основе Викиданных. 
С помощью SPARQL-запросов, вычисляемых на объектах типа <<язык программирования>>, решён ряд задач. 
Получены перечени всех языков программирования под пермиссивными лицензиями 
и~языков с закрытыми лицензиями и рассчитано их процентное соотношение. 
Построена пузырьковая диаграмма по количеству разных расширений файлов для одного языка программирования. 
Получены карты, отображающие месторасположение учебных заведений и компаний, 
в которых учились или работали люди, связанные с созданием языков программирования. 
Построена пузырьковая диаграмма, отображающая популярные профессии среди людей, 
причастных к созданию и разработке языков программирования. 
Получен список всех объектно-ориентированных языков программирования 
и~сделан вывод об исчерпывающей полноте Викиданных относительно них. 
Проведено сравнение и анализ результатов SPARQL-запросов за разные годы, отмечены основные изменения. 

%\footnotetext{
%\marginnote{
    Перечислим используемые в этой главе объекты Викиданных:
\begin{itemize}
	\item\href{https://www.wikidata.org/wiki/Q9143}{programming language}~--- язык программирования;
	\item\href{https://www.wikidata.org/wiki/Q899523}{object-based language}~--- объектно-ориентированный язык программирования.
\end{itemize}
И используемые в SPARQL-запросах свойства объектов Викиданных:
\begin{itemize}
	\item\href{https://www.wikidata.org/wiki/Property:P737}{influenced by}~--- какие языки программирования оказали влияние;
	\item\href{https://www.wikidata.org/wiki/Property:P178}{developer}~--- кто разработал язык программирования;
	\item\href{https://www.wikidata.org/wiki/Property:P275}{copyright license}~--- какая лицензия;
	\item\href{https://www.wikidata.org/wiki/Property:P31}{instance of}~--- к какому более общему типу (типам) относится этот язык;
	\item\href{https://www.wikidata.org/wiki/Property:P1195}{file extension}~--- расширение файлов;
	\item\href{https://www.wikidata.org/wiki/Property:P159}{headquarters location}~--- место расположения штаб-квартиры разработчиков языка;
	\item\href{https://www.wikidata.org/wiki/Property:P625}{coordinate location}~--- географические координаты объекта;
	\item\href{https://www.wikidata.org/wiki/Property:P69}{educated at}~--- где учился объект;
	\item\href{https://www.wikidata.org/wiki/Property:P19}{place of birth}~--- где родился объект;
	\item\href{https://www.wikidata.org/wiki/Property:P106}{occupation}~--- род занятий объекта (профессия).
\end{itemize}
%}



%%
% Список языков программирования
%%
\section{Список языков программирования}
%
\marginnote{
    \label{question:prog_lang_1}
    \MarginQuestion
    Соотнесите язык программирования и его разработчика.

    \vspace{3pt}
	\begin{tabular}{ll}
        Язык & Разработчик\\
		\hline
        \href{https://www.wikidata.org/wiki/Q154755}{Ада} & \href{https://ru.wikipedia.org/wiki/Ишбиа,_Жан}{Жан Ишбиа}\\
        \href{https://www.wikidata.org/wiki/Q275472}{Форт} & \href{https://ru.wikipedia.org/wiki/Мур,_Чарльз_(программист)}{Чарльз Мур}\\
        \href{https://www.wikidata.org/wiki/Q334879}{Erlang} & \href{https://ru.wikipedia.org/wiki/Армстронг,_Джо_(программист)}{Джо Армстронг}\\
	\end{tabular}

    \vspace{3pt}
    См. ответ~\ref{answer:prog_lang_1} на с.~\pageref{answer:prog_lang_1}.
}
%
Выведем (запрос~\ref{lst:prog_langs}) список всех языков программирования, 
с которыми будем работать дальше. 
Для этого используем объект \wdqName{programming language}{9143} 
и свойство \href{https://www.wikidata.org/wiki/Property:P31}{instance of (P31)}.

\begin{lstlisting}[
	language=SPARQL,
	label=lst:prog_langs,
	caption={\href{https://w.wiki/uGs}{Список языков программирования}\protect\footnotemark},
    numbers=none,
	texcl 
]
#List of programming languages
SELECT ?lang ?langLabel WHERE
{
    ?lang wdt:P31 wd:Q9143. # instances of programming language
    SERVICE wikibase:label { bd:serviceParam wikibase:language "ru"}
}
\end{lstlisting}
\footnotetext{Получено: 732 языка программирования в~2017 году, 1422 в~2020 году. Ссылка на SPARQL-запрос: \href{https://w.wiki/uGs}{https://w.wiki/uGs}.}

На 2020 год наиболее проработанными на Викиданных языками программирования были: \href{https://www.wikidata.org/wiki/Q2407}{C++} (26 свойств), \href{https://www.wikidata.org/wiki/Q251}{Java} (26 свойств), \href{https://www.wikidata.org/wiki/Q2005}{JavaScript} (25 свойств), \href{https://www.wikidata.org/wiki/Q206904}{R} (25 свойств).
Почти пустыми и малоинформативными языками на 2020 год оказались\autocite{prowd_langs_link}: 
\href{https://www.wikidata.org/wiki/Q3991643}{Tiny}, 
\href{https://www.wikidata.org/wiki/Q3924253}{Proteus}, 
\href{https://www.wikidata.org/wiki/Q21524853}{Comfy}~--- всего по одному свойству.
Недостаток полученного списка в том, 
что ряд объектов получился безымянным на Викиданных (No label defined). 
Попробуем получить список языков, у которых поле \lstinline|label| на~русском языке будет непустым (запрос~\ref{lst:labeled_languages}).




\newpage
\begin{lstlisting}[
	language=SPARQL,
	label=lst:labeled_languages,
	caption={\href{https://w.wiki/v2a}{Список языков программирования с заполненным свойством label на~русском языке}\protect\footnotemark},
    numbers=none,
	texcl
]
SELECT ?lang ?lang_label WHERE {
    ?lang wdt:P31 wd:Q9143
    ; rdfs:label ?lang_label FILTER (LANG(?lang_label) = "ru") . 
}
\end{lstlisting}
\footnotetext{Получено: 630 языков программирования с заполненным свойством label на~русском языке в~2020 году. 
              Ссылка на SPARQL-запрос: \href{https://w.wiki/v2a}{https://w.wiki/v2a}.}




%%
\subsection{Операции над множествами в SPARQL}
%%
Получим список всех языков программирования, 
являющихся открытым программным обеспечением (free software) 
и испытавших на себе влияние хотя бы одного из следующих языков программирования: 
\href{https://en.wikipedia.org/wiki/C_(programming_language)}{Си}, 
\href{https://ru.wikipedia.org/wiki/Python}{Python}, 
\href{https://ru.wikipedia.org/wiki/Java}{Java}. 
При этом пусть в разработке этих языков не участвуют следующие две фирмы: 
\href{https://ru.wikipedia.org/wiki/Sun_Microsystems}{Sun Microsystems}, 
\href{https://en.wikipedia.org/wiki/Johnson_Space_Center}{Космический центр имени Линдона Джонсона}. 
Таким образом, из множества <<открытых>> языков 
нужно выбрать множество языков, окружающих Си, Python, Java, 
и вычесть языки двух фирм. 
То есть, действительно, в итоговом запросе~\ref{lst:free_license_languages} 
нужно выполнить операции над множествами с~помощью команд \texttt{UNION} и \texttt{MINUS}. 


\index{SPARQL!MINUS!Список свободных языков программирования на русском языке}
\begin{lstlisting}[
	language=SPARQL,
	label=lst:free_license_languages,
	caption={\href{https://w.wiki/v2n}{Список свободных языков программирования}\protect\footnotemark},
    numbers=none,
	texcl
]
SELECT DISTINCT ?prog ?progLabel WHERE
{
  ?prog wdt:P31 wd:Q9143 # instance of programming language
  SERVICE wikibase:label { bd:serviceParam wikibase:language "ru" } 
  {
    { ?prog wdt:P737 wd:Q15777 } UNION # influenced by C
    { ?prog wdt:P737 wd:Q28865 } UNION # influenced by Python
    { ?prog wdt:P737 wd:Q251 } UNION # influenced by Java
    { ?prog wdt:P31 wd:Q341 } # is free software
  } MINUS
  { #developer
    { ?prog wdt:P178 wd:Q14647 } UNION # developer Sun Microsystems
    { ?prog wdt:P178 wd:Q208371 } # developer Lyndon Johnson Space Center
  }
}
\end{lstlisting}
\footnotetext{%
        Получено: в~2017 году 115 таких языков, в~2020 году~--- 125 языков. 
        Ссылка на~SPARQL-запрос: \href{https://w.wiki/v2n}{https://w.wiki/v2n}.}



%%
\subsection{Пермиссивные лицензии}
\index{Языки программирования!Определения!Пермиссивные лицензии}%
%%
С помощью запроса~\ref{lst:permessive_license} 
построим список языков программирования, 
находящихся под \href{https://en.wikipedia.org/wiki/Permissive_software_license}{пермиссивными лицензиями}, 
то есть такими лицензиями на свободное программное обеспечение, 
которые почти не ограничивают свободу действий пользователей и разработчиков. 
%
\marginnote{
    \MarginQuestion
    Подсчитайте, больше ли пишут программ под пермиссивными лицензиями те, 
    кто выбирает для работы язык с пермиссивной лицензией?
}

%\begin{marginfigure}[-22\baselineskip]
\begin{lstlisting}[
	language=SPARQL,
	label=lst:permessive_license,
    caption={\href{https://w.wiki/5CLF}{Языки программирования с пермиссивными лицензиями\protect\footnotemark}},
    numbers=none,
	texcl
]
SELECT DISTINCT ?lang ?langLabel WHERE
{
  ?lang wdt:P31 wd:Q9143. # is programming language
  {?lang wdt:P275 wd:Q308915 } UNION # Mozilla Public
  {?lang wdt:P275 wd:Q334661 } UNION # MIT license
  {?lang wdt:P275 wd:Q191307 } UNION # BSD licenses
  {?lang wdt:P275 wd:Q6905323}       # Creative Commons

  SERVICE wikibase:label {bd:serviceParam wikibase:language "ru,en"}.
}
\end{lstlisting}
\footnotetext{%
    Получено: в~2017 году 37 языков программирования, находящихся под пермиссивными лицензиями,  
              в~2020 году~--- 83 языка. 
              В этот список из 83 <<свободных>>  языков попали, например, 
              \href{https://ru.wikipedia.org/wiki/CoffeeScript}{CoffeeScript}, 
              \href{https://ru.wikipedia.org/wiki/Go}{Go}, 
              \href{https://ru.wikipedia.org/wiki/Haml}{Haml}. 
              Ссылка на SPARQL-запрос: \href{https://w.wiki/5CLF}{https://w.wiki/5CLF}.
}
%\end{marginfigure}



Рассмотрим соотношение числа языков с~пермиссивной лицензией 
и языков с~проприетарными (закрытыми) лицензиями.
Запрос~\ref{lst:license_compare} состоит из нескольких частей: 
\begin{enumerate}[%labelindent=\parindent, 
                  font=\sffamily\bfseries, 
                  leftmargin=8em,
                 ]
    \item[строки 5--12:] в первом подзапросе подсчитываем число \texttt{?free} свободных языков программирования;
    \item[строки 14--22:] во втором~--- число \texttt{?not\_free} <<закрытых>> языков;
    \item[строка 3:] вычисляем отношение числа свободных и несвободных языков.
\end{enumerate}


\newpage
\marginnote[26pt]{%
\label{question:prog_lang_2}
    \MarginQuestion
    Приведённые ниже изображения~--- логотипы разных языков программирования. 
    Какое из изображений является логотипом языка \href{https://www.wikidata.org/wiki/Q513238}{LOLCODE}?

    \vspace{5pt}
	\begin{tabular}{c c c c}
        \includegraphics[width=2cm]{./chapter/programming_language/task_2_logo_1.PNG} & 
        \includegraphics[width=2cm]{./chapter/programming_language/task_2_logo_2.PNG} & 
        \includegraphics[width=2cm]{./chapter/programming_language/task_2_logo_3.PNG} & 
		\includegraphics[width=2cm]{./chapter/programming_language/task_2_logo_4.PNG}
	\end{tabular}

    \vspace{5pt}
    См. ответ~\ref{answer:prog_lang_2} на с.~\pageref{answer:prog_lang_2}.
}
%\footnotetext[1???][-3cm]{Todo test
%}
\begin{lstlisting}[
	language=SPARQL,
	label=lst:license_compare,
	caption={\href{https://w.wiki/v3B}{Расчёт отношения числа свободных языков к~числу закрытых}\protect\footnotemark},
	texcl
]
#The script calculates the percentage of programming languages 
#with a free license in relation to languages with a closed license
SELECT (COUNT(?not_free)* 100 / (COUNT(?free)) as ?total) WHERE
{{
    SELECT ?free WHERE {
      ?free wdt:P31 wd:Q9143 # instances of programming language
       SERVICE wikibase:label { bd:serviceParam wikibase:language "ru" }
       { ?free wdt:P275 wd:Q308915  }  UNION  # license Mozilla Public
       { ?free wdt:P275 wd:Q334661  }  UNION  # license MIT
       { ?free wdt:P275 wd:Q191307  }  UNION  # license BSD
       { ?free wdt:P275 wd:Q6905323 }         # license CC
    }
} UNION {
    SELECT ?not_free WHERE {
      ?not_free wdt:P31 wd:Q9143 # instances of programming language
      SERVICE wikibase:label { bd:serviceParam wikibase:language "ru" }
      { ?not_free wdt:P275 wd:Q6165015 } UNION # Java Research License
      { ?not_free wdt:P275 wd:Q218616 } UNION # proprietary software
      { ?not_free wdt:P275 wd:Q3238057 } UNION # proprietary license 
      { ?not_free wdt:P275 wd:Q31202214 } UNION # proprietary software 
      { ?not_free wdt:P275 wd:Q979794 } # Aladdin Free Public License
    }}
}
\end{lstlisting}%
\footnotetext{Запрос подсчитывает отношения числа языков программирования со свободной лицензией к числу языков с закрытой лицензией. 
              На 2020 год это значение равно 26\,\%. 
              Ссылка на SPARQL-запрос: \href{https://w.wiki/v3B}{https://w.wiki/v3B}.}






%%
\section{Количество форматов файлов исходного кода}
%%
В зависимости от языка программирования 
файлы с исходным кодом программ могут иметь разные расширения. 
\index{Языки программирования!Определения!Расширение имени файла}%
Расширение имени файла~--- это последовательность символов, 
добавляемых к имени файла и предназначенных для идентификации типа (формата) файла. 
Это один из распространённых способов, 
с помощью которых пользователь или программное обеспечение компьютера может определить тип данных, 
хранящихся в файле, например: \texttt{file.jpg}~--- это фотография, \texttt{file.avi}~--- видеофайл.


\newpage
\begin{marginfigure}
\centering
	\includegraphics[width=0.74\linewidth]{./chapter/programming_language/File_extensions_quantity_of_source_code_2017.png}
    \caption[Число форматов файлов для языков программирования, 2017 год.]{Пузырьковая диаграмма с числом форматов файлов исходного кода для разных языков программирования, 2017 год. Размер пузырька соответствует числу форматов для одного языка. Ссылка на SPARQL-запрос: \href{https://w.wiki/uGQ}{https://w.wiki/uGQ}}
	\label{fig:source-format-2017}
\end{marginfigure}
%
С помощью запроса~\ref{lst:source_formats} построим пузырьковую диаграмму, 
показывающую долю различных форматов файлов исходного кода. 
Сравним диаграммы, построенные в~2017 году (рис.~\ref{fig:source-format-2017}) 
и в~2020 году (рис.~\ref{fig:source-format-2020}) по этому запросу. 

На рис.~\ref{fig:source-format-2017} видно, что на 2017 год самыми исторически богатыми на форматы и расширения файлов оказались такие языки программирования, как: \href{https://en.wikipedia.org/wiki/C++}{C++} (10 форматов), \href{https://en.wikipedia.org/wiki/Geometric_Description_Language}{Geometric Description Language} (8 форматов), \href{https://en.wikipedia.org/wiki/Racket_(programming_language)}{Racket} (7 форматов). Таким образом, одному языку программирования может соответствовать больше одного расширения имени файла. Например, файлы с программой на языке \href{https://en.wikipedia.org/wiki/Racket_(programming_language)}{Racket} могут иметь расширения rkt, rktl, rktd, scrbl, plt, ss или scm.

\index{График!BubbleChart!Количество форматов файлов исходного кода}
\begin{lstlisting}[
	language=SPARQL,
	label=lst:source_formats,
	caption={\href{https://w.wiki/uGQ}{Количество форматов файлов исходного кода}},
    numbers=none,
	texcl
]
#defaultView:BubbleChart
SELECT ?lang_name (count(*) as ?count) WHERE
{
    ?lang wdt:P31 wd:Q9143. # instance of programming language
 	?lang wdt:P1195 ?count. # file extension
 	?lang rdfs:label ?lang_name FILTER (lang(?lang_name) = "ru,en").
}
GROUP BY ?lang_name 
ORDER BY DESC(?count)
\end{lstlisting}
%\footnotetext{Рис.~\ref{fig:source-format-2017} и~\ref{fig:source-format-2020} построены при при помощи запроса~\ref{lst:source_formats}.}


К 2020 году (рис.~\ref{fig:source-format-2020}) \href{https://en.wikipedia.org/wiki/C++}{C++} и \href{https://en.wikipedia.org/wiki/Geometric_Description_Language}{Geometric Description Language (GDL)} остались на лидирующем месте (10 и 8 форматов-расширений). За три года подтянулись и вошли в первую восьмёрку также такие языки, как \href{https://ru.wikipedia.org/wiki/Racket_(язык_программирования)}{Racket}, \href{https://en.wikipedia.org/wiki/Raku_(programming_language)}{Raku} (9 форматов), \href{https://en.wikipedia.org/wiki/Rexx}{REXX} и \href{https://en.wikipedia.org/wiki/Scratch_(programming_language)}{Scratch} (по 6 форматов), \href{https://ru.wikipedia.org/wiki/Java}{Java} и \href{https://en.wikipedia.org/wiki/Wolfram_Language}{Wolfram Language} (по 5 форматов).

\begin{marginfigure}[2\baselineskip]
\centering
	\includegraphics[width=0.8\linewidth]{./chapter/programming_language/File_extensions_quantity_of_source_code_2020.png}
	\caption[Число форматов файлов для языков программирования, 2020 год.]{Пузырьковая диаграмма с числом форматов файлов исходного кода для 108 языков программирования, 2020 год. Размер пузырька соответствует числу форматов для одного языка. Ссылка на SPARQL-запрос: \href{https://w.wiki/uGQ}{https://w.wiki/uGQ}}
	\label{fig:source-format-2020}
\end{marginfigure}

Из этого можно сделать вывод, что развитие языков программирования продолжается непрерывно, 
постоянно возникают новые форматы файлов исходного кода, 
но лидеры по числу расширений файлов с трудом уступают лидирующие позиции. 
Это может быть связано с тем, что, раз возникнув, язык уже с трудом может избавиться от расширения 
ввиду опасности потери совместимости с более ранними версиями языка. 
Таким образом, наличие множества расширений, скорее, 
указывает на~отсутствие единого стандарта и строгих правил на первых порах.








%%
% Страны, в которых живут люди и располагаются организации, связанные с созданием языков программирования
%%
\section{Страны, в которых располагаются организации и проживают люди, связанные с разработкой языков программирования}

\begin{marginfigure}%[2\baselineskip]
\centering
	\includegraphics[width=1.0\linewidth]{./chapter/programming_language/residence_of_Einstein_2023.png}
	\caption[Место жительства Альберта Эйнштейна.]{
Альберт Эйнштейн жил в городке Капут в Германии с 1929 по 1932 год, данные о месте жительства учёного в~Викиданных, фрагмент страницы объекта \wdqName{Albert Einstein}{937}, 2023 год}
	\label{fig:Einstein-residence}
\end{marginfigure}
%
Отобразим на карте мира те страны, в которых живут люди и располагаются организации, 
связанные с созданием языков программирования. 
Заметим, что разработчиком языка 
(свойство: \href{https://www.wikidata.org/wiki/Property:P178}{developer (P178)}, 
см.~запрос~\ref{lst:countries_map}) может выступать как организация, 
так и отдельный человек 
(далее по тексту под <<разработчиком>> понимается одно из этих двух понятий). 
Для определения месторасположения 
(свойство: \href{https://www.wikidata.org/wiki/Property:P625}{coordinate location (P625)}) 
организации будем использовать координаты её штаб-квартиры 
(свойство: \href{https://www.wikidata.org/wiki/Property:P159}{headquarters location (P159)}), 
для человека~--- координаты места его рождения 
(свойство: \href{https://www.wikidata.org/wiki/Property:P19}{place of birth (P19)}). 

Более точным подходом было бы использование не только места рождения, но и тех мест, где человек живёт. 
На~Викиданных это описывается свойством~\wdProperty{551}{residence}. 
Трудность в том, что это свойство редакторы Викиданных достаточно редко заполняют, 
сейчас его можно увидеть только у самых известных людей с наиболее проработанными объектами, 
например эти данные заполнены на странице Альберта Эйнштейна (рис.~\ref{fig:Einstein-residence}). 
Поэтому в~запросе~\ref{lst:countries_map} мы ограничились учётом только места рождения. 

\index{График!Map!Карта с указанием места жительства или работы разработчиков языков программирования}
\begin{lstlisting}[
	language=SPARQL,
	label=lst:countries_map,
	caption={\href{https://w.wiki/v3b}{Карта с указанием места жительства или работы разработчиков языков программирования}\protect\footnotemark},
    numbers=none,
	texcl
]
#defaultView:Map
SELECT ?lang_label ?developerLabel ?locationLabel ?coord
WHERE
{
   ?lang wdt:P31 wd:Q9143. # instances of programming language
   ?lang wdt:P178 ?developer. # developer
   { ?developer wdt:P159 ?location. } UNION # headquarters location
   { ?developer wdt:P19 ?location. } # place of birth
   ?location wdt:P625 ?coord. # coordinate location
   SERVICE wikibase:label { bd:serviceParam wikibase:language "ru" } 	
}
\end{lstlisting}
\footnotetext{Результатом запроса является карта, где красными точками указаны места проживания людей или штабы фирм, причастных к разработке языков программирования. На рис.~\ref{fig:countries_2017} изображен результат запроса на 2017 год, а на рис.~\ref{fig:countries_2020} показаны аналогичные данные на 2020 год.}


По рис.~\ref{fig:countries_2017} и \ref{fig:countries_2020} можно сделать вывод, 
что наиболее благоприятными местами для~разработки языков программирования являются 
Восточное побережье \href{https://en.wikipedia.org/wiki/USA}{США}, 
\href{https://ru.wikipedia.org/wiki/Центральная_Европа}{Центральная Европа} и 
\href{https://ru.wikipedia.org/wiki/Великобритания}{Великобритания}.


\begin{figure}[h]
\centering
	\includegraphics[width=1\textwidth]{./chapter/programming_language/Map_showing_contries_2017.png}
	\caption[Страны, в которых живут люди или расположены организации, связанные с созданием языков программирования, 2017 год.]{Страны, в которых живут люди или расположены организации, связанные с созданием языков программирования, 2017 год. Ссылка на SPARQL-запрос: \href{https://w.wiki/v3b}{https://w.wiki/v3b}}
	\label{fig:countries_2017}
\end{figure}
\begin{figure}
\centering
	\includegraphics[width=1\textwidth]{./chapter/programming_language/Map_showing_contries_2020.png}
	\caption[Страны, в которых живут люди или расположены организации, связанные с созданием языков программирования, 2020 год.]{Страны, в которых живут люди или расположены организации, связанные с созданием языков программирования, 2020 год. Ссылка на SPARQL-запрос: \href{https://w.wiki/v3b}{https://w.wiki/v3b}}
	\label{fig:countries_2020}
\end{figure}

\begin{marginfigure}
\includegraphics[width=\linewidth]{./chapter/programming_language/The_most_favorable_countries_for_the_emergence_of_people_capable_of_developing_programming_languages_2020_RU.png}
  \caption[Наиболее благоприятные страны для разработчиков языков программирования, 2020 год.]{Наиболее благоприятные страны для появления людей, способных к разработке языков программирования, 2020 год. Размер пузырька соответствует числу людей в стране, причастных к разработке языков программирования. Ссылка на SPARQL-запрос: \href{https://w.wiki/6fxx}{https://w.wiki/6fxx}}%
  \label{fig:countries_2_2020}%
\end{marginfigure}



С помощью запроса~\ref{lst:countries_bubble} 
построим пузырьковую диаграмму стран  
наиболее благоприятных для~размещения штаб-квартир и/или 
рождения разработчиков языков программирования. 
Видим на~рис.~\ref{fig:countries_2_2020}, 
что такой страной оказались, в первую очередь, 
\href{https://en.wikipedia.org/wiki/USA}{Соединённые Штаты Америки} (241 штаб-квартира и/или человек). 
Далее (около двух дюжин штаб-квартир и/или человек) идут 
Швейцария, Великобритания и Франция.  
В России подобных штаб-квартир и/или родившихся разработчиков языков оказалось 4. 
С помощью модификации запроса~\ref{lst:countries_bubble} получим 
следующий список языков программирования, разработанных в России, 
вместе с разработчиками и штаб-квартирами 
(см. ссылку на SPARQL-запрос: \href{https://w.wiki/6g2c}{https://w.wiki/6g2c}):
\begin{compactitemize}
	\item язык программирования \wdqName{РЕФАЛ}{2626418};  
        первую версию языка придумал в 1968 году В.~Ф.~Турчин, 
        родившийся в Подольске в Московской области; 

    \item язык программирования \wdqName{Аналитик}{4064746} 
        был разработан также в 1968 году в Институте кибернетики АН УССР 
        под руководством В. М. Глушкова; Виктор Михайлович Глушков родился в Ростове-на-Дону;

	\item встроенный язык программирования \wdqName{1С:Предприятие}{65065977} начали разрабатывать в 1996 году, 
        штаб-квартира компании 1C находится в Москве; 

    \item языка программирования \wdqName{Factor}{1391724}
        придуман и разрабатывается С. Пестовым, родившемся в Томске.
\end{compactitemize}


\marginnote[4\baselineskip]{
    \MarginQuestion
    Сколько языков программирования появилось за последние три года? 
    В~каком году было изобретено максимальное число языков?%
}

\index{График!BubbleChart!Пузырьковая диаграмма благоприятных стран для появления разработчиков языков программирования}
\begin{lstlisting}[
	language=SPARQL,
	caption={\href{https://w.wiki/6fxx}{Пузырьковая диаграмма благоприятных стран для появления разработчиков языков программирования}},
	label=lst:countries_bubble,
    numbers=none,
	texcl
]
#defaultView:BubbleChart
SELECT ?stateLabel (count(*) as ?count)
WHERE {
  ?lang wdt:P31 wd:Q9143. # instances of programming language
  ?lang wdt:P178 ?developer. # developer
  	
  { ?developer wdt:P159 ?location. } UNION # headquarters location
  { ?developer wdt:P19 ?location. } # place of birth
  
  ?location wdt:P17 ?state.
  SERVICE wikibase:label { bd:serviceParam wikibase:language "ru" } 	
}
GROUP BY ?stateLabel
ORDER BY DESC(?count)
\end{lstlisting}







%%
\section{Университеты, в которых учились разработчики языков программирования}
%%


Отобразим (запрос~\ref{lst:developer_university}) на карте учебные заведения, в которых учились студенты, впоследствии разработавшие языки программирования.
По рис.~\ref{fig:universities_2017} и~\ref{fig:universities_2020} видно, что большая часть людей, причастных к созданию языков программирования, учились в Европе или в США и динамика не~сильно изменилась за три года.
%
\begin{marginfigure}[3\baselineskip]
\centering
	\includegraphics[width=1\textwidth]{./chapter/programming_language/Map_showing_educational_institutes_2017.png}
	\caption[Учебные заведения, в которых учились разработчики языков программирования, 2017 год.]{Учебные заведения, в которых учились разработчики языков программирования, 2017 год. Ссылка на SPARQL-запрос: \href{https://w.wiki/uGb}{https://w.wiki/uGb}}
	\label{fig:universities_2017}
\end{marginfigure}


\index{График!Map!Карта с указанием университетов, в которых учились разработчики языков программирования}
\begin{lstlisting}[
	language=SPARQL,
	label=lst:developer_university,
	caption={\href{https://w.wiki/v3v}{Карта с указанием университетов, в которых учились разработчики языков программирования}\protect\footnotemark},
    numbers=none,
	texcl
]
#defaultView:Map
SELECT ?langLabel ?developerLabel ?educational_institutionLabel ?coord 
WHERE {
  ?lang wdt:P31 wd:Q9143.    # instances of programming language
  ?lang wdt:P178 ?developer. # developer
  ?developer wdt:P69 ?educational_institution. # educated at
  ?educational_institution wdt:P625 ?coord. # coordinate location
  SERVICE wikibase:label { bd:serviceParam wikibase:language "ru,en" }
}
\end{lstlisting}
\footnotetext{Результатом запроса будет карта, на которой красными точками отмечены места расположения университетов, в которых учились люди, создавшие языки программирования. На 2017 год получено 142 университета, к 2020 число записей увеличилось до 282.}

\begin{marginfigure}[1\baselineskip]
\centering
	\includegraphics[width=1\textwidth]{./chapter/programming_language/Map_showing_educational_institutes_2020.png}
	\caption[Учебные заведения, в которых учились разработчики языков программирования, 2020 год.]{Учебные заведения, в которых учились разработчики языков программирования, 2020 год. Ссылка на SPARQL-запрос: \href{https://w.wiki/uGb}{https://w.wiki/uGb}}
	\label{fig:universities_2020}
\end{marginfigure}



Построим пузырьковую диаграмму (запрос~\ref{lst:developer_universities_bubble}) 
по самым популярным среди будущих создателей языков программирования учебным заведениям. 
На первых местах по числу обучающихся разработчиков оказались: 
\href{https://www.wikidata.org/wiki/Q21578}{Принстонский университет} и 
\href{https://www.wikidata.org/wiki/Q41506}{Стэнфордский университет} (по 8 студентов). 
В~\href{https://ru.wikipedia.org/wiki/Московский_государственный_университет}{Московском государственном университете} 
учились три таких студента 
(см. ссылку на~SPARQL-запрос: \href{https://w.wiki/6gC3}{https://w.wiki/6gC3}):
\begin{itemize}
    \item \wdqName{Энтони Ричард Хоар}{92602}, разработавший язык \wdqName{ALGOL W}{1538458} в 1958 году;

    \item снова В.~Ф.~Турчин, разработавший \wdqName{РЕФАЛ}{2626418};

    \item снова В. М. Глушков, под чьим руководством был разработан 
        язык программирования \wdqName{Аналитик}{4064746}.
\end{itemize}
Этот список университетов, в~который попал МГУ, на 2023 год содержит 191 вуз мира.



\newpage
%\begin{marginfigure}[-12\baselineskip]
\index{График!BubbleChart!Университеты, в которых учились разработчики языков программирования}
\begin{lstlisting}[
	language=SPARQL,
	label=lst:developer_universities_bubble,
    caption={\href{https://w.wiki/5CMt}{Университеты, в которых учились разработчики языков программирования}\protect\footnotemark},
    numbers=none,
	texcl
]
#defaultView:BubbleChart
SELECT ?eduInstitutionLabel (count(*) as ?count) 
WHERE {
  ?lang wdt:P31 wd:Q9143;     # is programming language
        wdt:P178 ?developer.  # developer
  ?developer wdt:P69 ?eduInstitution. # educated at
  ?eduInstitution wdt:P625 ?coord. # location
  SERVICE wikibase:label {bd:serviceParam wikibase:language "ru,en"}
}
GROUP BY ?eduInstitutionLabel
ORDER BY DESC(?count)
\end{lstlisting}
\footnotetext{Получено: 168 университетов в 2022 году. Ссылка на SPARQL-запрос: \href{https://w.wiki/5CMt}{https://w.wiki/5CMt}.}




%%
% Профессии создателей языков программирования
%%
\section{Профессии создателей языков программирования}

\begin{marginfigure}[0pt]
\includegraphics[width=0.8\textwidth]{./chapter/programming_language/Bubble_chart_showing_the_quantity_of_professions_people_,creating_programming_languages,_have_2017.png}
  \caption[Профессии разработчиков языков программирования, 2017 год.]{Профессии разработчиков языков программирования, 2017 год. Размер пузырька показывает число разработчиков с соответствующей профессией\\}
  \label{fig:2017_profession}
\end{marginfigure}

С помощью запроса~\ref{lst:developer_profession} 
построим пузырьковую диаграмму, отображающую информацию о~том, 
какие профессии преобладают среди людей, разрабатывающих языки программирования. 
Результаты этого запроса в 2017 и 2020 годах можно видеть 
на~рис.~\ref{fig:2017_profession} и~\ref{fig:2020_profession} соответственно.


\index{График!BubbleChart!Профессии создателей языков программирования}
\begin{lstlisting}[
	language=SPARQL,
	label=lst:developer_profession,
	caption={\href{https://w.wiki/v42}{Профессии создателей языков программирования}\protect\footnotemark},
    numbers=none,
	texcl
]
#defaultView:BubbleChart
SELECT ?occupationLabel (count(*) as ?occupation)
WHERE {
  ?lang wdt:P31 wd:Q9143. # instances of programming language 
  ?lang wdt:P178 ?developer. # developer
  ?developer wdt:P106 ?occupation. # occupation
  SERVICE wikibase:label { bd:serviceParam wikibase:language "ru" }
}
GROUP BY ?occupationLabel 
ORDER BY DESC(?count)
\end{lstlisting}
\footnotetext{Получено: 48 профессий в~2017 году и 74~профессии в 2020 году. 
              Ссылка на SPARQL-запрос: \href{https://w.wiki/v42}{https://w.wiki/v42}.}




\newpage
\begin{marginfigure}[0pt]
    \includegraphics[width=1.0\textwidth]{./chapter/programming_language/Bubble_chart_showing_the_quantity_of_professions_people,_creating_programming_languages_RU_2020.png}
    \caption[Профессии разработчиков языков программирования, 2020 год.]{Профессии разработчиков языков программирования, 2020 год. Размер пузырька показывает число разработчиков с соответствующей профессией\\ \\}
    \label{fig:2020_profession}
\end{marginfigure}


Наиболее распространенными профессиями оказались: 
\href{https://www.wikidata.org/wiki/Q21198}{специалист в области компьютерных наук}, 
\href{https://www.wikidata.org/wiki/Q81096}{инженер} 
и \href{https://www.wikidata.org/wiki/Q37226}{учитель}. 
Встречаются даже такие профессии, как джазовый музыкант и политик. 
Например, обеими этими профессиями владел 
\href{https://www.wikidata.org/wiki/Q181529}{Герберт Александер Саймон} (1916--2001), 
разработчик \href{https://en.wikipedia.org/wiki/Information_Processing_Language}{Языка обработки информации (IPL)}. 
На 2020 год среди разработчиков языков программирования оказалось больше всего специалистов 
в области компьютерных наук (172 человека), а также 96 инженеров, 57 учителей, 56 программистов и 43 математика.



%%
% Объектно-ориентированные языки программирования
%%
\section{Объектно-ориентированные языки программирования}

С помощью запроса~\ref{lst:oopl} 
получим список всех объектно-ориентированных языков программирования. 
Подсчитаем отношение числа объектно-ориентированных языков программирования 
(запрос~\ref{lst:oopl} вернул 118 языков) 
ко всем языкам программирования 
(запрос~\ref{lst:prog_langs} вернул 1422 языка).
Получили, что 8\,\% языков программирования на 2020 год являются объектно-ориентированными. 

\begin{lstlisting}[
	language=SPARQL,
	label=lst:oopl,
	caption={\href{https://w.wiki/v47}{Список объектно-ориентированных языков программирования}\protect\footnotemark},
    numbers=none,
	texcl
]
SELECT DISTINCT ?lang ?langLabel
WHERE {
 ?lang wdt:P31 wd:Q899523 # is object-oriented programming language
 SERVICE wikibase:label { bd:serviceParam wikibase:language "ru" }
}
\end{lstlisting}
%\footnotetext[11][-3cm]
\footnotetext{Получено: 116 объектно-ориентированных языков программирования на~2017 год, 
                        118 языков на~2020 год. 
              Ссылка на SPARQL-запрос: \href{https://w.wiki/v47}{https://w.wiki/v47}.}





%%
% Полнота викиданных
%%
%\section{Полнота языков программирования в Викиданных}
%По данным Боровского исследовательского университета\autocite{oo_langs_bourabai}, 
%существует как минимум 26 языков программирования, 
%которые поддерживают объектно-ориентированную парадигму. 
%В статьях, посвящённых объектно-ориентированному программированию, 
%к этому списку добавляются ещё четыре\autocite{oo_langs_science_wikia} 
%и три\autocite{oo_langs_garshin} языка программирования. 
%При этом SPARQL-запрос~\ref{lst:oopl} вернул 118 результатов.

%Судить о полноте данных по трём приведённым выше источникам достаточно сложно, 
%так как в них обсуждается большое количество малоизвестных, устаревших и узконаправленных языков, 
%которые не освещаются в авторитетных источниках, 
%и при этом не рассматриваются недавно возникшие языки программирования. 
%Полагаем, что Викиданные предоставляют достаточно полный список объектно-ориентированных языков программирования.



\newpage
%%
% Степень заполненности объектов
%%
\section{Степень заполненности имён разработчиков языков программирования на~русском языке}

Сравним конструкции \texttt{serviceParam} и \texttt{rdfs:label}, 
с~помощью которых можно получить метки (имена) объектов. 
%
В~запросе~\ref{lst:serviceParam} используется конструкция serviceParam, 
которая позволила получить имена 261 разработчика языков. 
%
В~запросе~\ref{lst:rdfs} в~строках 4 и 7 используется конструкция \texttt{rdfs:label}, 
которая позволила получить имена 261 разработчика на английском языке (языковой код~--- en) 
и 150 разработчиков с~именами на~русском языке. 
%
Таким образом, если метка на русском языке не~заполнена, 
то конструкция \texttt{serviceParam} возвращает безымянный объект, с~номером объекта вместо имени. 
Запросы с~конструкциями \texttt{rdfs:label} являются более строгими и пропускают 
объекты без меток на заданном языке. 


\begin{lstlisting}[
	language=SPARQL,
	label=lst:serviceParam,
	caption={\href{https://w.wiki/6gDu}{Список названий языков программирования и их разработчиков, 
    найденных с~помощью сервиса serviceParam}\protect\footnotemark},
    numbers=none,
	texcl
]
SELECT ?lang ?langLabel ?developer ?developerLabel WHERE {
  ?lang wdt:P31 wd:Q9143;    # ?lang is programming language
        wdt:P178 ?developer. #       has developer 
  ?developer wdt:P31 wd:Q5.  # developer is human
  SERVICE wikibase:label { bd:serviceParam wikibase:language "ru" }
}
\end{lstlisting}
\footnotetext{Получено: 261 разработчик в~2023 году. 
              Ссылка на SPARQL-запрос: \href{https://w.wiki/6gDu}{https://w.wiki/6gDu}.}

\begin{lstlisting}[
	language=SPARQL,
	label=lst:rdfs,
	caption={\href{https://w.wiki/6gDx}{Список названий языков программирования и их разработчиков, 
    найденных с~помощью конструкции rdfs:label}\protect\footnotemark},
	texcl
]
SELECT ?lang ?langLabel ?developer ?developerLabel WHERE {
  ?lang wdt:P31 wd:Q9143.   # ?lang is programming language
  ?lang wdt:P178 ?developer #       has developer
  ; rdfs:label ?langLabel FILTER (LANG(?langLabel) = "ru").
  
  ?developer wdt:P31 wd:Q5 # instances of human
  ; rdfs:label ?developerLabel FILTER (LANG(?developerLabel) = "ru").
}
\end{lstlisting}
\footnotetext{Получено: 150 разработчиков с~именами на~русском языке в~2023 году. Ссылка на SPARQL-запрос: \href{https://w.wiki/6gDx}{https://w.wiki/6gDx}.}


%%
% Упражнения
%%
\section{Упражнения}
\label{prog_lang_test}
\begin{enumerate}
	\item Вывести все языки программирования со свойством <<\href{https://www.wikidata.org/wiki/Property:P822}{персонаж-талисман (P822)}>> (узнаваемый персонаж, олицетворяющий собой некий коллектив: школу, спортивную команду, сообщество, воинское подразделение, мероприятие или бренд).
Ответ на~с.~\pageref{answer:prog_langs_4}.

	\item Подчитать количество языков программирования, созданных ранее 1992 года (используйте свойство: <<\href{https://www.wikidata.org/wiki/Property:P571}{дата-основания/создания (P571)}>>).
Ответ на~с.~\pageref{answer:prog_langs_4}.

\marginnote[21pt]{%
        \index{Языки программирования!Определения!Хештег}Хештег (\#)~--- 
        ключевое слово, 
        используемое в микроблогах и социальных сетях, 
        облегчающее поиск сообщений по теме или содержанию 
        и начинающееся со знака решётки.}
%
	\item Построить столбчатую диаграмму, отражающую количество известных хештегов в Твиттере 
        для~каждого языка программирования (свойство: <<\href{https://www.wikidata.org/wiki/Property:P2572}{хештег Твиттера (P2572)}>>).
Ответ на~с.~\pageref{answer:prog_langs_4}.
\end{enumerate}
