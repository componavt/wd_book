\chapter{Где учатся и кем работают изобретатели языков программирования}
\label{ch:programming languages}
\section*{Аннотация}
В статье исследуются свойства языков программирования на основе базы знаний международного проекта Викиданные. С помощью SPARQL-запросов, вычисляемых на объектах типа "язык программирования" в Викиданных, решен ряд задач. Получены перечени всех языков программирования под пермиссивными лицензиями и языков с закрытыми лицензиями и рассчитано их процентное соотношение. Построена пузырьковая диаграмма по количеству форматов файлов исходного кода. Получены карты, отображающие месторасположение учебных заведений и компаний, в которых учились или работали люди, связанные с созданием языков программирования. Построена пузырьковая диаграмма, отображающая профессии людей, причастных к созданию и разработке языков программирования. Получен список всех объектно-ориентированных языков программирования и сделан вывод об исчерпывающей полноте Викиданных относительно них. Проведено сравнение и анализ результатов SPARQL-запросов 2017 года и 2020 года, отмечены основные изменения. 


%\begin{figure}[h]
%\includegraphics[scale=0.5]{abstract_pic.png}
%\centering 
%\caption{Сравнение полученных графиков количества форматов.}
%\label{ris:image}
%\end{figure}
