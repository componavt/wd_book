\hyphenation{аудио-драмах}
\hyphenation{аудио-записях}

% Newline in table of contents, see https://tex.stackexchange.com/q/170724/99685
\addtocontents{toc}{\protect\pagebreak[4]}
\chapter{Аниме: загадочный и поразительный мир японской анимации}
\label{ch:anime}

В рамках этой главы исследуется объект Викиданных \wdqName{<<аниме>>}{1107}. 
С помощью SPARQL-запросов, вычисляемых на объектах типа <<аниме>> в Викиданных, 
получен список сэйю\sidenote[][-1\baselineskip]{%
%
Сэйю~--- это японские актёры озвучивания. 
Сэйю обычно озвучивают роли персонажей в аниме, видеоиграх, фильмах, 
а также на радио и телевидении или выступают в роли рассказчика в радиопостановках. 
Кроме того, голоса сэйю используются в~рекламе, голосовых объявлениях, 
аудиозаписях книг и учебных материалов, а~также для переозвучивания. 
Сэйю могут быть как мужчины, так~и~женщины, как~взрослые, так и дети.% eo sidenote

\vspace{5pt}
\includegraphics[width=0.5\linewidth]{chapter/anime/seyu.jpg}

\noindent Сэйю Кэндзи Акабанэ озвучивал роль персонажа Sasuke Sarutobi в видеоигре Ikemen Sengoku, 2018 год. 
Wikimedia Commons / numan (CC BY-SA)
\vspace{5pt}
%
}, упорядоченный по числу озвученных ими аниме, 
построен график числа аниме, озвученных одним сэйю, 
построен граф, связывающий сэйю
и озвученные ими аниме, получена оценка трудоспособного возраста сэйю. 

%%%\begin{marginfigure}[0.0cm]{
%	\setlength{\fboxsep}{0pt}%
%	\setlength{\fboxrule}{1pt}%
%	\fcolorbox{gray}{gray}{
%%%    \includegraphics[width=0.5\linewidth]{chapter/anime/seyu.jpg}%}
%%}
%%%\caption[Сэйю Кэндзи Акабанэ.]%
%%%        {Сэйю Кэндзи Акабанэ озвучивал роль персонажа Sasuke Sarutobi в видеоигре Ikemen Sengoku, 2018 год. 
%%%Wikimedia Commons / numan (CC BY-SA)}
%%%\label{fig:seiyu}
%%\end{marginfigure}

\section{Экземпляры объекта <<Аниме>>}

Аниме~--- это японская мультипликация. 
Она стоит особняком и выделяется своим визуальным стилем, 
однако есть и менее очевидные отличия: 
так, по сравнению с американской и японской анимацией у аниме значительно шире список жанров~--- 
от детских и семейных комедий до драматических историй, 
которые на Западе обычно представляются только в фильмах 
с~живыми актёрами\sidenote{%   \autocite{sister_city}
%
    Slowpoke T. Аниме vs мультипликация. 2020. 
    URL: \href{https://cadelta.ru/anime/id6119}
              {https://cadelta.ru/anime/id6119}.%
}. % eo sidenote


У каждого аниме есть свои актёры озвучивания. 
В дальнейшем под <<сэйю>> будем понимать японского актёра озвучивания. 
В японской анимации слова <<актёры озвучивания>> и <<сэйю>> являются синонимами\sidenote{%
%
    Шикимори~--- энциклопедия аниме и манги. 
    URL:~\href{https://shikimori.one/}
              {https://shikimori.one/}.%
}.\, % eo sidenote
%
Под словом <<тайтл>> (от англ. \emph{title}, \emph{название}) обычно понимают конкретное аниме\autocite{anime_social}. 
В~общем же смысле слово <<тайтл>> означает понятие, объединяющее различные виды продукции 
(от~кино\-фильма до~романа), созданные на основе конкретного произведения, за которым закреплено строго определённое название\autocite{anime_title_def}.

Чтобы работать со списком аниме в Викиданных, 
нам понадобятся объект \wdqName{<<аниме>>}{1107} и свойство \wdProperty{31}{<<экземпляр>>}. 
Получим список всех аниме без учёта подклассов (запрос~\ref{lst:anime}).

\newpage

\begin{lstlisting}[ language=SPARQL, 
                    caption={\href{https://w.wiki/4ABw}{Список аниме без учёта подклассов}\protect\footnotemark},
                    label=lst:anime,
                    texcl,
                    numbers=none,
                    ]
# List of instances of anime
SELECT ?anime ?animeLabel
WHERE
{
    ?anime wdt:P31 wd:Q1107. # instance of anime
    SERVICE wikibase:label{bd:serviceParam wikibase:language "ru,en,ja"}
}
\end{lstlisting}%
\footnotetext{Получено: \num{683} результата в 2017 году и \num{216} результатов в 2021 году. Ссылка на~SPARQL-запрос: \href{https://w.wiki/4ABw}{https://w.wiki/4ABw}.}

\index{График!Sunburst diagram}
\begin{marginfigure}[1\baselineskip]
{\includegraphics[width=0.5\linewidth]{chapter/anime/anime-subclasses-sunburst-diagram-2021.png}}
\vspace{-7pt}
\caption{Жанры аниме на круговой диаграмме, 2021 год}%
\label{fig:anime_piechart}
\end{marginfigure}

В действительности в Викиданных объектов аниме гораздо больше, 
но они являются экземплярами не объекта <<аниме>>, а его подклассов, 
таких как, например, \wdqName{<<аниме-сериал>>}{63952888}. 
Чтобы получить список жанров аниме и количество аниме, 
относящихся к этим жанрам, выполним запрос~\ref{lst:anime_w_subclass}.

\begin{lstlisting}[ language=SPARQL, 
                    caption={\href{https://w.wiki/4ABj}{Список жанров аниме и количество аниме, относящихся к этим жанрам}\protect\footnotemark},
                    label=lst:anime_w_subclass,
                    numbers=none,
                    texcl 
                    ]
# Select anime and its subclasses with number of titles
# corresponding to these subclasses
SELECT ?subAnime ?subAnimeLabel (COUNT(?subAnimeInst) AS ?count)
WHERE {
  ?subAnime wdt:P279* wd:Q1107.   # select anime subclass list
  ?subAnimeInst wdt:P31 ?subAnime # link titles and subclasses
    SERVICE wikibase:label{bd:serviceParam wikibase:language "ru,en,ja"}
}
GROUP BY ?subAnime ?subAnimeLabel
ORDER BY DESC(?count)
\end{lstlisting}%
\footnotetext{Получено: \num{11} результатов в 2021 году. Ссылка на~SPARQL-запрос: \href{https://w.wiki/4ABj}{https://w.wiki/4ABj}.}


\index{Визуализация данных!Rawgraphs}
Визуализировать распределение аниме по~жанрам можно в~виде 
круговой диаграммы (рис.~\ref{fig:anime_piechart}). 
построенной с~помощью сервиса 
Rawgraphs (\href{https://app.rawgraphs.io}{https://app.rawgraphs.io}). 
По-английски такой вид диаграм, имеющих радиально расходящиеся лучи, 
называется \emph{sunburst diagram}. 
%
Полученная классификация аниме по жанрам не идеальна, 
так как есть большое смещение в сторону аниме-сериалов: 
из \num{4875} тайтлов \num{2984} отнесены к жанру <<аниме-сериал>> (\num{62,7}\,\%); 
вероятно, классификация жанров аниме в Викиданных требует дальнейшего уточнения. 
Также в подклассы попали понятия, относящиеся не к общей классификации, 
а~к~отдельным аниме, например \href{https://w.wiki/4L5p}{<<Евангелион>>}.



\newpage
Получим список всех аниме, включая тайтлы, относящиеся к жанрам аниме (запрос~\ref{lst:all_anime_list}).

%\begin{minipage}{\linewidth}
\begin{lstlisting}[ language=SPARQL, 
                    caption={\href{https://w.wiki/49zY}{Список всех аниме в Викиданных}\protect\footnotemark},
                    label=lst:all_anime_list,
                    numbers=none,
                    texcl 
                    ]
# List of instances of anime and subclasses of anime
SELECT ?anime ?animeLabel
WHERE
{
    ?anime wdt:P31/wdt:P279* wd:Q1107. # instance of anime/subclass
    SERVICE wikibase:label{bd:serviceParam wikibase:language "ru,en,ja"}
}
\end{lstlisting}%
\footnotetext{Получено: 683 результата в 2017 году и 4875 результатов в 2021 году. Ссылка на~SPARQL-запрос: \href{https://w.wiki/49zY}{https://w.wiki/49zY}.}
%\end{minipage}

Аниме, о которых есть наиболее полная информация на Викиданных~--- 
это \wdqName{Гуррен-Лаганн}{4277}, \wdqName{Space Battleship Yamato}{4292}, 
\wdqName{Project A-ko}{4316}. 
Аниме с~малоинформативными записями на~Викиданных оказались 
\wdqName{Doraemon}{711311}, 
\wdqName{The Animal Conference on the Environment}{97195557}, 
\wdqName{Assassins Pride}{96737300}.

Среди всех аниме-тайтлов в Викиданных больше всего свойств, 
по данным сервиса ProWD, %deadlink \autocite{anime_prowd}, 
у~фильма \wdqName{Fullmetal Alchemist: The Sacred Star of Milos}{1004318}\footnote{%
%
    <<Стальной алхимик: Священная звезда Милоса>>~--- полнометражный аниме-фильм, 
    являющийся продолжением аниме-сериала <<Стальной алхимик>>. 
    Его главные герои~--- братья-алхимики, 
    использующие свои магические способности 
    для~борьбы с силами зла и противостояния преступникам.%
} (\num{24} свойства).



\subsection{Список сэйю, упорядоченный по числу озвученных ими аниме}

Разумеется, в большинстве аниме участвует не один, а множество персонажей. 
Соответственно, разных персонажей озвучивают разные сэйю. 
Большинство сэйю озвучили за свою карьеру несколько тайтлов, 
а многие~--- десятки тайтлов. 
Талантливых сэйю приглашают озвучивать сразу нескольких персонажей в одном аниме. Одним из самых популярных сэйю является \href{https://w.wiki/4L5q}{Хироси Камия}, имеющий множество наград и озвучивший более 180 аниме. Самым известным аниме с его участием является \href{https://w.wiki/4L5r}{<<Атака титанов>>}, где он озвучил одного из главных персонажей~--- капитана Леви.

Построим упорядоченный список сэйю по числу озвученных ими аниме (запрос~\ref{lst:seiyu_titles_sorted}).

\newpage
\index{SPARQL!Поиск подклассов!wdt:P31/wdt:P279*}
\begin{lstlisting}[ language=SPARQL, 
                    caption={\href{https://w.wiki/4Xph}{Упорядоченный список сэйю по числу озвученных ими аниме}\protect\footnotemark},
                    label=lst:seiyu_titles_sorted,
                    numbers=none,
                    texcl 
                    ]
# Ordered list of actors-seiyu according to the number of anime where they took part in
SELECT ?seiyu ?seiyuLabel (COUNT(?anime) AS ?count) WHERE
{
  ?anime wdt:P31/wdt:P279* wd:Q1107;  # instance of anime/subclass
         wdt:P725 ?seiyu.  # instance of seiyu (voice actor)
  SERVICE wikibase:label { bd:serviceParam wikibase:language "ru,en,ja" }
}
GROUP BY ?seiyu ?seiyuLabel  # group by seiyu 
ORDER BY DESC(?count)  # order by count of voiced anime
\end{lstlisting}%
\footnotetext{Получено: 148 результатов в 2017 году и 2910 результатов в 2021 году. Ссылка на~SPARQL-запрос: \href{https://w.wiki/4Xph}{https://w.wiki/4Xph}.}



\subsection{График по числу сэйю, озвучивших одно и более аниме}

Построим линейную диаграмму сэйю, озвучивших аниме, с~помощью запроса~\ref{lst:seiyu_titles_graph}. 
При этом чем~больше аниме озвучил сэйю, тем правее на диаграмме он будет находиться (рис.~\ref{fig:Seiyu_num_chart_2021_ru}). 

%\begin{fullwidth}
%\begin{minipage}
%\lstset{numbers=left, firstnumber=1, frame=single}
\index{SPARQL!FILTER!<}
\index{SPARQL!SELECT!вложенный}
\begin{lstlisting}[ 
    language=SPARQL, 
    caption={\href{https://w.wiki/4JvT}{Построение графика по числу сэйю, озвучивших одно и более аниме}\protect\footnotemark},
    label=lst:seiyu_titles_graph,
%    texcl,
    xleftmargin=18pt, 
    numbers=left,
    ]
# Graph of the number of voice actings of different seiyu
#defaultView:LineChart
SELECT ?seiyuRoles (COUNT(?seiyuRoles) AS ?quantity) WHERE {
  FILTER(?seiyuRoles < 71) # limit a numbef of seiyu in graph
  {       # count quantity of voice acting by one seiyu
    SELECT (COUNT(?seiyu) AS ?seiyuRoles) WHERE { 
      ?anime wdt:P31/wdt:P279* wd:Q1107; # instance of anime and its subclasses
                 wdt:P725 ?seiyu.        # instance of seiyu
      SERVICE wikibase:label { bd:serviceParam wikibase:language "ru,en,ja"}
    }
    GROUP BY ?anime            # group list by number of voiced anime
    ORDER BY DESC(?seiyuRoles) # order by voice acting quantity (descending)
  }
}
GROUP BY ?seiyuRoles       # grouping and
ORDER BY DESC(?seiyuRoles) # sorting seiyu by number of voice actings
\end{lstlisting}%
\footnotetext{Получено: 13 результатов в 2017 году и 58 результатов в 2021 году. Ссылка на~SPARQL-запрос: \href{https://w.wiki/4JvT}{https://w.wiki/4JvT}.}
%\lstset{numbers=none}
%\end{minipage}
%\end{fullwidth}


\newpage
На рис.~\ref{fig:Seiyu_num_chart_2021_ru} хорошо видно, 
что чем выше планка количества озвученных аниме, 
тем меньше сэйю достигают этой планки. 
В~строке~4 запроса~\ref{lst:seiyu_titles_graph} установлено ограничение в~71~аниме, 
поскольку сэйю, которые отметились в большем количестве аниме,~--- единицы и расширение графика вправо было бы не слишком информативным.

Многие сэйю, как показано на рис.~\ref{fig:Seiyu_num_chart_2021_ru}, 
озвучили только одно аниме~--- на графике их оказалось~254. 
Однако сэйю~--- это профессия, которой люди зачастую посвящают всю свою жизнь. 
То,~что, согласно Викиданным, человек за многие годы принял участие в озвучке только одного аниме, 
может быть связано с отсутствием информации о других его ролях в Викиданных. 

\index{График!LineChart}
\begin{figure*}[h]
%    \setlength{\fboxsep}{0pt}%
%    \setlength{\fboxrule}{1pt}%
%    \fcolorbox{gray}{gray}{
    \includegraphics[width=\linewidth]{./chapter/anime/Seiyu_chart_2021_ru.png}%}
	\caption[График числа ролей, озвученных различными сэйю, 2021 год.]{График числа ролей, озвученных различными сэйю, 2021 год.\\График построен на~основе данных, полученных с~помощью запроса~\protect\ref{lst:seiyu_titles_graph}}%
    \label{fig:Seiyu_num_chart_2021_ru}%
\end{figure*} 




\newpage
\subsection{Граф, связывающий сэйю и озвученные ими аниме}

Итак, многие сэйю за свою карьеру успевают озвучить несколько аниме. 
Чтобы нагляднее показать эту взаимосвязь, 
построим граф, связывающий сэйю и озвученные ими аниме с~помощью запроса~\ref{lst:seiyu_graph}. 
Фрагмент итогового графа представлен на рис.~\ref{fig:Seiyu_graph_2021_ru}. 


\lstset{numbers=left, firstnumber=1, frame=single, texcl}
\index{SPARQL!DISTINCT}
\index{SPARQL!BIND}
\index{SPARQL!IF}
\begin{lstlisting}[ 
    language=SPARQL, 
    caption={\href{https://w.wiki/4HFt}{Построение графа, связывающего сэйю и озвученные ими аниме}\protect\footnotemark},
    label=lst:seiyu_graph,
    texcl,
    xleftmargin=18pt, 
    numbers=left,
                    ]
# Graph of seiyu with more than one anime
#defaultView:Graph
SELECT DISTINCT ?item ?itemLabel ?rgb ?link
WHERE
{ # voice actors (seiyu) with more than one anime
  VALUES ?toggle { true false }
  VALUES ?seiyu { wd:Q6381410 wd:Q1347031 wd:Q1207010 
                  wd:Q233902  wd:Q1323728 wd:Q2440809 
                  wd:Q355173  wd:Q957795  wd:Q50033}
  ?anime  wdt:P31/wdt:P279* wd:Q1107; # instance of anime/subclass
          wdt:P725 ?seiyu;            # seiyu who voiced this anime 
  SERVICE wikibase:label{bd:serviceParam wikibase:language "ru,en,ja"}
  BIND(IF(?toggle,?anime,?seiyu) AS ?item).
  BIND(IF(?toggle,?animeLabel,?seiyuLabel) AS ?itemLabel).
  BIND(IF(?toggle,"FFFFFF","7FFF00") AS ?rgb).
  BIND(IF(?toggle,"",?anime) AS ?link).
}
\end{lstlisting}%
\footnotetext{Получено: \num{826} результатов в 2017 году и \num{494} результата в 2021 году. Ссылка на~SPARQL-запрос: \href{https://w.wiki/4HFt}{https://w.wiki/4HFt}.}
\lstset{numbers=none}

В переменную \lstinline|?seiyu| (строки 7--9 запроса~\ref{lst:seiyu_graph}) 
с~помощью оператора \lstinline|VALUES| 
записан массив объектов Викиданных, 
соответствующих некоторым известным сэйю~--- \wdqName{Кадзуэ Комия}{6381410} и другим. 
Мы выбрали только девять сэйю в иллюстративных целях, 
поскольку для~большего числа сэйю граф стал~бы неудобным для~восприятия.

\index{SPARQL!BIND!IF}
\index{SPARQL!IF}
Конструкция \lstinline|BIND(IF(?toggle, ?anime, ?seiyu)...)| в строке \num{13} 
позволяет определить тип вершины графа: 
если \lstinline|?toggle| принимает значение \lstinline|true|, 
то вершина графа соответствует аниме, иначе~--- сэйю. 
В строках 14 и 15 определяются тип подписи для вершины и цвет вершины. 
Строка 16 позволяет отобразить связи между сэйю и аниме.

\newpage
\begin{figure*}[h!]
    \index{График!Graph}
%\begin{flushright}
	\includegraphics[width=0.7\linewidth]{./chapter/anime/Seiyu_graph_2021_ru.jpg}\centering
%\end{flushright}
	\caption[Граф сэйю и аниме, 2021 год.]{Фрагмент графа, связывающего сэйю и озвученные ими аниме, 2021.\\Граф построен на основе данных, полученных с помощью запроса~\protect\ref{lst:seiyu_graph}}%
      \label{fig:Seiyu_graph_2021_ru}%
\end{figure*} 




\section{Полнота Викиданных по числу аниме и актёров}

Список тайтлов в Русской Википедии\footnote{%
%
    Проект:Аниме и манга/Списки/Список аниме. 
    URL:~\href{https://w.wiki/4JVE}{https://w.wiki/4JVE}.%
%
} содержит 1638 аниме. 
Также можно посмотреть телевизионные показы аниме в России по годам\footnote{%
%
    Проект:Аниме и манга/Телевизионные показы аниме в России по годам.\\ 
    URL:~\href{https://w.wiki/4JVH}{https://w.wiki/4JVH}.%
%
}. На сайте любительской энциклопедии аниме Shikimori\footnote{% %\autocite{shikimori} 
%
    Шикимори~--- энциклопедия аниме и манги. 
    URL:~\href{https://shikimori.one/}
              {https://shikimori.one/}.%
%
} список аниме включает 801~страницу по~20~наименований. 
\marginnote{%
%
    \MarginQuestion 
    С помощью SPARQL подсчитайте, сколько аниме вышло за минувший год.%
%
} 
Нетрудно подсчитать, что на сайте есть информация о 16\,020 тайтлах, 
в~то~время как в~Викиданных объектов, описывающих аниме, всего 4875 
(см.~запрос~\ref{lst:all_anime_list}). 
К тому же стоит учитывать, что скорость выхода новых аниме довольно велика. 
Из этого можно сделать вывод, что Викиданные крайне неполно отражают данные 
(есть информация только о 29.6\,\% аниме). 
%То~же~самое касается и жанров: 
%в~разделе <<Лучшие аниме>> сайта Шикимори доступны 42 жанра аниме, 
%в~то~время как Викиданные содержат информацию только о~17\footnote{%
%
%    Список жанров (категорий) аниме, указанных в Викиданных, 
%    можно посмотреть на странице 
%\href{https://en.wikipedia.org/wiki/Category:Lists\_of_anime\_by\_genre}{Category:Lists of anime by genre}.%
%}.

Возможно, приведённые ниже статьи и сайты не будут являться~АИ\footnote{%
%
    Авторитетный источник (АИ)~--- это ресурс, который владеет информацией, 
    в~компетентности и актуальности которой не может быть никаких сомнений. 
    См.~\href{https://w.wiki/3u9v}{https://w.wiki/3u9v}.%
%
}, но~с~их~помощью можно проанализировать информацию об~имеющихся аниме 
и сделать дополнительные выводы о~неполноте Викиданных.


\begin{itemize}
	\item На сайте любительской озвучки \href{http://online.anidub.best/}{AniDub}\footnote{%
            АниДаб. URL: \href{http://online.anidub.best/}
                              {http://online.anidub.best/}.} 
            приведён список из 5756 аниме.
	\item На сайте онлайн-кинотеатра \href{http://animespirit.tv/}{AnimeSpirit}\footnote{%
            AnimeSpirit. URL: \href{http://animespirit.tv/}
                                   {http://animespirit.tv/}.} 
            приведён список из 1968 аниме.
	\item На новостном форуме по тематике аниме \href{http://animeland.su/}{AnimeLand}\footnote{%
            AnimeLand. URL: \href{http://animeland.su/}
                                 {http://animeland.su/}.} 
            приведён список из \num{4795} аниме.
	\item На сайте онлайн-кинотеатра \href{https://anivost.org/}{Anivost}\footnote{%
            Anivost. URL: \href{https://amedia.cc/}
                               {https://amedia.cc/}.} 
            приведён список из 420 аниме.
\end{itemize}

Какие-то сайты появились позже, какие-то раньше, поэтому количество аниме на них может разниться, причём довольно серьёзно. Если упорядочить все приведённые сайты, данные Русской Википедии, Английской Википедии и Викиданные по количеству аниме, то Викиданные окажутся не~на~последнем месте, но, например, вышеупомянутой энциклопедии \href{https://shikimori.one/}{Shikimori} они уступают почти в~4~раза.



\newpage
\noindent\begin{marginfigure}%
{%
\index{График!Sunburst diagram}%
%\setlength{\fboxsep}{0pt}%
%\setlength{\fboxrule}{1pt}%
%\fcolorbox{gray}{gray}{
    \includegraphics[width=1\linewidth]{./chapter/anime/actors-role-counts-sunburst-diagram-2021.png}%}%
}%
    \caption[Круговая диаграмма числа ролей, озвученных различными сэйю, 2021 год.]
            {Круговая диаграмма числа ролей, озвученных различными актёрами,\\по~данным на~2021 год}%
\label{fig:roles_piechart}%
\end{marginfigure}%
%
%\begin{figure*}[h!]
%    \index{График!Sunburst diagram!Количество ролей, озвученных разными актёрами}
%	\includegraphics[width=0.7\linewidth]{./chapter/anime/actors-role-counts-sunburst-diagram-2021.png}
%	\caption[Круговая диаграмма числа ролей, озвученных различными сэйю, 2021 год.]{Диаграмма <<солнечные лучи>> числа ролей, озвученных различными актёрами, построенная в 2021 году с помощью сервиса Rawgraphs (\href{https://app.rawgraphs.io}{https://app.rawgraphs.io}).}%
%      \label{fig:roles_piechart}%
%\end{figure*}
Вспомним запрос~\ref{lst:seiyu_titles_sorted}, в котором говорилось о \num{2910} сэйю на Викиданных. 
Дело в том, что поиск производился только по актёрам озвучивания, 
связанным с аниме, поэтому результат оказался таким скромным. 
Если запросить информацию о всех актёрах озвучивания 
(то есть убрать ограничение на~категорию аниме), 
то количество результатов может увеличиться в~5~раз (запрос~\ref{lst:voice_actors_list}). 
Значительный прирост числа результатов относительно запроса~\ref{lst:seiyu_titles_sorted} 
напоминает нам о~том, что в~индустрии озвучивания гораздо больше направлений, 
чем только аниме, например озвучивание фильмов и видеоигр. 
Сэйю могут участвовать в работе и над такими проектами, что нужно учитывать при формировании запросов. 

\begin{lstlisting}[ language=SPARQL, 
                    caption={\href{https://w.wiki/4aQt}{Получение списка актёров озвучки и числа озвученных ими проектов}\protect\footnotemark},
                    label=lst:voice_actors_list,
                    texcl 
                    ]
# Ordered list of actors according to the quantity of projects
# voiced by them
SELECT ?actor ?actorLabel (COUNT(?project) AS ?count)
WHERE
{
  ?project wdt:P725 ?actor.	# instance of voice actor
  SERVICE wikibase:label {bd:serviceParam wikibase:language "ru,en,ja"}
}
GROUP BY ?actor ?actorLabel
ORDER BY DESC(?count)       # order by number of voiced projects
\end{lstlisting}%
\footnotetext{Получено: \num{3965} результатов в 2017 году и \num{14744} результата в 2021 году. Ссылка на~SPARQL-запрос: \href{https://w.wiki/4aQt}{https://w.wiki/4aQt}.}


\index{Визуализация данных!Rawgraphs}
Круговая диаграмма 
на~рис.~\ref{fig:roles_piechart}~--- один из вариантов визуализации данных, 
полученных с~помощью скрипта~\ref{lst:voice_actors_list} 
и~затем обработанных сервисом \href{https://app.rawgraphs.io}{Rawgraphs}. 
С помощью подобных диаграмм можно, например, оценить, 
какой актёр внёс наибольший вклад в развитие индустрии озвучивания.



\section{Указана ли дата публикации у аниме?}

Каждый ценитель японской анимации желает знать, 
в каком году вышло его любимое аниме. 
Викиданные располагают этой информацией не в полной мере. 
В~следующем запросе~\ref{lst:anime_no_pub_date} подсчитывается количество аниме с незаполненным полем publication date (дата публикации). 
То,~что~поле должно быть пустым, указано в строке~6 запроса~\ref{lst:anime_no_pub_date} с помощью пустых квадратных скобок.



\newpage
%
\marginnote{\MarginQuestion Напишите скрипт для вычисления доли аниме, 
у которых не указана дата публикации, относительно всех аниме на Викиданных. 
Сравните эту долю с долей за 2021 год (62\,\%) и сделайте вывод об изменении качества Викиданных.}
%
\index{SPARQL!FILTER!NOT EXISTS}
\lstset{numbers=left, firstnumber=1, frame=single, texcl}
\begin{lstlisting}[ 
    language=SPARQL, 
    caption={\href{https://w.wiki/4Hcz}{Получение списка аниме, у которых на Викиданных не указана дата выхода}\protect\footnotemark},
    label=lst:anime_no_pub_date,
    texcl,
    xleftmargin=18pt, 
    numbers=left,
                    ]
# List of anime the release date of which is empty
SELECT ?anime ?animeLabel
WHERE
{
    ?anime wdt:P31/wdt:P279* wd:Q1107;  # instance of anime
    FILTER NOT EXISTS { ?anime wdt:P577 [] }
    SERVICE wikibase:label{bd:serviceParam wikibase:language "ru,en,ja"}
}
\end{lstlisting}%
\footnotetext{Получено: 237 результатов в 2017 году и 2940 результатов в 2021 году. Ссылка на~SPARQL-запрос: \href{https://w.wiki/4Hcz}{https://w.wiki/4Hcz}.}
\lstset{numbers=none}

На 2021 год из \num{4875} аниме на Викиданных (см. запрос~\ref{lst:all_anime_list}) 
у \num{2940}, а это 62\,\%, не указана дата выхода. 
В 2017 году из 683 аниме на Викиданных только 237 (то есть 35\,\%) не имели указанной даты выхода. 
Похоже, что, к сожалению, увеличение количества информации 
не~всегда сопровождается сохранением её качества.%





\section{Анализ возраста, в котором сэйю озвучивают аниме}

Как и в любой другой профессии, у актёра озвучки есть возраст, 
когда он находится в~<<расцвете сил>> и может озвучить множество аниме. 
Использование SPARQL и внешних инструментов для~анализа данных, 
подобных языку программирования Python\index{Программирование!Язык!Python}\footnote[][-1cm]{%
    Язык Python~--- интерпретируемый язык программирования, 
    благодаря своей гибкости используемый для решения разнообразных задач. 
    Его можно применять в том числе и для работы с Викиданными: 
    например, в разделе~\ref{ch:bots} (с.~\pageref{ch:bots}) 
    описан процесс создания программных ботов для Викиданных.%
%
}, может позволить оценить такой активный возраст на основе информации из Викиданных.


Чтобы получить исходные данные для исследования, необходимо выполнить три SPARQL-скрипта 
и экспортировать результаты их выполнения в формате .csv\footnote[][-0.3cm]{%
%
    CSV (comma-separated values)~--- формат представления табличных данных, 
    в~котором таблица хранится в виде последовательности строк текста. 
    Эти строки содержат значения полей таблицы, разделённые запятыми.%
%
}.\, CSV-файлы затем используются в~скрипте на~языке Python, 
который генерирует выходной график. 
Запускать программы на Python можно, например, 
на платформе Google Colaboratory\footnote[][0.2cm]{%
%
    \index{Программирование!Среда разработки!Colab}
    Google Colaboratory (Colab)~--- облачная среда разработки от компании Google, 
    в~которой можно создавать и запускать скрипты на языке программирования Python, 
    а~также делиться результатами своей работы с другими людьми. 
    \mbox{Также} этот сервис удобен тем, что предоставляет вычислительные мощности~--- 
    как~обычные процессоры, так и видеокарты для, например, работы с нейронными сетями. 
    Сервис доступен по ссылке: \href{https://colab.research.google.com}{https://colab.research.google.com}.%
}.



\newpage
\index{SPARQL!SERVICE}
\index{SPARQL!rdfs:label}
Получить список всех зарегистрированных в Викиданных сэйю и их дат рождения 
можно двумя способами (запросы~\ref{lst:seiyu_bd_w_service} и~\ref{lst:seiyu_bd_w_rdfs}): 
с~помощью команды \lstinline|SERVICE| 
и с~помощью конструкции \mbox{\lstinline|rdfs:label|}. 
Различия между запросами~\ref{lst:seiyu_bd_w_service} и~\ref{lst:seiyu_bd_w_rdfs} заключаются в~том, что:
%
\begin{itemize}%[noitemsep,topsep=0pt]
    \item метка (имя) сэйю в первом случае получается с помощью переменной \lstinline|?seiyuLabel| 
        (в~таком случае нужно указать команду \lstinline|SERVICE| для установки языков, на котором будут возвращены имена), 
        а во втором~--- с помощью конструкции \lstinline|rdfs:label|;
    \item в первом варианте скрипта необходимо указывать \lstinline|?seiyuLabel| 
        как параметр \lstinline|GROUP BY|\index{SPARQL!GROUP BY}, чтобы связать объекты сэйю и их метки.
\end{itemize}

% # Get list of all seiyu objects, their names and birth dates
%\begin{fullwidth}
%
%\noindent\begin{minipage}[]{.49\linewidth}
\begin{lstlisting}[ language=SPARQL, breaklines=false, numbers=none,
                    caption={Получение списка дат рождения сэйю с помощью \lstinline|SERVICE|\protect\footnotemark},
                    label=lst:seiyu_bd_w_service,
                    texcl,
                    escapechar=!
                    ]
SELECT ?seiyu ?seiyuLabel ?bDate
WHERE {
  ?anime (wdt:P31/(wdt:P279*)) wd:Q1107;
    wdt:P725 ?seiyu.       
  ?seiyu wdt:P569 ?bDate. 
  SERVICE wikibase:label {bd:serviceParam wikibase:language "ru,en,ja"}
}
GROUP BY ?seiyu ?seiyuLabel ?bDate
\end{lstlisting}%
    \footnotetext[30]{Получено: \num{2515} результатов в 2021 году. SPARQL-запрос: \href{https://w.wiki/4FPq}{https://w.wiki/4FPq}.}
%\end{minipage}%
%just a break for lines between two columns of listings
%\hfill
%\begin{minipage}[]{.5\linewidth}
\begin{lstlisting}[ language=SPARQL, breaklines=true, numbers=none,
                    caption={Получение списка дат рождения сэйю с помощью \lstinline|rdfs:label|\protect\footnotemark},
                    label=lst:seiyu_bd_w_rdfs,
                    texcl,
                    escapechar=!
                    ]
SELECT ?seiyu (SAMPLE(?seiyu) AS ?seiyuLabel) ?bDate
WHERE {
  ?anime (wdt:P31/(wdt:P279*)) wd:Q1107;
    wdt:P725 ?seiyu.      # seiyu is anime voice actor
  ?seiyu wdt:P569 ?bDate. # has a birthday 
  ?seiyu rdfs:label ?label.
}
GROUP BY ?seiyu ?bDate
\end{lstlisting}%
    \footnotetext[31]{Получено: \num{2515} результатов в 2021 году. SPARQL-запрос: \href{https://w.wiki/4FPn}{https://w.wiki/4FPn}.}
%\end{minipage}
%\end{fullwidth}%








\newpage
Получим список всех зарегистрированных в Викиданных 
аниме и дат их выхода (запрос~\ref{lst:all_anime_releases})\marginnote[1cm]{%
%
    \MarginQuestion
    Визуализируйте результаты работы скрипта~\ref{lst:all_anime_releases}.\\
    Усложните задачу~--- добавьте на график даты окончания показа сериалов.}.

\begin{lstlisting}[ language=SPARQL, 
                    caption={\href{https://w.wiki/4ENc}{Получение дат выхода аниме}\protect\footnotemark},
                    label=lst:all_anime_releases,
                    texcl 
                    ]
# Get all anime objects, their names and release dates
SELECT ?anime ?animeLabel ?animePubDate ?animeSeriesStartDate
WHERE {
  ?anime (wdt:P31/(wdt:P279*)) wd:Q1107.          # object of anime/subclass
  OPTIONAL { ?anime wdt:P577 ?animePubDate. }       # release date of a movie
  OPTIONAL { ?anime wdt:P580 ?animeSeriesStartDate. } # start date of a series
  SERVICE wikibase:label{bd:serviceParam wikibase:language "ru,en,ja"}
}
\end{lstlisting}%
\footnotetext{Получено: \num{5264} результата в 2021 году. Ссылка на~SPARQL-запрос:\\\href{https://w.wiki/4ENc}{https://w.wiki/4ENc}.}

Обратите внимание, что запрос~\ref{lst:all_anime_releases} 
получает не только \wdProperty{577}{даты выхода полнометражных~аниме}, но~и~\wdProperty{580}{даты начала показа сериалов}. 

Получим ссылки между объектами сэйю и аниме, которые они озвучивали (запрос~\ref{lst:link_anime_seiyu}).

\begin{lstlisting}[ language=SPARQL, 
                    caption={\href{https://w.wiki/4ELh}{Получение ссылок между сэйю и аниме}\protect\footnotemark},
                    label=lst:link_anime_seiyu,
                    texcl 
                    ]
# List of links between seiyu and anime where they are involved in
SELECT DISTINCT ?item ?itemLabel ?link ?itemType
WHERE
{
  VALUES ?toggle { true false }
  ?anime  wdt:P31/wdt:P279* wd:Q1107;   # instance of anime/subclass
          wdt:P725 ?seiyu.              # list seiyu who acted in this anime
  
  BIND(IF(?toggle,?anime,?seiyu) AS ?item).                 # anime/seiyu object
  BIND(IF(?toggle,?animeLabel,?seiyuLabel) AS ?itemLabel).  # anime/seiyu labels link
  BIND(IF(?toggle,?seiyu,?anime) AS ?link).                 # seiyu/anime link
  BIND(IF(?toggle,?seiyu,"seiyu") AS ?itemType).
  SERVICE wikibase:label{bd:serviceParam wikibase:language "ru,en,ja"}
}
\end{lstlisting}%
\footnotetext{Получено: \num{27092} результата в 2021 году. Ссылка на~SPARQL-запрос:\\\href{https://w.wiki/4ELh}{https://w.wiki/4ELh}.}
% [40][-2cm]


% new page because sidenotes numbers should be > then footnote numbers
\newpage
Результат анализа удобно представить в виде гистограммы. 
Для её построения воспользуемся средствами таких библиотек для Python, 
как \href{https://ru.wikipedia.org/wiki/Pandas}{pandas}\footnote[][-\baselineskip]{%
%
    Библиотека pandas предоставляет различные функции для обработки табличных данных 
    для тех, кто пишет программы на языке Python.%
%
}\index{Программирование!Библиотека!pandas} 
и \href{https://ru.wikipedia.org/wiki/Matplotlib}{Matplotlib}\footnote{%[][-0.1cm]{%
%   
    \index{Программирование!Библиотека!Matplotlib}
    Библиотека Matplotlib, также написанная на языка Python, 
    позволяет программистам рисовать графики и диаграммы.%
%
}. Код скрипта, создающего гистограмму, 
опубликован на сервисе \href{https://git.io/J1UGA}{GitHub}\footnote{%
    Ссылка на скрипт в проекте wd\_book на GitHub: \href{https://git.io/J1UGA}{https://git.io/J1UGA}.%
}.


В результате получим гистограмму, по оси абсцисс которой отложен возраст в годах, 
а~по~оси ординат~--- суммарное количество ролей, озвученных всеми сэйю такого возраста. Получившаяся гистограмма представлена на рис.~\ref{fig:Seiyu_age_hist_RU}. 

\index{График!Histogram}
\begin{figure*}[h!]
%    \setlength{\fboxsep}{0pt}%
%    \setlength{\fboxrule}{1pt}%
%    \fcolorbox{gray}{gray}{
    \includegraphics[width=\linewidth]{./chapter/anime/Seiyu_age_hist_RU.png}%}
	\caption[Гистограмма числа аниме, озвученных сэйю разных возрастов, 2021 года.]{Гистограмма с числом аниме, озвученных сэйю разных возрастов, 2021 год. Гистограмма построена на~основе данных, полученных с помощью запросов~\protect\ref{lst:seiyu_bd_w_service} (или~\protect\ref{lst:seiyu_bd_w_rdfs}), \protect\ref{lst:all_anime_releases} и \protect\ref{lst:link_anime_seiyu}.}%
    \label{fig:Seiyu_age_hist_RU}%
\end{figure*} 


\newpage
Отметим следующий забавный факт: в Викиданных нашлись случаи, когда сэйю родился позже, чем вышло аниме с его участием. Вероятно, это связано с отсутствием информации в Викиданных о~втором сезоне/перезапуске аниме. 
Например, в~2021 году такая ситуация была 
с~аниме \wdqName{Sazae-san}{11304591} и~сэйю \wdqName{Нобунага Симадзаки}{5968283}: 
сэйю родился в 1988 году, а дата выхода аниме с его участием~--- 1969 год.

\section{Упражнения}

\begin{enumerate}
    \item Вывести 10 самых популярных аниме, вышедших на экраны в текущем году. 
        Популярность оценить по числу статей в разных языковых разделах. 
        Для подсчёта числа статей об объекте Викиданных используйте SPARQL-конструкцию \lstinline|wikibase:sitelinks|. 
        Например, если статья про аниме есть в~трёх Википедиях на~русском, английском и~испанском языках, то его популярность равна трём. 
    \item Вывести пять аниме, в которых задействовано самое большое число сэйю-женщин.
    \item Построить пузырьковую диаграмму (BubbleChart) распределения аниме по жанрам (сколько аниме в каждом жанре), воспользовавшись свойством \wdProperty{279}{<<подкласс>>}.
    \item Отметить на карте места рождения сэйю.
    \item Построить гистограмму или пузырьковую диаграмму национальностей сэйю.
    \item Построить гистограмму количества вышедших аниме по годам или количества сэйю по~годам рождения.
    \item Построить гистограммы, аналогичные рис.~\ref{fig:Seiyu_age_hist_RU}, 
        но с~учётом пола сэйю (одну для~мужчин, другую для~женщин).
\end{enumerate}
