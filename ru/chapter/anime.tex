\hyphenation{аудио-драмах}
\hyphenation{аудио-записях}

\chapter{Аниме: загадочный и поразительный мир японской анимации}
Статья посвящена исследованию объекта Викиданных <<аниме>>. С помощью SPARQL-запросов, вычисляемых на объектах типа <<аниме>>
в Викиданных, решены такие задачи: выведен упорядоченный список сэйю по числу озвученных ими аниме, построена гистограмма по числу сэйю, озвучивших одно и более аниме, построен граф, связывающий сэйю и озвученные ими аниме. 

\marginnote[0.0cm]{Сэйю — это японские актёры озвучивания. Сэйю обычно озвучивают роли персонажей в аниме, видеоиграх, фильмах, а также на радио и телевидении, или выступают в роли рассказчика в радиопостановках и аудиодрамах. Кроме того, голоса сэйю используются в рекламе, голосовых объявлениях, аудиозаписях книг и учебных материалов, а также для переозвучивания. Сэйю могут быть как мужчины, так и женщины.}

\begin{marginfigure}[0.0cm]
{
	\setlength{\fboxsep}{0pt}%
	\setlength{\fboxrule}{1pt}%
	\fcolorbox{gray}{gray}{\includegraphics{chapter/anime/seyu.jpg}}
}
\caption
[Сэйю]
{
Сэйю Кэндзи Акабанэ озвучивал роль персонажа Sasuke Sarutobi в видеоигре Ikemen Sengoku.\newline
2018 / numan / CC BY-SA 3.0
}
\label{fig:seyu}
\end{marginfigure}

\label{ch:anime}

\section{Экземпляры объекта <<Аниме>>}

Аниме — это японская анимация. У каждого аниме есть свои актёры озвучивания. В дальнейшем под <<сэйю>> будем понимать японского актёра озвучивания. В японской анимации слова <<актёры озвучивания>> и <<сэйю>> являются синонимами \cite{seiyu_def}. Под словом <<тайтл>> (от англ. <<title>>, название) обычно понимают конкретное аниме \cite{anime_social}. В общем же смысле слово <<тайтл>> означает понятие, объединяющее различные виды продукции (от кинофильма до романа), созданные на основе конкретного произведения, за которым закреплено строго определённое название \cite{anime_title_def}.

\begin{itemize}
	\item Свойство: экземпляр (\href{https://www.wikidata.org/wiki/Property:P31}{P31})
	\item Объект: аниме (\href{https://www.wikidata.org/wiki/Q1107}{Q1107})
\end{itemize}

Если получить только список аниме, без учёта подклассов, с помощью такого запроса:

\begin{lstlisting}[ language=SPARQL, 
                    caption={\href{https://w.wiki/4ABw}{Список аниме без учёта подклассов.}\protect\footnotemark},
                    label=lst:anime,
                    texcl 
                    ]
# List of instances of anime
SELECT ?anime ?animeLabel
WHERE
{
    ?anime wdt:P31 wd:Q1107. # instance of anime
    SERVICE wikibase:label { bd:serviceParam wikibase:language "ru,en,ja" }
}
\end{lstlisting}%
\footnotetext{Получено \num{216} результатов в 2021 году. Ссылка на SPARQL-запрос: \href{https://w.wiki/4ABw}{https://w.wiki/4ABw}}

В действительности в Викиданных объектов аниме гораздо больше, но они являются экземплярами не объекта <<аниме>>, а его подклассов.

\begin{lstlisting}[ language=SPARQL, 
                    caption={\href{https://w.wiki/4ABj}{Список аниме, включая аниме, относящиеся к подклассам.}\protect\footnotemark},
                    label=lst:anime_w_subclass,
                    texcl 
                    ]
# Select anime and its subclasses with number of titles corresponding to these subclasses
SELECT ?subAnime ?subAnimeLabel (COUNT(?subAnimeInstance) AS ?count) WHERE {
  ?subAnime wdt:P279* wd:Q1107.       # select anime subclass list
  ?subAnimeInstance wdt:P31 ?subAnime # connect titles and their subclasses
  SERVICE wikibase:label { bd:serviceParam wikibase:language "ru,en,ja". }
}
GROUP BY ?subAnime ?subAnimeLabel
ORDER BY DESC(?count)
\end{lstlisting}%
\footnotetext{Получено \num{4757} результатов в 2021 году. Ссылка на SPARQL-запрос: \href{https://w.wiki/4ABj}{https://w.wiki/4ABj}}

Эта классификация не идеальна, так как есть большое смещение в сторону аниме-сериалов: из 4757 тайтлов 2917 отнесены к жанру «аниме-сериал» (61,3 \%); вероятно, классификация жанров аниме в Викиданных требует дальнейшего уточнения. Также в подклассы попали понятия, относящиеся не к общей классификации, а к отдельным аниме, например, \href{https://ru.wikipedia.org/wiki/Евангелион}{«Евангелион»}.
