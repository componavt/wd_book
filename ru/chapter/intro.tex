\chapter*{Введение}
\label{ch:intro}

\newthought{Эта книга} предназначена для детей, родителей и учителей. 
Она познакомит вас с основами программирования, современными словами и понятиями в информатике\marginnote{%
todo about Computer science and Informatics
}. 
Прочитав книгу вы сможете извлекать из Викиданных информацию с~помощью SPARQL-скриптов, 
затем обрабатывать её и строить по ней таблицы, графики и карты.

Викиданные~--- это искусно сделанная база данных, которая как огромный кит лежит в основе громадной планеты Википедия\marginnote{%
todo об объёме Викиданных: ...
}. Впечатляет скорость роста этого кита, во многих главах мы будем обращать на это внимание.
Если изначально Викиданные создавались\marginnote{%
    todo about Дата создания Викиданных и организации, вложившиеся в создание проекта...
} %
для обслуживания нужд Википедии, 
то сейчас Викиданные используются крайне широко и в самых разных целях.

\newthought{Первая часть} книги включает в себя 10 уроков, описывающих язык программирования и протокол SPARQL.

\newthought{Вторая часть} содержит рецепты решения самых разных практических задач, 
возникающих при работе с объектами Викиданных.

\newthought{В третьей части} описано несколько исследований по Викиданным. 
Исследования выполнены студентами ПетрГУ, в том числе в рамках курса <<Программирование Викиданных>>, 
представленного на сайте Викиверситет\marginnote{%
todo add link
}. 
Этот курс растёт и пишется вместе с этой книгой. 

Потребовалось несколько лет работы со студентами в Википедии, прежде чем мы пришли с ними к таким проектам, 
как Викиверситет и Викиданные. Результатом работы в Википедии стало учебное пособие для тех, 
кто хочет научиться редактировать мировую энциклопедию\cite{Krizhanovsky2015}.


\newthought{Цель} этой книги быть учебником по Викиданным и языку SPARQL.


% \newthought{Книга научит} вас делать ... todo
