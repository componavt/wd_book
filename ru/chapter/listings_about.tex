\chapter{Программы и программки в книге}
\label{ch:listing_about}

В книге будет приводиться \emph{программный код}\footnote[][0cm]{
    Программный код также называют  \emph{исходным кодом} или 
    \emph{листингом}.
%   
} на языке SPARQL. 
Именно на этом языке пишут запросы к Викиданным.


Вот пример SPARQL-скрипта (листинг~\ref{lst:cities}), 
с помощью которого можно получить из Викиданных список городов, 
точнее экземляров объекта\footnote{\label{question:instance-in-OOP-vs-Wikidata}Что такое экземпляр объекта? 
    Какая разница между экземпляром объекта 
    в объектно-ориентированном программировании и в Викиданных?
    См. ответ~\ref{answer:instance-in-OOP-vs-Wikidata} на с.~\pageref{answer:instance-in-OOP-vs-Wikidata}.
    }
\wdqName{city}{515}.

%       escapebegin=ы,escapeend=я>
%       escapechar=ы
% # после знака # (то есть в комментариях) можно ставить \footnote в lstlisting
\begin{lstlisting}[ language=SPARQL, 
                    caption={\href{https://w.wiki/jcE}{Список городов}\protect\footnotemark},
                    label=lst:cities,
                    texcl 
                    ]
SELECT ?city ?cityLabel WHERE { 
  ?city wdt:P31 wd:Q515.       # instance of city
  SERVICE wikibase:label { bd:serviceParam wikibase:language "ru" }
}
\end{lstlisting}%
\footnotetext{Получено \num{20800} городов в 2017 году, \num{9260} городов в 2020 году. Ссылка на SPARQL-запрос: \href{https://w.wiki/jcE}{https://w.wiki/jcE}}

Мы будем регулярно ссылаться на объекты Викиданных. 
Например, \wdqName{city}{515}\footnote[][0cm]{%
%    
    В электронной версии книги имена объектов Викиданных включают гиперссылки на соответствующие страницы Викиданных.
} 
--- это имя объекта Викиданных. 
Здесь ``city''~--- имя метки объекта (Label), 
a ``Q515''~--- это уникальный идентификатор объекта 
и название страницы Викиданных с описанием этого объекта.
\href{https://en.wikipedia.org/wiki/Rule_of_thumb}{Rule of thumb}\footnote[][0cm]{%
%
        \TODO{ О значении и про этимологию из Викисловаря...}
} 
для номеров объектов Викиданных таков, что более значимые объекты 
(например, Солнце --- идентификатор \wdq{525}) 
имеют меньший номер, чем менее известные
(мушка дрозофилы --- \wdq{312154}).
