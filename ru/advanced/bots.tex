\chapter{Боты в Викиданных}
\label{ch:bots}

%\renewcommand{\lstlistingname}{Листинг} % Запрос -> Листинг
\makeatletter
\long\def\@makecaption#1#2{%
    \renewcommand{\lstlistingname}{Листинг}
\centering%
#1. #2\par%
\vspace{5pt}%
}%
\makeatother



В этой главе рассматривается автоматизация процессов в Викиданных. 
Часто нужно исправить повторяющиеся ошибки или ввести большие объёмы данных 
вместо того, чтобы изменять свойства по~одному. 
Для ввода данных в нашем распоряжении есть несколько инструментов, облегчающих работу, 
таких как OpenRefine\sidenote[][-58pt]{%
%
\index{Обработка данных!OpenRefine}
OpenRefine~--- программа для обработки табличных данных. 
С 2021 года доступна редакторам Википедии в виде онлайн-сервиса (\url{https://openrefine.org}).%
} %
 или QuickStatements\sidenote[][-16pt]{%
%
\index{Обработка данных!QuickStatements}
QuickStatements~--- сервис пакетной обработки Викиданных\\
(\url{https://quickstatements.toolforge.org}).%
%
}, но регулярно повторяющиеся добавления и~вставки должны выполняться с помощью бота.


\section{Требования}
\label{sec:requirements}

\index{Библиотека!Pywikibot}
Ботов программируют на разных языках, мы для упрощения задачи воспользуемся библиотекой Pywikibot, 
написанной на языке Python и предназначенной для облегчения доступа 
к информации в~проектах Викимедиа. Есть три варианта запуска наших программ:
\begin{enumerate}[1)]
%  \setlength{\itemindent}{2em}
  \item использовать веб-оболочку наподобие PAWS\sidenote[][-48pt]{PAWS (a web shell)~--- веб-оболочка и сервис, включает блокноты Jupyter для~использования всеми участниками Викимедиа.}; 
  \index{Программирование!Среда разработки!Toolforge}
  \item установить облачную инфраструктуру наподобие Toolforge\sidenote[][-24pt]{%
%
          Toolforge~--- платформа энтузиастов-программистов, 
          разрабатывающих сервисы и программы для обработки данных 
          проектов Викимедиа (\url{https://wikitech.wikimedia.org/wiki/Portal:Toolforge}).%
%          
          };
  \item запустить на своём собственном компьютере.
\end{enumerate}

Благодаря наличию документации\sidenote[]{%
%
    См. руководство по установке Pywikibot: 
    \url{https://www.mediawiki.org/wiki/Manual:Pywikibot/Installation/ru}.%
%
} можно воспользоваться любым из этих трёх способов. 
Для~работы с ботами со своего компьютера нужно предварительно установить среду и язык программирования Python. 
После установки мы можем проверить, 
корректно ли она установлена, набрав

\begin{lstlisting}[ language=Python,
                    numbers=none,
                  ]
python --version
\end{lstlisting}
в командной строке Windows или в консоли Linux (в дальнейшем всё делаем в консоли), 
и тогда в~консоли мы увидим версию Python, с которой будем работать. 
Затем устанавливаем и~настраиваем Pywikibot. 



\newpage
Эта настройка Pywikibot включает несколько этапов:
\begin{itemize}
%  \setlength{\itemindent}{2em}
%\DrawEnumitemLabel
  \item Создайте файл конфигурации: \lstinline|py pwb.py generate_user_files|.
  \item Выполните команду входа в систему: \lstinline|py pwb.py login|.
  \item Просмотрите файлы \lstinline|user-config.py| и \lstinline|user-password.py|, где мы можем указать имена пользователей, которые будут вносить изменения.
\end{itemize}

Для просмотра информации или выполнения небольших тестов 
мы можем воспользоваться нашим именем пользователя\sidenote{%
    Имя пользователя (логин)~--- это то имя, под которым мы зарегистрировались в~проектах Викимедиа. 
    В этих проектах (Википедия, Викиданные, Викиверситет~и~др.) 
    используется единая пара логин/пароль.}, 
%
но в случае внесения большого количества изменений в Викиданные 
рекомендуется создать учетную запись бота и запросить разрешение на его запуск. 
Чтобы получить этот флаг, обычно требуются некоторые знания о редактировании в~проектах Викимедиа, 
подтверждающие, что мы не сделаем серьёзных ошибок и~у~нас есть поддержка других пользователей.

Обычно для получения этого флага нас просят выполнить определённое количество тестовых правок, 
чтобы убедиться, что такие правки полезны, а ошибок практически нет. 
Мы также можем продолжать выполнять задачи через нашу учётную запись, но с небольшой частотой правок.


\section{Наши первые скрипты}

\label{sec:firstScript}
\index{Библиотека!Pywikibot}
После того как мы правильно настроили Pywikibot, мы можем запустить 
нашу первую программу~(листинг~\ref{lst:page-get}). 
Перейдём в папку, где находится программный код Pywikibot, 
и создадим новый файл с именем \lstinline|test.py|.

Мы можем вносить изменения в файл при помощи любых текстовых редакторов 
или редакторов кода, таких как 
Visual Studio Code, Notepad++, или обычного редактора Notepad. 
Открыв файл, добавим в него следующий код~(листинг~\ref{lst:page-get}).


\begin{lstlisting}[ language=Python,
                    caption={Получение содержимого страницы Викиданных о реке Обнице в Польше},
                    label=lst:page-get,
                    xleftmargin=18pt, 
                    numbers=left,
                  ]
import pywikibot
site=pywikibot.Site('wikidata', "wikidata")
page=pywikibot.Page(site, u"Q16583338")     # river in Poland
print(page.get())
\end{lstlisting}

Сохраним и запустим этот файл, набрав следующую инструкцию: \lstinline|py pwb.py test.py|.

Отметим, что в Викиданных есть две реки с одинаковым названием Обница~--- 
одна в Польше, другая в Сербии. Одноимённые объекты легко различимы в Викиданных
по номерам: \wdqName{Obnitsa}{16583338} и~\wdqName{Obnitsa}{959190}.

\index{Библиотека!Pywikibot}
Если всё работает корректно, мы увидим в консоли весь текст, 
написанный на странице реки \wdqName{Обница}{16583338} в Викиданных. 
В первой строке программы~\ref{lst:page-get} мы импортировали библиотеку \lstinline|Pywikibot|. 
Во второй строке указали название проекта Викимедиа, над которым хотим работать, 
и~языковую версию проекта Викимедиа или особо~--- Викисклад\sidenote[][0pt]{%
%
    Например, Русская Википедия (код~ru) и Английская Википедия (код~en)~--- 
    это языковые версии Википедии. А~у~Викисклада (его также называют Commons) нет языковых версий и есть только один код: commons.%
    }. 
%
В~третьей строке указываем искомый объект и, наконец, указываем, что хотим получить весь контент.


Page (страница)~--- это модуль, включающий множество полезных методов\sidenote[][12pt]{%
%
    См. дополнительную информацию о модуле Page: 
    \url{https://doc.wikimedia.org/pywikibot/master/api_ref/pywikibot.html\#pywikibot.Page}.%
%
}.\,  В~дополнение к~методу \lstinline|get()| в~нашем распоряжении есть возможность 
узнать заголовок элемента с~помощью метода \lstinline|title()| (листинг~\ref{lst:page-title}).

\index{Библиотека!Pywikibot}
\begin{lstlisting}[ language=Python,
                    numbers=none,
                    caption={Получение заголовка объекта с номером Q16583338},
                    label=lst:page-title
                  ]
import pywikibot
site=pywikibot.Site('wikidata', "wikidata")
page=pywikibot.Page(site, u"Q16583338")     # river in Poland
print(page.title())
\end{lstlisting}

С помощью метода \lstinline|isRedirectPage()| 
мы можем узнать, является ли этот объект страницей-перенаправлением. 
Для этого в~листинге~\ref{lst:is-redirected} 
укажем объект, соответствующий странице-перена\-пра\-вле\-нию 
(\href{https://www.wikidata.org/w/index.php?title=Q16583333&redirect=no}
      {Q16583333 $\rightarrow$ Q2149254})\sidenote[][12pt]{%
%
    См. как выглядит страница-перенаправление: 
    \url{https://www.wikidata.org/w/index.php?title=Q16583333&redirect=no}.%
%    
}. 



\index{Библиотека!Pywikibot}
\begin{lstlisting}[ 
    language=Python,
    numbers=none,
    caption={Получение ответа на вопрос, является ли элемент страницей-перенаправлением\\
                             (истина или ложь)},
                    label=lst:is-redirected
                  ]
import pywikibot
site=pywikibot.Site('wikidata', "wikidata")
page=pywikibot.Page(site, u"Q16583333")
print(page.title())
print(page.isRedirectPage())
\end{lstlisting}

Обратите внимание, что до сих пор упражнения были очень простыми, но по мере усложнения мы должны быть осторожны с отступами, которые позволяют указать в языке Python, какая информация входит в структуры управления.

Предыдущее упражнение~(листинг~\ref{lst:is-redirected}) возвращает только истину или ложь. 
Такой запрос нужен, чтобы узнать, является ли заданная страница перенаправлением, 
и если <<да>>, 
то~предупредить об~этом пользователя~(листинг~\ref{lst:attention}).



\newpage
\begin{lstlisting}[ language=Python,
                    numbers=none,
                    caption={Предупреждение пользователя о том, что объект является страницей-перенаправлением},
                    label=lst:attention
                  ]
import pywikibot
site=pywikibot.Site('wikidata', "wikidata")
page=pywikibot.Page(site, u"Q16583333")    # redirect page
if (page.isRedirectPage()):
  print("Обратите внимание, это перенаправление, а не объект")
\end{lstlisting}

Но что произойдёт, если мы проанализируем несколько объектов и встретим несуществующий? 
Возможно, что программа вернёт ошибки, потому что не сможет проанализировать содержимое объекта. 
Мы можем это предусмотреть и отловить ошибки в Python, то есть проверить, существует~ли объект (листинг~\ref{lst:page-exists}).

\index{Библиотека!Pywikibot}
\begin{lstlisting}[ language=Python,
                    numbers=none,
                    caption={Проверка существования объекта},
                    label=lst:page-exists
                  ]
import pywikibot
site=pywikibot.Site('wikidata', "wikidata")
page=pywikibot.Page(site, u"Q107043778")    # non-existent object
if not page.exists():
  print("К сожалению, этот объект не существует!")
\end{lstlisting}
\marginnote[-2.0cm]{
    \MarginQuestion
    Измените программу~\ref{lst:page-exists} так, 
    чтобы отлавливать возможные ошибки с~помощью исключений (конструкция try~--- except).
}

%Анализируем объект дальше. 
%С помощью метода \lstinline|properties()| посмотрим, какие свойства он содержит 
%(листинг~\ref{lst:page-properties}).

Посмотрим с помощью метода \lstinline|properties()| 
в~листинге~\ref{lst:page-properties}, 
какие свойства содержит наш объект.

\index{Библиотека!Pywikibot}
\begin{lstlisting}[ language=Python,
                    numbers=none,
                    caption={Получение списка свойств объекта},
                    label=lst:page-properties
                  ]
import pywikibot
site=pywikibot.Site('wikidata', "wikidata")
page=pywikibot.Page(site, u"Q16583338")     # river in Poland
print(page.properties())
\end{lstlisting}

При выполнении программы~\ref{lst:page-properties} мы видим, что метод \lstinline|properties| 
возвращает словарь Python, то есть набор свойств и соответствующих им значений в этом объекте. 
Например, свойство \lstinline|page_image_free| возвращает изображение, видимое на странице объекта, 
свойство \lstinline|wb_claims|~--- количество утверждений (свойств), 
а свойство \lstinline|wb_sitelinks|~--- количество ссылок, связывающих этот объект с проектами Викимедиа.

Метод \lstinline|contributors()| также возвращает словарь, содержащий пользователей, 
редактировавших этот объект Викиданных. 
Он покажет нам имена (ники) этих редакторов, 
а~также число правок каждого из них. 
Ещё один метод \lstinline|revision_count()| возвращает общее количество правок в объекте. Оба этих метода задействованы в~листинге~\ref{lst:contributors-revision}.

\begin{lstlisting}[ language=Python,
                    numbers=none,
                    caption={Перечисление редакторов объекта Викиданных и числа правок},
                    label=lst:contributors-revision
                  ]
import pywikibot
site=pywikibot.Site('wikidata', "wikidata")
page=pywikibot.Page(site, u"Q16583338")     # river in Poland
print(page.contributors())
print(page.revision_count())
\end{lstlisting}

Существуют и другие методы, которые возвращают более сложную информацию, 
чем словарь данных. Например, при обращении к методу \lstinline|linkedPages| (связанные страницы) 
мы получим объект. 
Если мы просто обратимся к этому методу, как в листингe~\ref{lst:for-loop}, 
и выведем результат на~печать с~помощью функции \lstinline|print()|, то получим строчку: 
\lstinline|<pywikibot.data.api.PageGenerator object at 0x000002C6C7DB2FD0>|. 
Эта строка сообщает, что получен объект типа \lstinline|PageGenerator|. 
Прочитать данные этого объекта мы можем в цикле \lstinline|for| (листинг~\ref{lst:for-loop}).

\index{Библиотека!Pywikibot}
\begin{lstlisting}[ language=Python,
                    numbers=none,
                    caption={Чтение объекта типа \lstinline|PageGenerator|},
                    label=lst:for-loop
                  ]
import pywikibot
site=pywikibot.Site('wikidata', "wikidata")
page=pywikibot.Page(site, u"Q16583338")     # river in Poland
for linked in page.linkedPages():
  print(linked)
\end{lstlisting}

Итак, с помощью метода \lstinline|page.linkedPages()| мы получаем список тех ссылок, 
которые перечислены на исследуемой странице \lstinline|page|. 
Сначала будут показаны объекты, а затем свойства (включая информацию, содержащуюся в ссылках).

Чтобы узнать количество объектов, ссылающихся на заданный объект, 
воспользуемся методом \lstinline|backlinks()|, возвращающим массив объектов. 
Так же как в предыдущем случае (листинг~\ref{lst:for-loop}), 
обойдём этот список страниц-объектов 
и напечатаем их (листинг~\ref{lst:back-links}). 
Существуют страницы-сироты\sidenote[][0.0cm]{%
%
        О~статьях-сиротах см.~руководство на~странице 
        \href{https://ru.wikipedia.org/?curid=350049}{Википедия:Статьи-сироты}.  
        С~этими статьями связан внутренний проект Википедии по~повышению связности: \url{https://w.wiki/4cMm}.
%
},\, 
на~которые никто не~ссылается. 
Для таких объектов метод \lstinline|backlinks()| вернёт пустой список. 
Подберём такой объект в Викиданных (например, \wdqName{Pisueña}{6980876}), 
на который есть ссылки, чтобы было что выводить на печать в программе~\ref{lst:back-links}.


\newpage
\index{Библиотека!Pywikibot}
\begin{lstlisting}[ language=Python,
                    numbers=none,
                    caption={Использование метода \lstinline|backlinks()| для получения объектов,\\
                             ссылающихся на~заданный объект},
                    label=lst:back-links
                  ]
import pywikibot
site=pywikibot.Site('wikidata', "wikidata")
page=pywikibot.Page(site, u"Q6980876")     # river in Spain
for links in page.backlinks():
  print(links)
\end{lstlisting}

Итак, мы выбрали испанскую реку \wdqName{Pisueña}{6980876}, на которую ссылаются 
два других объекта Викиданных: это \wdqName{река Пас}{14360} и безымянная \wdqName{страница-перенаправле\-ние}{23993869}. 
И мы знаем число ссылок на~реку Pisueña. 

\marginnote[-1\baselineskip]{
    \MarginQuestion
    Напишите свою программу, 
    творчески объединив листинги~\ref{lst:page-get}--\ref{lst:back-links} 
    для~получения данных о каком-либо выбранном вами объекте Викиданных.%
}

\index{Библиотека!Pywikibot}
Кроме модуля \lstinline|Page|, 
библиотека Pywikibot даёт нам множество других инструментов. 
Отметим, что работа только с одним объектом малополезна. 
Обычно в одном скрипте анализируют и меняют сразу несколько объектов. 
Для этого нам пригодятся все те запросы, которые мы написали в этой главе.





\section{SPARQL-запросы внутри программы на Python}
\label{sec:running-queries}

Сначала откроем страницу сервиса \href{https://query.wikidata.org/}{WDQS}\footnote[][0pt]{%
%    
    См. пояснения о службе WDQS на c.~\pageref{sect:WDQS}.%
%    
}  и~напишем свой запрос на языке SPARQL. 
Викиданные заполняются в основном вручную, поэтому неизбежны ошибки. 
Попробуем найти с помощью SPARQL-запроса такую ошибку. 
Например, часто у людей (соответствующих им объектов на Викиданных) 
вместо свойства \wdProperty{27}{гражданство} 
ошибочно указывают свойство \wdProperty{17}{страна}. 
Запрос для поиска такой ошибки может выглядеть так~(листинг~\ref{lst:bug-finding}):

\begin{lstlisting}[ language=SPARQL,
        numbers=none,
        caption={\href{https://w.wiki/https://w.wiki/4cBg}
                      {Запрос для получения персон с ошибочным отнесением к стране\\
                       вместо указания гражданства}\protect\footnotemark},
        label=lst:bug-finding
                  ]
SELECT ?item ?itemLabel WHERE 
{
  ?item wdt:P31 wd:Q5;  # instance of human
        wdt:P17 wd:Q36. # has country Poland
  SERVICE wikibase:label {bd:serviceParam wikibase:language "ru,en"}
}
\end{lstlisting}
\footnotetext{Получено: 3 профессии в 2021 году. Ссылка на SPARQL-запрос: \href{https://w.wiki/4cBg}{https://w.wiki/4cBg}.}


\newpage
Мы должны проверить, возвращает ли запрос~\ref{lst:bug-finding} какое-либо значение, 
поскольку может оказаться, что ошибок нет. 
В этом случае мы можем изменить страну. 
Например, можем указать следующую непроверенную страну Литву (Q37) 
вместо проверенной Польши (Q36), 
поскольку наверняка будут другие страны с такой же ошибкой.

После того как мы построили SPARQL-запрос~\ref{lst:bug-finding} для получения объектов-людей с~ошибками, 
мы можем добавить запрос в наш листинг~\ref{lst:identify-bugs} на языке Python.

% caption=[Идентификация персон с ошибкой страна-гражданство]{Включение SPARQL-запроса по идентификации персон с ошибкой страна-гражданство на примере Польши (Q36) в программу на языке Python, с выводом имён всех таких людей},
%\begin{fullwidth}
\index{Библиотека!Pywikibot}
\begin{lstlisting}[ language=Python,
    numbers=none,
    caption={Идентификация персон с ошибкой страна~--- гражданство на примере Польши (Q36) (включение SPARQL-запроса в программу Python)},
    label=lst:identify-bugs
    ]
import pywikibot
from pywikibot import pagegenerators
site=pywikibot.Site('wikidata', "wikidata")
query = """
  SELECT DISTINCT ?item WHERE {
    SERVICE wikibase:label {bd:serviceParam wikibase:language "ru,en"}
    ?item wdt:P31 wd:Q5.
    ?item wdt:P17 wd:Q36.
  } """

pages = pagegenerators.WikidataSPARQLPageGenerator(query,site=site)
for item in pages:
  print(item.title())
\end{lstlisting}    
%\end{fullwidth}

Из предыдущих упражнений мы взяли только модуль \lstinline|pagegenerators|, 
который позволяет нам получить список объектов с помощью фильтра в виде SPARQL-запроса 
(листинг~\ref{lst:bug-finding}). 
Мы сохранили текст запроса в переменной \lstinline|query| (листинг~\ref{lst:identify-bugs}), 
а затем передали её в метод генераторов страниц \lstinline|WikidataSPARQLPageGenerator()|. 
Данные, возвращаемые этим методом, были обработаны с помощью цикла \lstinline|for|, 
и мы напечатали названия объектов.
Таким образом, эта программа~(листинг~\ref{lst:identify-bugs}) печатает список имён людей, у которых в свойстве <<страна>> указана Польша.

Мы уже научились получать несколько элементов сразу. Займёмся этим. 
Вместо того чтобы отображать заголовок страницы, отобразим некоторые её части. 
Например, ссылки на проекты Викимедиа, синонимы, метки или утверждения~(листинг~\ref{lst:display-linkslabels}).



\newpage
\index{Библиотека!Pywikibot}
\begin{lstlisting}[ language=Python,
                    numbers=none,
                    caption={Отображение ссылок, синонимов, меток и утверждений для~персон из~Польши},
                    label=lst:display-linkslabels
                  ]
import pywikibot
from pywikibot import pagegenerators
site=pywikibot.Site('wikidata', "wikidata")
query = """
  SELECT DISTINCT ?item WHERE {
    SERVICE wikibase:label {bd:serviceParam wikibase:language "ru,en"}
    ?item wdt:P31 wd:Q5.
    ?item wdt:P17 wd:Q36.
  } """

pages = pagegenerators.WikidataSPARQLPageGenerator(query, site=site)
for item in pages:
  item.get()
  print(item.sitelinks)
  print(item.aliases)
  print(item.labels)
  print(item.claims)
\end{lstlisting} 

\marginnote[-2\baselineskip]{
        \MarginQuestion
        Наберите и запустите в консоли на своём компьютере 
        предложенные в~этой главе листинги программ. 
        Текст, который будут отображать эти программы в~консоли, 
        поможет вам понять, как они работают.%
}

У нас получилось извлечь информацию из объектов Викиданных, теперь попробуем изменить её. 




\section{Изменение значений, введённых в Викиданные}
\label{sec:modifying the values entered in Wikidata}

Поиск повторяющихся с течением времени ошибок~--- одна из задач программ-ботов. 
Ботов регулярно запускают, они ищут и исправляют ошибки. 
Разработчики ботов стараются улучшать поиск ошибок и делать работу ботов более надёжной. 
Рассмотрим следующий запрос, который возвращает все профессии, 
отсортированные по количеству их использования в Викиданных~(листинг~\ref{lst:occupation-list}).



\newpage
\begin{lstlisting}[ language=SPARQL,
                    numbers=none,
                    caption={\href{https://w.wiki/4bLo}{Cписок профессий, отсортированных по количеству их использования в~Викиданных}\protect\footnotemark},
                    label=lst:occupation-list
                  ]
SELECT ?instanceLabel ?value WHERE {
  {
    SELECT ?instance (COUNT(DISTINCT ?item) AS ?value) 
    WHERE { ?item wdt:P106 ?instance. } # P106 - occupation
    GROUP BY ?instance
    ORDER BY DESC (?value)
  }
  SERVICE wikibase:label {bd:serviceParam wikibase:language "ru,en"}
}
ORDER BY DESC (?value)
\end{lstlisting} 
\footnotetext{Получено: \num{16979} профессий в 2021 году. Ссылка на SPARQL-запрос: \href{https://w.wiki/4bLo}{https://w.wiki/4bLo}.}

Запрос~\ref{lst:occupation-list} позволяет найти ошибки 
(случайные или намеренные) в~значении свойства \wdProperty{106}{профессия}. 
В полученном списке у некоторых объектов в свойстве <<профессия>> 
указана неверная или бессмысленная информация, 
например 
\wdqName{Salinas de Ibargoiti}{7404672}, 
\wdqName{Order of Saint Basil the Great}{7319129}, 
\wdqName{Promoted to Glory}{7249866}, 
\wdqName{Princes}{7244433} и 
\wdqName{Point Loma Nazarene Uni\-versity}{7208069}. 
Мы можем составить список этих ошибок (массив \lstinline|delete_occupation|)
и найти объекты тех персон, 
у~которых в~поле <<профессия>> есть ошибки~(листинг~\ref{lst:occupation-mistakeslist}).




\newpage
\index{Библиотека!Pywikibot}
\begin{lstlisting}[ language=Python,
                    caption={Отображение персон с ошибками в поле <<профессия>>},
                    label=lst:occupation-mistakeslist,
                    xleftmargin=18pt, 
                    numbers=left,
                  ]
import pywikibot
from pywikibot import pagegenerators
site=pywikibot.Site('wikidata', "wikidata")
delete_occupation={"Q7404672", "Q7319129", "Q7249866", "Q7244433", 
"Q7208069"}
for occupation in delete_occupation:
  query = """
    SELECT DISTINCT ?item WHERE {
      SERVICE wikibase:label {bd:serviceParam wikibase:language "ru,en".}
      ?item wdt:P31 wd:Q5.     # is human
      ?item wdt:P106 wd:""" + occupation + """.
    } """

  pages=pagegenerators.WikidataSPARQLPageGenerator(query, site=site)
  for item in pages:
    print("Люди, чья профессия {occupation}: {title}".format(occupation=occupation, title=item.title()))
\end{lstlisting} 
\marginnote[-2.0cm]{
        \MarginNB
        Обратите внимание, 
        что до сих пор мы обращались к функции \mbox{\lstinline|print()|} 
        для печати значений переменных, 
        но в~листинге~\ref{lst:occupation-mistakeslist} 
        для~большей наглядности использована функция \mbox{\lstinline|format()|}.%
}

Итак, список <<ошибочных>> профессий (объектов Викиданных), подлежащих удалению, 
сохранён в~массиве \lstinline|delete_occupation|. 
Мы обходим этот список с помощью цикла \lstinline|for|, 
чтобы с помощью SPARQL-запроса получить список людей для каждой <<подозрительной>> профессии. 
В~отличие от~листинга~\ref{lst:occupation-list}, 
в~листинге~\ref{lst:occupation-mistakeslist} есть переменная \lstinline|occupation|, 
которая берёт значение из массива \lstinline|delete_occupation| и подставляет его в SPARQL-запрос.

Теперь, если мы уверены, что информация о профессии неверна, 
и~хотим её удалить, то~достаточно указать свойство, которое нужно удалить.

Есть ещё одна трудность. 
Дело в том, что персона может иметь несколько профессий, а удалить мы хотим только одну. 
Для удаления этой одной профессии нам нужно знать её значение. 
Пример такого удаления профессии по~известному значению 
показан в~листинге~\ref{lst:delete-occupation}. 



\newpage
\index{Библиотека!Pywikibot}
%\begin{fullwidth}
\begin{lstlisting}[ language=Python,
                    caption={Корректное удаление значения в поле «профессия>>},
                    label=lst:delete-occupation, 
                    numbers=none
                  ]
import pywikibot
from pywikibot import pagegenerators
site=pywikibot.Site('wikidata', "wikidata")
delete_occupation={"Q7404672", "Q7319129", "Q7249866", "Q7244433", 
"Q7208069"}
for occupation in delete_occupation:
  query = """
    SELECT DISTINCT ?item WHERE {
      SERVICE wikibase:label {bd:serviceParam wikibase:language "ru,en"}
      ?item wdt:P31 wd:Q5.
      ?item wdt:P106 wd:""" + occupation + """.
    } """

  pages = pagegenerators.WikidataSPARQLPageGenerator(query, site=site)
  for item in pages:
    print("Люди, чья профессия {occupation}: {title}".format(occupation=occupation, title=item.title()))
    item.get()

    for valor in item.claims['P106']:
      if (str(valor.getTarget())=='[[wikidata:' + occupation + ']]'):
        print("<<<<< Удалить {occupation} от {title} >>>>>" . format(occupation=occupation, title=item.title()))
        item.removeClaims(valor, summary=u'Удаление ошибочных знач. в P106')
\end{lstlisting} 
%\end{fullwidth}

\marginnote[-2.0cm]{
        \MarginQuestion
        Добавьте больше ошибочных и неправильных профессий 
        в переменную \lstinline|delete_occupation| 
        и запустите бота для удаления этих свойств.%
}

В листинге~\ref{lst:delete-occupation} 
мы сохранили код листинга~\ref{lst:occupation-mistakeslist}, 
но добавили последние строки, 
в~которых просматриваем значения, содержащиеся в свойстве \wdProperty{106}{}, 
с помощью свойства \mbox{\lstinline|claims|.} 
Если это совпадает с искомой профессией, мы удаляем его методом \mbox{\lstinline|removeClaims|.} 
При~вызове этого метода нужно указать удаляемое значение и комментарий к редакторской правке\sidenote[]{%
        Комментарии к правкам (или~summary) позволяют другим редакторам быстро понять, 
        какие изменения были внесены на~страницу или в~объект Викиданных.%
%
}, сохраняемый в истории правок вики-страницы.


Обратите внимание, 
что такие изменения могут вызвать множество проблем с Викиданными 
и~администраторы могут заблокировать вашу учётную запись. 
Вначале внесите несколько изменений и~всегда проверяйте историю изменений, 
чтобы узнать, какая информация была изменена.






%\renewcommand{\lstlistingname}{Запрос}  % Листинг -> Запрос
\makeatletter
\long\def\@makecaption#1#2{%
    \renewcommand{\lstlistingname}{Запрос}
\centering%
#1. #2\par%
\vspace{5pt}%
}%
\makeatother
