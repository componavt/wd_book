\chapter{Боты в Викиданных}
\label{ch:bots}
В этой главе рассматривается автоматизация процессов в Викиданных. Во многих случаях мы хотим исправить повторяющиеся ошибки или ввести большие объемы данных в Викиданные вместо того, чтобы изменять свойства по одному. Для ввода данных в нашем распоряжении имеется несколько инструментов, облегчающих нашу работу, таких как OpenRefine или QuickStatements, но повторяющиеся изменения и вставки со временем должны выполняться с помощью бота.


\section{Требования}
\label{sec:requirements}
Мы можем запрограммировать бота на нескольких языках программирвания, но, чтобы облегчить нашу задачу, мы можем использовать Pywikibot, набор инструментов, запрограммированных на Python для облегчения доступа к информации в проектах Фонда Викимедиа. У нас есть три варианта запуска наших программ:
\begin{enumerate}
  \setlength{\itemindent}{2em}
  \item использовать веб-оболочку, такую как PAWS,
  \item создать облаччную инфраструктуру, такую как Toolforge, или 
  \item запустить на нашем собственном компьютере.
\end{enumerate}