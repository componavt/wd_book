\chapter{Боты в Викиданных}
\label{ch:bots}

В этой главе рассматривается автоматизация процессов в Викиданных. 
Часто нужно исправить повторяющиеся ошибки или ввести большие объёмы данных 
вместо того, чтобы изменять свойства по одному. 
Для ввода данных в нашем распоряжении есть несколько инструментов, облегчающих работу, 
таких как OpenRefine\sidenote[][-58pt]{%
%
OpenRefine~--- это программа для обработки табличных данных. 
С 2021 года эта программа доступна редакторам Википедии в виде онлайн-сервиса. \url{https://openrefine.org/}%
} %
 или QuickStatements\sidenote[][-16pt]{%
%
QuickStatements~--- это сервис пакетной обработки Викиданных. \url{https://quickstatements.toolforge.org/}%
%
}, но регулярно повторяющиеся добавления и вставки должны выполняться с помощью бота.


\section{Требования}

\label{sec:requirements}
Ботов программируют на разных языках, но для упрощения задачи воспользуемся библиотекой Pywikibot, 
написанной на языке Python и предназначенной для облегчения доступа 
к информации в проектах Викимедиа. Есть три варианта запуска наших программ:
\begin{enumerate}
%  \setlength{\itemindent}{2em}
  \item использовать веб-оболочку наподобие PAWS\sidenote[][-48pt]{PAWS (a web shell)~--- это веб-оболочка и сервис, в котором размещаются блокноты Jupyter для использования всеми участниками Викимедиа}; 
  \item установить облачную инфраструктуру наподобие Toolforge\sidenote[][-24pt]{Toolforge~--- это платформа энтузиастов-программистов, разрабатывающих сервисы и программы для обработки данных проектов Викимедиа. \url{https://wikitech.wikimedia.org/wiki/Portal:Toolforge}};
  \item запустить на своём собственном компьютере.
\end{enumerate}

Благодаря наличию документации\sidenote[]{См. руководство по установке Pywikibot: \url{https://w.wiki/4CsU}} 
можно воспользоваться любым из этих трёх способов. 
Для работы с ботами со своего компьютера нужно предварительно установить среду и язык программирования Python. 
После установки мы можем проверить, 
корректно ли она установлена, 
набрав\marginnote[0.5cm]{Документация по установке: \url{https://www.mediawiki.org/wiki/Manual:Pywikibot/Installation}}

\begin{lstlisting}[ language=Python,
                    numbers=none,
                  ]
python --version
\end{lstlisting}
в командной строке Windows или в консоли Linux (в дальнейшем всё делаем в консоли), 
и тогда в консоли мы увидим версию Python, с которой будем работать. Затем устанавливаем Pywikibot и настраиваем его. 

\newpage
Эта настройка включает несколько этапов:
\begin{itemize}
  \setlength{\itemindent}{2em}
  \item Выполните команду для создания файла конфигурации:\newline \lstinline|py pwb.py generate_user_files|.
  \item Выполните команду входа в систему: \lstinline|py pwb.py login|.
  \item Просмотрите файлы \lstinline|user-config.py| и \lstinline|user-password.py|, где мы можем указать имена пользователей, которые будут вносить изменения.
\end{itemize}

Для просмотра информации или выполнения небольших тестов мы можем воспользоваться нашим именем пользователя\sidenote{Имя пользователя (логин)~--- это то имя, под которым мы зарегистрировались в проектах Викимедиа. В этих проектах (Википедия, Викиданные, Викиверситет и другие) используются единые логин-пароль.}, но в случае внесения большого количества изменений в Викиданные рекомендуется создать учетную запись бота и запросить разрешение на его запуск. Чтобы получить этот флаг, обычно требуются некоторые знания о редактировании в проектах Викимедиа, подтверждающие, что мы не сделаем серьёзных ошибок и у нас есть поддержка других пользователей.

Обычно для получения этого флага нас просят выполнить определённое количество тестовых правок, чтобы убедиться, что такие правки полезны, а ошибок практически нет. Мы также можем продолжать выполнять задачи через нашу учётную запись, но с небольшой частотой правок.


\section{Наши первые скрипты}

\label{sec:firstScript}
После того как мы правильно настроили Pywikibot, мы можем запустить 
нашу первую программу~(листинг~\ref{lst:page-get}). 
Перейдём в папку, где находится Pywikibot, и создадим новый файл с именем \lstinline|test.py|.

Мы можем вносить изменения в файл при помощи любых текстовых редакторов 
или редакторов кода, таких как: 
Visual Studio Code, Notepad++ или обычного редактора Notepad. 
Открыв файл, добавим в него следующий код~(листинг~\ref{lst:page-get}).

\marginnote[2\baselineskip]{Отметим, что в Викиданных есть две реки с одинаковым названием <<Обница>>
одна в Польше, другая в Сербии. Одноимённые объекты легко различимы в Викиданных
по номерам: \wdqName{Obnitsa}{16583338} и \wdqName{Obnitsa}{959190}.}
%
\begin{lstlisting}[ language=Python,
                    caption={Получение содержимого страницы Викиданных о реке Обнице в Польше},
                    label=lst:page-get
                  ]
import pywikibot
site=pywikibot.Site('wikidata', "wikidata")
page=pywikibot.Page(site, u"Q16583338")     # river in Poland
print(page.get())
\end{lstlisting}

Сохраним и запустим этот файл, набрав следующую инструкцию: \lstinline|py pwb.py test.py|.

Если всё работает корректно, мы увидим в консоли весь текст, 
написанный на странице реки \wdqName{Обница}{16583338} в Викиданных. 
В первой строке программы~\ref{lst:page-get} мы импортировали библиотеку \lstinline|Pywikibot|. 
Во второй строке указали название проекта Викимедиа, над которым хотим работать, 
и языковую версию проекта Викимедиа или особо~--- Викисклад (Commons)\sidenote[][-28pt]{%
%
Например, Русская Википелия (код~``ru'') и Английская Википедия (код~``en'')~--- 
это языковые версии Википедии. А у Викисклада нет языковых версий и есть только один код: ``commons''.}. 
%
В строке 3 указываем искомый объект и, наконец, указываем, что хотим получить весь контент.


``Page'' (страница)~--- это модуль, включающий множество полезных методов\sidenote[][-14pt]{%
%
См. дополнительную информацию о модуле ``Page'' \url{https://doc.wikimedia.org/pywikibot/master/_modules/pywikibot/page.html}}. 
%
В дополнение к методу \lstinline|get()| в нашем распоряжении есть возможность 
узнать заголовок элемента с помощью метода \lstinline|title()|~(листинг~\ref{lst:page-title}).

\begin{lstlisting}[ language=Python,
                    numbers=none,
                    caption={Получение заголовка объекта с номером Q16583338},
                    label=lst:page-title
                  ]
import pywikibot
site=pywikibot.Site('wikidata', "wikidata")
page=pywikibot.Page(site, u"Q16583338")     # river in Poland
print(page.title())
\end{lstlisting}

Мы также можем узнать, является ли этот объект страницей-перенаправлением
с помощью метода \lstinline|isRedirectPage()|. 
Для этого заменим предыдущий объект на другой 
(\href{https://www.wikidata.org/w/index.php?title=Q16583333&redirect=no}{Obnitsa (Q16583338)}),
являющийся страницей-перенаправлением (листинг~\ref{lst:is-redirected}).

\begin{lstlisting}[ language=Python,
                    numbers=none,
                    caption={Получение ответа на вопрос, является ли элемент страницей перенаправления (истина или ложь)},
                    label=lst:is-redirected
                  ]
import pywikibot
site=pywikibot.Site('wikidata', "wikidata")
page=pywikibot.Page(site, u"Q16583333")
print(page.title())
print(page.isRedirectPage())
\end{lstlisting}

Обратите внимание, что до сих пор упражнения были очень простыми, но по мере усложнения мы должны быть осторожны с отступами, которые позволяют указать в языке Python, какая информация входит в структуры управления.

Предыдущее упражнение~(листинг~\ref{lst:is-redirected}) возвращает только истину или ложь. 
Такой запрос нужен, чтобы узнать, является ли заданная страница перенаправлением и если <<да>>, 
то предупредить пользователя~(листинг~\ref{lst:attention}).

\begin{lstlisting}[ language=Python,
                    numbers=none,
                    caption={Предупреждение пользователя о том, что объект является страницей-перенаправлением},
                    label=lst:attention
                  ]
import pywikibot
site=pywikibot.Site('wikidata', "wikidata")
page=pywikibot.Page(site, u"Q16583333")    # redirect page
if (page.isRedirectPage()):
  print("Обратите внимание, это перенаправление, а не объект")
\end{lstlisting}

Но что произойдёт, если мы проанализируем несколько объектов и встретим несуществующий? 
Возможно, что программа вернёт ошибки, потому что не сможет проанализировать содержимое объекта. 
Мы можем это предусмотреть и отловить ошибки в Python, то есть проверить, существует ли объект (листинг~\ref{lst:page-exists}).

\begin{lstlisting}[ language=Python,
                    numbers=none,
                    caption={Проверка существования объекта},
                    label=lst:page-exists
                  ]
import pywikibot
site=pywikibot.Site('wikidata', "wikidata")
page=pywikibot.Page(site, u"Q107043778")    # non-existent object
if not page.exists():
  print("К сожалению, этот объект не существует!")
\end{lstlisting}
\marginnote[-2.0cm]{Упражнение. Измените программу~\ref{lst:page-exists} так, 
чтобы отлавливать возможные ошибки с помощью исключений.}

Анализируем объект дальше. Посмотрим, какие свойства он содержит 
с помощью метода \lstinline|properties()| (листинг~\ref{lst:page-properties}).

\begin{lstlisting}[ language=Python,
                    numbers=none,
                    caption={Получение списка свойств объекта},
                    label=lst:page-properties
                  ]
import pywikibot
site=pywikibot.Site('wikidata', "wikidata")
page=pywikibot.Page(site, u"Q16583338")     # river in Poland
print(page.properties())
\end{lstlisting}

При выполнении программы~\ref{lst:page-properties} мы видим, что метод \lstinline|properties| 
возвращает словарь Python, то есть набор свойств и соответствующих им значений в этом объекте. 
Например, свойство \lstinline|page_image_free| возвращает изображение, видимое на странице объекта, 
свойство \lstinline|wb_claims| содержит количество утверждений (свойств), содержащихся в объекте, 
а свойство \lstinline|wb_sitelinks| содержит количество ссылок, связывающих этот объект с проектами Викимедиа.

Метод \lstinline|contributors()| также возвращает словарь, содержащий пользователей, 
редактировавших этот объект Викиданных. 
Он покажет нам имена (ники) этих редакторов, 
а также число правок каждого из них. 
Ещё один метод \lstinline|revision_count| возвращает общее количество правок в объекте. Оба этих метода задействованы в~листинге~\ref{lst:contributors-revision}.

\begin{lstlisting}[ language=Python,
                    numbers=none,
                    caption={Перечисление редакторов объекта Викиданных и числа правок},
                    label=lst:contributors-revision
                  ]
import pywikibot
site=pywikibot.Site('wikidata', "wikidata")
page=pywikibot.Page(site, u"Q16583338")     # river in Poland
print(page.contributors())
print(page.revision_count())
\end{lstlisting}

Существуют и другие методы, которые возвращают более сложную информацию, 
чем словарь данных, например, мы получим объект при обращении 
к методу \lstinline|linkedPages| (связанные страницы). Если мы просто обратимся к этому методу как в листингe~\ref{lst:for-loop} и выведем результат на печать с помощью функции \textit{print()}, то получим строчку: 

<pywikibot.data.api.PageGenerator object at 0x000002C6C7DB2FD0>.
Эта строка сообщает, что получен объект типа \textit{PageGenerator}. Прочитать данные этого объекта мы можем в цикле \textit{for}~(листинг~\ref{lst:for-loop}):

\begin{lstlisting}[ language=Python,
                    numbers=none,
                    caption={Чтение объекта типа \textit{PageGenerator}},
                    label=lst:for-loop
                  ]
import pywikibot
site=pywikibot.Site('wikidata', "wikidata")
page=pywikibot.Page(site, u"Q16583338")     # river in Poland
for linked in page.linkedPages():
  print(linked)
\end{lstlisting}

Итак, с помощью метода \textit{page.linkedPages()} мы получаем список тех ссылок, которые перечислены на исследуемой странице \textit{page}. Сначала будут показаны объекты, а затем свойства (включая информацию, содержащуюся в ссылках).

Чтобы узнать количество объектов, ссылающихся на заданный объект, воспользуемся методом \textit{backlinks()}, возвращающим массив объектов. Так же как в предыдущем случае~(листинг~\ref{lst:for-loop}) обойдём этот список страниц-объектов и напечатаем их~(листинг~\ref{lst:back-links}). Существуют объекты-страницы, на которые никто не ссылается. Для таких объектов метод \textit{backlinks()} вернёт пустой список. Поэтому мы подберём такой объект (\wdqName{Pisueña}{6980876}), у которого этот список не будет пустым~(листинг~\ref{lst:back-links}).

\begin{lstlisting}[ language=Python,
                    numbers=none,
                    caption={Использование метода \textit{backlinks()} для получения объектов, ссылающихся на заданный объект},
                    label=lst:back-links
                  ]
import pywikibot
site=pywikibot.Site('wikidata', "wikidata")
page=pywikibot.Page(site, u"Q6980876")     # river in Cantabria, Spain
for links in page.backlinks():
  print(links)
\end{lstlisting}

В этом случае (листинг~\ref{lst:back-links}) мы выбрали другую реку \\(\wdqName{Pisueña}{6980876}), которая связана дважды: другой рекой \\(\wdqName{Obnitsa}{16583338}) и страницей перенаправления \wdqName{}{16583333}. Таким образом, мы можем узнать, сколько существует связей у определённого элемента с другими элементами Викиданных.

\marginnote[0.0cm]{Упражнение: примените на практике всё, что было представлено ранее, но используя элемент с небольшим номером, который, вероятно, содержит гораздо больше информации, чем примеры, которые мы использовали до сих пор. О нумерации объектов возможно прочитать на с.~\pageref{WDObjectsNumbering}.}

Кроме модуля \textit{Page} Pywikibot даёт нам множество других инструментов. Отметим, что работа только с одним объектом малополезна. Обычно в одном скрипте анализируют и меняют сразу несколько объектов. Для этого нам пригодятся все те запросы, которые мы написали в этой главе.


\section{SPARQL-запросы внутри программы на Python}
\label{sec:running queries}

Сначала откроем страницу сервиса \href{https://query.wikidata.org/}{WDQS}\footnote[][0pt]{%
См. пояснения о службе WDQS на c.~\pageref{sect:WDQS}.%
}%
 и напишем свой запрос на языке SPARQL. 
Викиданные заполняются в основном вручную, поэтому неизбежны ошибки. 
Попробуем найти с помощью SPARQL-скрипта такую ошибку. 
Часто у людей (соответствующих им объектов на Викиданных) 
вместо свойства \wdProperty{27}{гражданство} указывают свойство \wdProperty{17}{страна}. 
Запрос для поиска такой ошибки может выглядеть так~(листинг~\ref{lst:bug-finding}).

\begin{lstlisting}[ language=SPARQL,
        numbers=none,
        caption={\href{https://w.wiki/https://w.wiki/4cBg}{Запрос для получения персон с ошибочным отнесением к стране, вместо указания гражданства}\protect\footnotemark},
        label=lst:bug-finding
                  ]
SELECT ?item ?itemLabel WHERE 
{
  ?item wdt:P31 wd:Q5;  # instance of human
        wdt:P17 wd:Q36. # has country Poland
  SERVICE wikibase:label {bd:serviceParam wikibase:language "ru,en"}
}
\end{lstlisting}
\footnotetext{Получили 3 профессии в 2021 году. Ссылка на SPARQL-запрос: \href{https://w.wiki/4cBg}{https://w.wiki/4cBg}}

Мы должны проверить, возвращает ли запрос~(листинг~\ref{lst:bug-finding}) какое-либо значение, поскольку может оказаться, что ошибок нет. В этом случае мы можем изменить страну (например, указать следующую страну (Q37~--- Литва), вместо проверенной (Q36~--- Польша)), поскольку наверняка будут другие страны с такой же ошибкой.

После того, как мы построили запрос для получения объектов ВД людей с ошибками~(листинг~\ref{lst:bug-finding}), мы можем добавить запрос в нашу программу~(листинг~\ref{lst:identify-bugs}).

\begin{fullwidth}
\begin{lstlisting}[ language=Python,
                    numbers=none,
                    caption={Включение SPARQL-запроса по идентификации персон с ошибкой страна-гражданство на примере Польши (Q36) в программу на языке Python, с выводом имён всех таких людей},
                    label=lst:identify-bugs
                  ]
import pywikibot
from pywikibot import pagegenerators
site=pywikibot.Site('wikidata', "wikidata")
query = """
  SELECT DISTINCT ?item WHERE {
    SERVICE wikibase:label {bd:serviceParam wikibase:language "ru,en"}
    ?item wdt:P31 wd:Q5.
    ?item wdt:P17 wd:Q36.
  } """

pages = pagegenerators.WikidataSPARQLPageGenerator(query,site=site)
for item in pages:
  print(item.title())
\end{lstlisting}    
\end{fullwidth}

Из предыдущих упражнений мы взяли только \textit{pagegenerators}, модуль, который позволяеn нам получить список объектов с помощью фильтра в виде SPARQL-запроса (листинг~\ref{lst:bug-finding}). Мы сохранили текст запроса в переменной /textit{query}, а затем передали её в метод генераторов страниц \textit{WikidataSPARQLPageGenerator()}. Данные, возвращаемые этим методом, были обработаны с помощью цикла for, и мы напечатали названия объектов.

Таким образом, эта программа (~(листинг~\ref{lst:identify-bugs})) печатает список имён людей, у которых в свойстве <<страна>> указана Польша.

Мы уже научились получать несколько элементов сразу. Займёмся этим. Вместо того, чтобы отображать заголовок страницы, отобразим некоторые её части. Например, ссылки на проекты Викимедиа, синонимы, метки или утверждения~(листинг~\ref{lst:display-linkslabels}).

\begin{lstlisting}[ language=Python,
                    numbers=none,
                    caption={Отображение ссылок, синонимов, меток и утверждений},
                    label=lst:display-linkslabels
                  ]
import pywikibot
from pywikibot import pagegenerators
site=pywikibot.Site('wikidata', "wikidata")
query = """
  SELECT DISTINCT ?item WHERE {
    SERVICE wikibase:label {bd:serviceParam wikibase:language "ru,en"}
    ?item wdt:P31 wd:Q5.
    ?item wdt:P17 wd:Q36.
  } """

pages = pagegenerators.WikidataSPARQLPageGenerator(query, site=site)
for item in pages:
  item.get()
  print(item.sitelinks)
  print(item.aliases)
  print(item.labels)
  print(item.claims)
\end{lstlisting} 

\marginnote[0.0cm]{Упражнение: наберите и запустите в консоли на своём компьютере предложенные в этой главе листинги программ. Текст, который будут печатать эти программы в консоли, поможет вам понять, как они работают}

У нас получилось извлечь информацию из объектов Викиданных, теперь попробуем изменить её: 

\section{Изменение значений, введённых в Викиданные}
\label{sec:modifying the values entered in Wikidata}

Поиск повторяющихся с течением времени ошибок~--- одна из задач компьютерной программы-бота, потому что в зависимости от задачи, выполняемой ботом, ее придется регулярно выполнять, и постепенно улучшать поиск ошибок. Например, следующий запрос возвращает все профессии, отсортированные по количеству их использования в Викиданных~(листинг~\ref{lst:occupation-list}).

\begin{lstlisting}[ language=SPARQL,
                    numbers=none,
                    caption={\href{https://w.wiki/4bLo}{Cписок профессий, отсортированных по количеству их использования в Викиданных}\protect\footnotemark},
                    label=lst:occupation-list
                  ]
SELECT ?instanceLabel ?value WHERE {
  {
    SELECT ?instance (COUNT(DISTINCT ?item) AS ?value) 
    WHERE { ?item wdt:P106 ?instance. } #P106 - occupation
    GROUP BY ?instance
    ORDER BY DESC (?value)
  }
  SERVICE wikibase:label {bd:serviceParam wikibase:language "ru,en"}
}
ORDER BY DESC (?value)
\end{lstlisting} 
\footnotetext{Получили \num{16979} профессий в 2021 году. Ссылка на SPARQL-запрос: \href{https://w.wiki/4bLo}{https://w.wiki/4bLo}}

Этот запрос~\ref{lst:occupation-list} 
можно использовать для проверки того, какие профессии указаны с ошибками или были намеренно испорчены вандалами. 
Просматривая список, мы видим, что есть некоторые профессии, 
которые неверны и содержат бессмысленную информацию, 
например, \wdqName{Salinas de Ibargoiti}{7404672}, \wdqName{Order of Saint Basil the Great}{7319129}, \wdqName{Promoted to Glory}{7249866}, \wdqName{Princes}{7244433} и \wdqName{Point Loma Nazarene University}{7208069}. 
Мы можем составить список этих ошибок и найти объекты тех персон, 
в которых указаны эти профессии с ошибками~(листинг~\ref{lst:occupation-mistakeslist}).

\begin{lstlisting}[ language=Python,
                    caption={Составление списка ошибок и отображение персон с ошибками в профессиях},
                    label=lst:occupation-mistakeslist
                  ]
import pywikibot
from pywikibot import pagegenerators
site=pywikibot.Site('wikidata', "wikidata")
delete_occupation={"Q7404672", "Q7319129", "Q7249866", "Q7244433", 
"Q7208069"}
for occupation in delete_occupation:
  query = """
    SELECT DISTINCT ?item WHERE {
      SERVICE wikibase:label { bd:serviceParam wikibase:
language "ru,en". }
      ?item wdt:P31 wd:Q5.     # P31 - instance of, Q5 - human
      ?item wdt:P106 wd:""" + occupation + """.
    } """

  pages=pagegenerators.WikidataSPARQLPageGenerator(query, site=site)
  for item in pages:
    print("Люди, чья профессия 
{occupation}: {title}" . format(occupation=occupation, 
title=item.title()))
\end{lstlisting} 
\marginnote[-2.0cm]{Обратите внимание, 
что до сих пор мы обращались к функции \lstinline|print()| 
для печати значений переменных, 
но в~листинге~\ref{lst:occupation-mistakeslist} 
для большей наглядности использована функция \lstinline|format()|.}

Список <<ошибочных>> профессий (объектов ВД), подлежащих удалению, сохранён в переменной \textit{delete\_occupation}. Мы обходим этот список с помощью цикла \textit{for}, чтобы с помощью SPARQL - запроса получить список людей для каждой <<подозрительной>> профессии. Этот~(листинг~\ref{lst:occupation-mistakeslist}) отличается от~(листинга~\ref{lst:occupation-list}) тем, что в ~листинге~\ref{lst:occupation-mistakeslist} есть переменная \textit{occupation}, которая берёт значение из массива \textit{delete\_occupation} и подставляет его в SPARQL - запрос.

Теперь, если мы уверены, что информация о профессии неверна, и мы хотим ее удалить, то достаточно указать свойство, которое нужно удалить.

Есть ещё одна трудность. Дело в том, что персона может и иметь несколько профессий, а удалить мы хотим только одну. Для удаления этой одной профессии нам нужно знать её значение. Вот, как можно это сделать, см.~(листинг~\ref{lst:delete-occupation}). 

\begin{lstlisting}[ language=Python,
                    caption={Корректное удаление профессии},
                    label=lst:delete-occupation
                  ]
import pywikibot
from pywikibot import pagegenerators
site=pywikibot.Site('wikidata', "wikidata")
delete_occupation={"Q7404672", "Q7319129", "Q7249866", "Q7244433", 
"Q7208069"}
for occupation in delete_occupation:
  query = """
    SELECT DISTINCT ?item WHERE {
      SERVICE wikibase:label {bd:serviceParam wikibase:language "ru,en"}
      ?item wdt:P31 wd:Q5.
      ?item wdt:P106 wd:""" + occupation + """.
    } """

  pages = pagegenerators.WikidataSPARQLPageGenerator(query, 
  site=site)
  for item in pages:
    print("Люди, чья профессия 
{occupation}: {title}" . format(occupation=occupation, 
title=item.title()))
    item.get()

    for valor in item.claims['P106']:
      if (str(valor.getTarget())=='[[wikidata:' + occupation + 
']]'):
        print("<<<<< Удалить {occupation} от {title} >>>>>" . 
format(occupation=occupation, title=item.title()))
        item.removeClaims(valor, summary=u'Удаление 
ошибочных значений в P106')
\end{lstlisting} 

В листинге~\ref{lst:delete-occupation} мы сохранили код листинга~\ref{lst:occupation-mistakeslist}, но добавили последние строки, в которых мы просматриваем значения, содержащиеся в свойстве \wdProperty{106}{}, с помощью свойства \textit{claims}. Если это совпадает с искомой профессией, мы удаляем его методом \textit{removeClaims}. При вызове этого метода нужно указать удаляемое значение и комментарий к редакторской правке (summary)\sidenote[]{Комментарии к правкам позволяют другим редактоарм быстро понять, какие изменения были внесены в страницу или объект Викиданных.}, сохраняемый в истории правок вики-страницы.

Обратите внимание, что такие изменения могут вызвать множество проблем с Викиданными, и администраторы могут заблокировать вашу учетную запись. Сначала внесите несколько изменений вначале и всегда проверяйте историю изменений, чтобы узнать, какая информация была изменена.

\subsection{Упражнения}

\begin{enumerate} 
\item Добавьте больше ошибочных и неправильных профессий в переменную \textit{delete\_occupation} и запустите бота для удаления этих свойств.
\end{enumerate}
