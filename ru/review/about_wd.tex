\chapter{Обзор Викиданных}
\label{ch:ReviewAboutWD}

\section{Викиданные}

Викиданные~--- это структурированная и совместно редактируемая база данных\footnote[][-15pt]{Викиданные (Wikidata)~--- это свободная, совместно наполняемая, многоязычная, вторичная база данных, в которой собрана структурированная информация для поддержки работы Википедии, Викисклада и других проектов Викимедиа.}. Проект был официально запущен 30 октября 2012 года, его разработка ведётся под руководством Wikimedia Deutschland\footnote[][5pt]{Немецкое отделение Фонда Викимедиа.}. Проект создавался за счёт пожертвований Allen Institute for Artificial Intelligence, Gordon and Betty Moore Foundation и Google. Викиданные~--- это бесплатная и свободная база знаний, которая может использоваться и редактироваться людьми и компьютерными программами\autocite{Vrandecic}.\begin{marginfigure}[0.0cm]
{
	\setlength{\fboxsep}{0pt}%
	\setlength{\fboxrule}{1pt}%
	\fcolorbox{gray}{white}{\includegraphics[width=0.41\linewidth]{./review/Wikidata-logo-en.png}}
}
\caption[Логотип Викиданных]{Логотип Викиданных. 
Wikimedia Commons / \href{https://commons.wikimedia.org/wiki/File:Wikidata-logo-en.svg}{Planemad} 
}
\label{fig:seyu}
\end{marginfigure}

Содержимое Викиданных распространяется по лицензии Creative Commons CC0, которая позволяет повторно использовать информацию самыми разными способами: пользователи могут копировать, изменять, распространять и обрабатывать эти данные в любых целях. Ещё одна особенность Викиданных~--- это многоязычность. Любой человек может редактировать Викиданные более чем на 350 языках.

Викиданные постоянно обновляются, добавляются новые объекты. 
На 2023 год насчитывается более 100 млн страниц и около 2 млрд правок 
(данные взяты с официальной страницы статистики Викиданных: 
\href{https://www.wikidata.org/wiki/Wikidata:Statistics}{https://www.wikidata.org/wiki/Wikidata:Statistics}). 
Уже в 2019 году на сайте Викиданных было совершено более 800 тыс. правок, что превзошло количество правок в Английской Википедии и сделало Викиданные наиболее редактируемым сайтом Викимедиа. %\footnote[][5px]{
Веб-сайт Викиданных, к которому мы будем регулярно обращаться, такой: \href{https://www.wikidata.org}{https://www.wikidata.org}. %}.




\section{Wikidata Query Service}
\label{sect:WDQS}

Любой объект Викиданных имеет свой уникальный идентификатор и свойства. 
Эта информация может быть обработана с помощью компьютера, 
и при этом она наглядно представлена и понятна пользователям без специальной предобработки. 
Сайт Викиданных содержит сервис Wikidata Query\footnote{%
%
Полное название инструмента Wikidata Query Service, кратко~--- WDQS. URL: 
\href{https://query.wikidata.org/}{https://query.wikidata.org/}.%
%
}, включающий набор инструментов для построения SPARQL-запросов 
и их визуализации в~виде таблиц, диаграмм, графов или географических карт.




\section{Об исследовании Викиданных}

В работе \textit{A large-scale collaborative ontological medical database}\footnote{%
\fullcite{Collaborative_ontological_database}.}\,описываются плюсы использования Викиданных для создания крупномасштабной 
совместно используемой медицинской базы данных. 
Основные требования к создаваемой базе данных таковы: 
это должна быть платформа с~обновлением в~реальном времени, 
с~лицензией, разрешающей дальнейшее использование полученной информации, 
с~возможностью редактирования на~любом языке и с~открытым доступом. 
Именно это и есть основные характеристики Викиданных. 
Во-первых, Викиданные~--- это открытая, редактируемая база знаний. 
Любой пользователь без навыков программирования может вносить изменения 
более чем на 350 языках. 
Во-вторых, информация постоянно обновляется, добавляются новые объекты. 
На~2023 год Викиданные насчитывают около 25~тыс. редакторов\footnote{Для сравнения: 
число активных редакторов в Русской Википедии 
составляет 11~тыс., в Английской~--- 130~тыс., на Викискладе~--- 40~тыс. 
Поясним термин \emph{активный редактор}~--- это такой пользователь сайта, 
который сделал хотя бы одну правку за~последние 30 дней.}. 
В-третьих, лицензия Creative Commons CC0 обеспечивает широкое использование полученной информации. 

Ввиду этих преимуществ у Викиданных сейчас нет конкурентов. 
Но принято указывать аналоги и~альтернативы. Укажем и мы. Есть несколько альтернативных баз знаний:
\begin{enumerate}
\item Cyc~--- проект компании Cycorp (Остин, США) по созданию онтологической базы знаний, 
    позволяющий решать задачи из области искусственного интеллекта. База Cyc имеет исследовательскую лицензию ResearchCyc. У этой базы есть некоторые недостатки: сложность системы (сложность добавления данных
вручную), недостаток документации, неполнота системы.
\item Evi (ранее True Knowledge)~--- 
    технологическая компания в Кембридже (Англия), 
        которая специализируется на базе знаний и программном обеспечении 
        \textit{семантического поиска}\footnote[][1\baselineskip]{%
            Семантический поиск\index{Информатика!Семантический поиск}~--- 
            это способ и технология поиска информации с использованием контекстного значения запрашиваемых фраз 
            вместо словарных значений отдельных слов или выражений, входящих в поисковый запрос.}%
.\,     Добавление информации в базу знаний осуществляется двумя способами: импорт из <<заслуживающих доверия>> внешних баз данных (например, Википедия) и добавление данных самими пользователями. Как и в~Википедии, пользователь может изменять
данные, <<соглашаться>> или <<не соглашаться>> с информацией, представленной системой Evi. Система может отклонить любые факты, которые семантически несовместимы с другими утверждениями, в отличие от Викиданных, где могут
храниться противоречивые данные.
\item DBPedia\index{Информатика!База знаний!DBPedia}~--- краудсорсинговый проект, 
    направленный на извлечение структурированной информации из данных, 
        созданных в рамках проекта Википедия, 
        и публикации её в виде доступных под свободной лицензией наборов данных. 
        Проект был отмечен как один из наиболее известных примеров реализации 
        концепции связанных данных\footnote[][]{%
%
О связанных данных и графах знаний см. в~главе <<\nameref{ch:BucketsAndBalls}>> на~с.~\pageref{ch:BucketsAndBalls}.%
%
}.      Он был начат группой добровольцев из Свободного университета Берлина и Лейпцигского университета 
        в сотрудничестве с фирмой OpenLink Software, первый набор данных опубликован в 2007 году. 
        С 2012 года активным участником проекта является Университет Мангейма.%
\end{enumerate}

В Викиданных информация представлена в виде объектов (или элементов), 
связанных между собой с помощью свойств\footnote{%
%
Например, существуют такие свойства: \wdProperty{31}{экземпляр}, 
\wdProperty{279}{подкласс}, \mbox{\wdProperty{361}{часть}}, \wdProperty{527}{имеет часть}.%
%
}. Мощь базы Викиданных в её большом объёме и в удивительно быстром росте и самоорганизации, 
в том, что к этой <<живой>> базе знаний можно обращаться с помощью  SPARQL-запросов, 
представлять результаты их выполнения в виде таблиц, графов, диаграмм или сохранять в~нужном формате (CSV, JSON, SVG).

Викиданные могут взять на себя роль централизованного хранилища данных. 
В~статье\autocite{Falcon} приводится пример использования Викиданных 
в~качестве централизованной и общедоступной базы знаний для~системы FALCON~2.0. 
Эта система идентифицирует сущности в коротком тексте или~вопросе, 
а~затем связывает их ссылками с~соответствующими объектами Викиданных.



\section{Неоднозначность объекта Викиданных}

Любой объект Викиданных имеет свойства. Одно из них~--- это <<экземпляр класса>> 
(\wdProperty{31}{instance of}). Оно определяет класс, к которому принадлежит объект. 
Мы обнаружили, что один объект Викиданных может соответствовать нескольким классам.
Некоторые объекты являются экземплярами совершенно разных классов. 
Например, \wdqName{Королевская шведская академия наук}{191583} является экземпляром 
сразу трёх классов: академии наук, сооружения (здания) и королевской академии Швеции. 
Такое определение классов верно, поскольку этот объект можно рассматривать 
и как организацию, целью которой является развитие науки, и как архитектурное сооружение. 
Мы рассмотрели пример многозначности в Викиданных, которая разрешается с помощью свойства instance of.

В Википедии принято иначе оформлять многозначность. 
Если слово имеет несколько значений, то есть если слово многозначное, 
то в Википедии есть несколько статей с одинаковым названием об~этих значениях, 
но в конце названия в скобках пишется уточнение, 
например: <<Коса (причёска)>> и <<Коса (рельеф)>>. 
Такое чёткое явное разнесение многозначности слов позволило использовать тексты Википедии 
в задаче разрешения лексической многозначности или WSD-задаче\autocite{Fogarolli}.




\section{Качество и техническая платформа Викиданных}

Викиданные существуют с 2012 года. 
На 2023 год в Викиданных зарегистрировано 6 млн пользователей, которые сделали около 2 млрд правок.

В диссертации Alessandro Piscopo\autocite{Piscopo} идёт речь 
о социально-технических процессах и качестве данных проекта <<Викиданные>>, 
о том, что пользователи Викиданных имеют возможность добавлять отдельные фрагменты информации, 
выполнять редактирование через различные интерфейсы 
и работать с такими платформами, как Википедия, 
но при этом они в полной мере несут ответственность за поддержание схемы 
графа знаний\autocite{KnowledgeGraphs} в~рабочем состоянии. 
Однако эту работу должна выполнять команда обученных специалистов 
в соответствии с чётко продуманными методами. 
Эти действия осуществляются с~помощью инструментов, которые составляют техническую основу системы.


Особым инструментом как в Викиданных, так и в Википедии являются \emph{боты}\footnote{%
%
Более подробно о~ботах см.~раздел <<\nameref{ch:bots}>>, с.~\pageref{ch:bots}.%
%
}. Это части программного обеспечения, которые автоматически могут выполнять 
различные действия на платформе с большой скоростью (более тысячи правок в минуту). 
Их основная задача~--- редактирование существующих данных, добавление и импорт новых даных из других ресурсов. 
Боты~--- один из ключевых технических компонентов Викиданных. 

В статье <<Сетевая структура научных революций>>\autocite{NetworkStructureRevolutions} 
на примере Википедии рассматривается процесс формирования знаний 
в виде постоянно растущих сетей из статей и связывающих их гиперссылок. 
Эта концепция реализуется за счёт заполнения пробелов в~знаниях. 
Цель этой работы сформулирована в одном предложении: 
<<Авторы проверяют теории научного прогресса на~растущих концептуальных сетях 
и раскрывают управляемые данными условия, лежащие в основе прорывов>>\footnote{%
%
Оригинальный текст (англ.):  The authors test theories of scientific progress 
    on growing concept networks and reveal data-driven conditions 
    underlying breakthroughs.}. 
В процессе исследования научных революций было проведено ранжирование всех статей Википедии 
в виде сети по определённым критериям. 
Каждый узел сети соответствует определённой статье, имя узла --- это заголовок статьи, 
год рождения узла~--- это первый год, указанный во введении или в разделе истории как год, когда концепция была задумана. 
Затем на~основе текущего состояния сетей были определены некоторые закономерности 
в эволюции этих структур на~протяжении времени и периоды, 
когда сеть наиболее быстро менялась. 
Полученные результаты показали, что человеческие знания растут и, как следствие, 
происходит постепенное изменение сетевой структуры (заполняются некоторые пробелы в~знаниях).\, 
Авторы исследования считают, что знания, 
обнаруженные при заполнении пробелов, будут иметь важное значения для~научных инноваций. 
Это исследование связано с качеством Викиданных, 
потому что информация для пополнения Викиданных чаще всего берётся из Википедии. 
Если будут заполнены пробелы в Википедии, 
то новые данные обязательно будут добавлены в Викиданные 
и база знаний станет более полной и подробной.
