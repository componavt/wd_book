\chapter{Обзор Викиданных}
\label{ch:ReviewAboutWD}

\section{Викиданные}
Викиданные — это структурированная и совместно редактируемая база данных, созданная Фондом Викимедиа\footnotemark \footnotetext{Викиданные (Wikidata) — это свободная, совместно наполняемая, многоязычная, вторичная база данных, в которой собрана структурированная информация для поддержки работы Википедии, Викисклада и других вики-проектов Фонда Викимедиа.}. Проект был официально запущен 30 октября 2012 года, его разработка ведется под руководством Wikimedia Deutschlaа\footnotemark \footnotetext{Немецкое отделение Фонда Викимедиа.}. Проект создавался за счёт пожертвований Allen Institute for Artificial Intelligence, Gordon and Betty Moore Foundation и Google. Викиданые — это бесплатная и свободная база знаний, которая может использоваться и редактироваться людьми и компьютерными программами\cite{Vrandecic}.\begin{marginfigure}[0.0cm]
{
	\setlength{\fboxsep}{0pt}%
	\setlength{\fboxrule}{1pt}%
	\fcolorbox{gray}{white}{\includegraphics{./review/Wikidata-logo-en.png}}
}
\caption
{Логотип Викиданных. \newline
2012 / Planemad / Общественное достояние
}
\label{fig:seyu}
\end{marginfigure}

Любой объект Викиданных имеет свой уникальный идентификатор и свойства. Эта информация может быть обработана с помощью компьютера, и при этом она понятна пользователям. Сайт Викиданных содержит сервис ‘Wikidata Query’, включающий набор инструментов для построения SPARQL-запросов и их визуализации в виде таблиц, диаграмм, графов или географических карт.

Содержимое Викиданных распространяется по лицензии Creative Commons CC0, которая позволяет повторно использовать информацию самыми разными способами: пользователи могут копировать, изменять, распространять и обрабатывать эти данные в любых целях. Ещё одна особенность Викиданных --- это многоязычность. Любой человек может редактировать Викиданные более чем на 350 языках.

Викиданные постоянно обновляются, добавляются новые объекты. На 2021 год насчитывается более 95 миллионов страниц и более полутора миллиардов правок\footnotemark. \footnotetext{Данные взяты с официальной страницы статистики Викиданных: \href{https://www.wikidata.org/wiki/Wikidata:Statistics}{https://www.wikidata.org/wiki/Wikidata:Statistics}} В 2019 года в Викиданных было совершено более 800 тысяч правок, что превзошло количество правок в Английской Википедии и сделало Викиданные наиболее редактируемым сайтом Викимедиа\footnotemark . \footnotetext{Веб-сайт Викиданных: \href{https://www.wikidata.org/}{https://www.wikidata.org/}}

\section{Об исследовании Викиданных}

В работе ``A large-scale collaborative ontological medical database''\cite{Collaborative_ontological_database} 
описываются плюсы использования Викиданных для создания крупномасштабной 
совместной медицинской базы данных. 
Основные требования к создаваемой базе данных таковы: 
это должна быть платформа с обновлением в реальном времени, 
с лицензией, разрешающей дальнейшее использование полученной информации, 
с возможностью редактирования на любом языке и с открытым доступом. 
Именно это и есть основные характеристики Викиданных. 
Во-первых, Викиданные — это открытая, редактируемая база знаний. 
Любой пользователь без навыков программирования может вносить изменения 
более чем на 350 языках и диалектах. 
Во-вторых, информация постоянно обновляется, добавляются новые объекты. 
На 2021 год Викиданные насчитывают более 18000 редакторов\footnotemark. 
%
\footnotetext{Число активных редакторов в Русской Википедии 
составляет примерно четыре тысячи пользователей 
(данные взяты с официальной страницы статистики Русской Википедии: 
%\href{https://stats.wikimedia.org/#/ru.wikipedia.org}{https://stats.wikimedia.org/#/ru.wikipedia.org}), 
на Викискладе~--- более 15 тысяч пользователей 
(данные взяты с официальной страницы статистики Викисклада: 
%\href{https://stats.wikimedia.org/#/commons.wikimedia.org}{https://stats.wikimedia.org/#/commons.wikimedia.org}).
}. 
В-третьих, лицензия Creative Commons CC0 обеспечивает широкое использование полученной информации. 

На самом деле у Викиданных сейчас нет конкурентов. Но принято указывать аналоги и альтернативы. Укажем и мы. Есть несколько альтернативных баз знаний:
\begin{enumerate}
\item Cyc — это проект компании Cycorp (Остин, США) по созданию онтологической базы знаний, позволяющий решать задачи из области искусственного интеллекта\cite{Cyc}. Сейчас Cyc имеет исследовательскую лицензию ResearchCyc. У данной базы знаний есть некоторые недостатки: сложность системы (сложность добавления данных
вручную), недостаток документации для изучения системы, неполнота системы.
\item Evi (ранее True Knowledge\cite{True_Knowledge}) --- это технологическая компания в Кембридже (Англия), которая специализируется на базе знаний и программном обеспечении \textit{семантического поиска}\footnotemark. \footnotetext{Семантический поиск — способ и технология поиска информации, основанная на использовании контекстного (смыслового) значения запрашиваемых фраз, вместо словарных значений отдельных слов или выражений при поисковом запросе.} Добавление информации в базу знаний осуществляется двумя способами: импорт из «заслуживающих доверия» внешних баз данных (например: Википедия) и из представления пользователей в соответствии с единообразным форматом и подробным процессом ввода. Как и в Википедии, пользователь может изменять
данные, «соглашаться» или «не соглашаться» с информацией, представленной True Knowledge. Система может отклонить любые факты, которые семантически несовместимы с другими утверждениями, в отличие от Викиданных, где могут
храниться противоречивые данные.
\end{enumerate}
Мы полагаем, что Викиданные являются наилучшим вариантом для представления информации, так как можно связывать объекты через их свойства\footnotemark \footnotetext{ Например: экземпляр P31, подкласс 279, часть P361, имеет часть P527.}, создавать SPARQL-запросы, представлять результаты их выполнения в виде таблиц, графов, диаграмм или сохранять в нужном формате (CSV, JSON, SVG).

Викиданные могут взять на себя роль централизованного хранилища данных. В статье ``Falcon 2.0: An Entity and Relation Linking Tool over Wikidata''\cite{Falcon_2.0} приводится пример использования Викиданных в качестве централизованной и общедоступной базы знаний для системы FALCON 2.0. Эта система идентифицирует сущности в коротком тексте или вопросе, а затем связывает их ссылками с соответствующими объектами Викиданных.
\section{Неоднозначность объекта Викиданных}
Любой объект Викиданных имеет свойства. Одно из них --- ``P31'' (instance of --- экземпляр класса). Оно определяет класс, к которому принадлежит объект. В правилах Викиданных и в некоторых статьях\cite{BabelNet}, найденных с помощью поисковой системы научных статей\footnotemark \footnotetext{Академия Google: \href{https://scholar.google.com}{https://scholar.google.com}}, написано, что объекту соответствует один класс.

Но в ходе исследований было обнаружено, что это не всегда так. Оказалось, что некоторые объекты являются экземплярами совершенно разных классов. Например, Королевская шведская академия наук (Q191583) является экземплярами сразу трех классов: академии наук, сооружения и королевской академии Швеции. Такое определение классов имеет место быть, поскольку этот объект можно рассматривать и как организацию, целью которой является развитие науки, и как архитектурное сооружение. 

Этот пример относится к задаче разрешения лексической многозначности или WSD-задаче. Ею занимался ученый Angela Fogarolli из итальянского университета. Результаты исследований были представлены в статье ``Word Sense Disambiguation based on Wikipedia Link Structure''\cite{Fogarolli}. Автор выделяет объекты неоднозначностей, которые соответствует нескольким классам в зависимости от контекста и допускает наличие нескольких классов в свойстве ``instance of''.
\section{Качество Викиданных}
Викиданные существуют с 2012 года. На 2021 год редакторами проекта являются более 200 тысяч пользователей, которые сделали более 50 миллионов правок.

В диссертации Alessandro Piscopo\cite{Piscopo} рассказывается о социально-технических процессах и качестве данных проекта Викиданные. В диссертации рассказывается о том, что пользователи Викиданных имеют возможность добавлять отдельные фрагменты информации, выполнять редактирование через различные интерфейсы и работать с такими платформами как Википедия, но при этом они в полной мере несут ответственность за поддержание схемы графа знаний в рабочем состоянии. Однако эту работу должна выполнять команда обученных специалистов в соответствии с чётко продуманными методами. Эти действия осуществляются с помощью инструментов, которые составляют техническую основу системы.

Особым инструментом как в Викиданных, так и в Википедии являются \textit{боты}\footnotemark. \footnotetext{Более подробно о ботах см. в главе <<Боты в Викиданных>> на стр. \pageref{ch:bots}.} Это части программного обеспечения, которые автоматически могут выполнять различные действия на платформе с большой скоростью (более тысячи правок в минуту). Их основная задача --- это редактирование существующих данных, добавление и импорт новых из других ресурсов. Боты создают отчеты, с помощью которых пользователь может исправлять некоторые неточности. 

Таким образом, боты являются одним из ключевых технических компонентов Викиданных. Пользователи добавляют и модифицируют данные, а также общаются между собой с помощью веб-интерфейса Викиданных. Также доступны плагины, которые предупреждают редакторов, когда они собираются выполнить ревизию, которая может привести к любым ошибкам в данных.

В статье <<Сетевая структура научных революций>>, в которой на примере Википедии рассматривается процесс формирования знаний в виде постоянно растущих сетей из статей и связывающих их гиперссылок. Эта концепция реализуется за счет заполнения пробелов в знаниях. Авторы сформулировали цель своей работы в одном предложении: <<Авторы проверяют теории научного прогресса на растущих концептуальных сетях и раскрывают управляемые данными условия, лежащие в основе прорывов>>\cite{Network_structure_revolutions}\footnotemark. \footnotetext{Оригинальный текст (англ.):  ‘The authors test theories of scientific progress on growing concept networks and reveal data-driven conditions underlying breakthroughs’.}

В процессе исследований (Zhou, Ju и Blevins, 2020) было проведено ранжирование всех статей Википедии на сети по определённым критериям. Каждый узел сети соответствует определённой статье, имя узла --- это заголовок статьи, год рождения узла --- это первый год, указанный во введении или в разделе истории как год, когда концепция была задумана. Затем на основе текущего состояния сетей были определены некоторые закономерности в эволюции этих структур на протяжении времени и периоды, когда сеть наиболее быстро менялась.

Полученные результаты показали, что человеческие знания растут и как следствие происходит постепенное изменение сетевой структуры (заполняются некоторые пробелы в знаниях). Авторы статьи \cite{Network_structure_revolutions} считают, что знания, обнаруженные при заполнении пробелов, будут иметь важное значения для научных инноваций. 

Эта статья имеет непосредственное отношение к качеству Викиданных, потому что информация для Викиданные чаще всего берется из Википедии. Если будут заполнены пробелы в Википедии, то новые данные обязательно будут добавлены в Викиданные, и база знаний станет более подробной.
