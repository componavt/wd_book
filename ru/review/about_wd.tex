\chapter{Обзор Викиданных}
\label{ch:ReviewAboutWD}

\section{Викиданные}
Викиданные — это структурированная и совместно редактируемая база данных, созданная Фондом Викимедиа\footnotemark \footnotetext{Викиданные (Wikidata) — это свободная, совместно наполняемая, многоязычная, вторичная база данных, в которой собрана структурированная информация для поддержки работы Википедии, Викисклада и других вики-проектов Фонда Викимедиа.}. \begin{marginfigure}[0.0cm]
{
	\setlength{\fboxsep}{0pt}%
	\setlength{\fboxrule}{1pt}%
	\fcolorbox{gray}{white}{\includegraphics{./review/Wikidata-logo-en.png}}
}
\caption
{Логотип Викиданных. \newline
2012 / Planemad / Общественное достояние
}
\label{fig:seyu}
\end{marginfigure}Проект был официально запущен 30 октября 2012 года, его разработка ведется под руководством Wikimedia Deutschland\cite{Wikipedia_review}. Проект создавался за счёт пожертвований Allen Institute for Artificial Intelligence, Gordon and Betty Moore Foundation и Google. В данный момент Викиданые — это бесплатная и свободная база знаний, которая может использоваться и редактироваться людьми и машинами\cite{Vrandecic}.

Любой объект Викиданных имеет свой уникальный идентификатор и свойства. Эта информация может быть обработана с помощью компьютера, и при этом она понятна пользователям. Сайт Викиданных содержит сервис ‘Wikidata Query’, включающий набор инструментов для построения SPARQL-запросов и их визуализации в виде таблиц, диаграмм, графов или географических карт.

Содержимое Викиданных распространяется по лицензии Creative Commons CC0, которая позволяет повторно использовать информацию самыми разными способами: пользователи могут копировать, изменять, распространять и обрабатывать эти данные в любых целях. Еще одна особенность Викиданных --- это многоязычность. Любой человек может редактировать Викиданные более чем на 350 языках.

Викиданные постоянно обновляются, добавляются новые объекты. На 2021 год насчитывается более 95 миллионов страниц и более полутора миллиардов правок\footnotemark. \footnotetext{Данные взяты с официальной страницы статистики Викидынных: \href{https://www.wikidata.org/wiki/Wikidata:Statistics}{https://www.wikidata.org/wiki/Wikidata:Statistics}} В 2019 года в Викиданных было совершено более 800 тысяч правок, что превзошло количество правок в английской Википедии и сделало Викиданные наиболее редактируемым сайтом Викимедиа\footnotemark . \footnotetext{Ссылка на веб-сайт Викидынных: \href{https://www.wikidata.org/}{https://www.wikidata.org/}}
\section{Об исследовании Викиданных}
В работе ‘A large-scale collaborative ontological medical database’\cite{Collaborative_ontological_database} описываются плюсы использования Викиданных для создания крупномасштабной совместной медицинской
базы данных. Основные требования к создаваемой базе данных — это платформа с обновлением в реальном времени, подходящая лицензия для последующего использования полученной информации, свободное редактирование на любом языке. Именно это и есть основные характеристики Викиданных. Во-первых, Викиданные — это открытая, редактируемая база знаний. Любой пользователь без навыков программирования может вносить изменения более чем на 350 языках и диалектах. Во-вторых, информация постоянно обновляется, добавляются новые объекты. В настоящее время Викиданные насчитывают более 18000 редакторов. В-третьих, лицензия Creative Commons CC0 обеспечивает широкое использование полученной информации.

Есть несколько альтернативных вариантов баз знаний:
\begin{enumerate}
\item Cyc — проект компании Cycorp (Остин, США) по созданию онтологической базы знаний, позволяющий решать задачи из области искусственного интеллекта\cite{Cyc}. Сейчас Cyc имеет исследовательскую лицензию ResearchCyc. У данной базы знаний есть некоторые недостатки: сложность системы (сложность добавления данных
вручную), недостаток документации для изучения системы, неполнота системы.
\item Evi (ранее True Knowledge\cite{True_Knowledge}) – технологическая компания в Кембридже (Англия), которая специализируется на базе знаний и программном обеспечении семантического поиска. Добавление информации в базу знаний осуществляется двумя способами: импорт из «заслуживающих доверия» внешних баз данных (например: Википедия) и из представления пользователей в соответствии с единообразным форматом и подробным процессом ввода. Как и в Википедии, пользователь может изменять
данные, «соглашаться» или «не соглашаться» с информацией, представленной True Knowledge. Система может отклонить любые факты, которые семантически несовместимы с другими утвержденными знаниями, в отличие от Викиданных, где могут
храниться противоречивые данные.
\end{enumerate}
По мнению авторов статьи, Викиданные являются лучшим вариантом для обработки информации, т.к. можно связывать объекты через их свойства (экземпляр P31, подкласс 279, часть P361, имеет часть P527), создавать SPARQL-запросы, визуализировать их результаты в виде таблиц, графов, диаграмм или сохранять в нужном формате (CSV, JSON, SVG).

Таким образом, авторы призывают обратить внимание на Викиданные, которые могут взять на себя роль централизованного хранилища данных. В статье ‘Falcon 2.0: An Entity
and Relation Linking Tool over Wikidata’\cite{Falcon_2.0} приводится пример использования Викиданных в качестве централизованной и общедоступной базы знаний для системы FALCON 2.0. Это инструмент, связывающий сущность и отношения через Викиданные. Эта система идентифицирует сущности в коротком тексте или вопросе, а затем связывает их с соответствующими URL в графе знаний Викиданных.
\section{Неоднозначность объекта Викиданных}
Как говорилось ранее любой объект Викиданных имеет свойства. Одно из них --- ‘P31’ (instance of --- экземпляр класса). Оно определяет класс, к которому принадлежит объект. В правилах Викиданных и в некоторых статьях\cite{BabelNet}, которые были найдены с помощью сайта \href{scholar.google.com}{scholar.google.com}\cite{GoogleAcademy}, написано, что объекту соответствует один класс.

Но в ходе исследований было обнаружено, что это не всегда так. Оказалось, что некоторые объекты являются экземплярами совершенно разных классов. Например, Королевская шведская академия наук (Q191583) является экземплярами сразу трех классов: академии наук, сооружении и королевской академии Швеции. На мой взгляд, такое определение классов в этом случае верно. Поскольку данный объект можно рассматривать как организацию, целью которой является развитие науки, и как архитектурное сооружение. 

Разрешением задачи о лексической многозначности занимался ученый Angela Fogarolli из итальянского университета. Результаты исследований были представлены в виде статьи ‘Word Sense Disambiguation based on Wikipedia Link Structure’\cite{Fogarolli}. Автор выделяет объекты неоднозначностей, которые соответствует нескольким классам в зависимости от контекста и допускает наличие нескольких классов в свойстве ‘instance of’.
\section{Качество Викиданных}
В докторской диссертации Alessandro Piscopo\cite{Piscopo} рассказывается о социально-технических процессах и качестве данных проекта Викиданные, в котором с даты запуска (2012 год) произошли большие изменения. На данный момент редакторами проекта являются более 200 тысяч пользователей, которые сделали более 50 миллионов правок.

В диссертации рассказывается о том, что пользователи Викиданных имеют возможность добавлять отдельные фрагменты информация, выполнять редактирование через различные интерфейсы и работать с такими платформами как Википедия, но при этом они несут ответственность за поддержание схемы графа знаний. Однако эту работу должна выполнять команда обученных специалистов в соответствии с четко продуманными методами. Эти действия осуществляются с помощью инструментов, которые составляют техническую основу системы.

Особым инструментом как в Викиданных, так и в Википедии являются боты. Это части программного обеспечения, которые автоматически могут выполнять различные действия на платформе с большой скоростью (более тысячи правок в минуту). Их основная задача ---редактирование существующих данных, добавление и импорт новых из других ресурсов. Боты создают отчеты, с помощью которых пользователь может исправлять некоторые неточности. 

Таким образом, боты являются одним из ключевых технических компонентов Викиданных. Пользователи добавляют и модифицируют данные, а также общаются между собой с помощью веб-интерфейса Викиданных. Также доступны плагины, которые предупреждают редакторов, когда они собираются выполнить ревизию, которая может привести к любым ошибкам в данных.

Также стоит обратить внимание на статью "Сетевая структура научных революций", в которой на примере Википедии рассматривается процесс формирования знаний в виде постоянно растущих сетей из статей и их взаимосвязанных гиперссылок. Эта концепция реализуется за счет заполнения пробелов в знаниях. Авторы сформулировали цель своей работы в одном предложении: ‘The authors test theories of scientific progress on growing concept networks and reveal data-driven conditions underlying breakthroughs’, т.е. авторы проверяют теории научного прогресса на растущих концептуальных сетях и раскрывают управляемые данными условия, лежащие в основе прорывов\cite{Network_structure_revolutions}. 

В процессе исследований было проведено ранжирование всех статей Википедии на сети по определенным критериям. Каждый узел сети соответствует определенной статье, имя узла --- это заголовок статьи, год рождения узла --- это первый год, указанный во введении или в разделе истории как год, когда концепция была задумана. Затем на основе текущего состояния сетей были определены некоторые закономерности в эволюции этих структур на протяжении времени и периоды, когда сеть наиболее быстро менялась.

Полученные результаты показали, что человеческие знания растут и как следствие происходит постепенное изменение сетевой структуры (заполняются некоторые пробелы в знаниях). Авторы статьи считают, что знания, обнаруженные при заполнении пробелов, будут иметь важное значения для научных инноваций. 

Данная статья имеет непосредственное отношение к качеству Викиданных, потому что информация для Викиданные чаще всего берется из Википедии. Если будут заполнены пробелы в Википедии, то новые данных обязательно буду добавлены в Викиданные. Следовательно, база знаний будет более подробной.
