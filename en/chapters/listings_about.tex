
\section{About listings}
\labsec{section:listing-about}

This book contains listings in SPARQL scripts.
These scripts extract and process data from Wikidata.%\todo{To add screenshot to this (?) sidenote.}

An example SPARQL script is shown in (Listing~\ref{lst:cities}). 
This script gets a list of cities from Wikidata, 
namely: instances of the object \wdqName{city}{515}.
The colored text in the listing title (\href{https://w.wiki/ktn}{List of cities}) 
is the link to the corresponding SPARQL script online, 
which is available online: 
% todo \href{https://w.wiki/ktn}{w.wiki/ktn}\sidenote[?]{How did we get such a short link to the script?}


%точнее экземляров объекта\footnote{\label{question:instance-in-OOP-vs-Wikidata}Что такое экземпляр объекта? 
%    Какая разница между экземпляром объекта 
%    в объектно-ориентированном программировании и в Викиданных?
%    См. ответ~\ref{answer:instance-in-OOP-vs-Wikidata} на с.~\pageref{answer:instance-in-OOP-vs-Wikidata}.
%    }

% newline in the caption due to: \\\hspace{\textwidth}
% see https://tex.stackexchange.com/a/101624/99685
\begin{lstlisting}[ language=SPARQL, 
                    caption={List of cities\\\hspace{\textwidth}
                        The result contains \num{20 800} cities in 2017, 
                        \num{9 260} cities in 2020.\\\hspace{\textwidth}
                        SPARQL query: \href{https://w.wiki/ktn}{w.wiki/ktn}
                        },
                    label=lst:cities,
                    texcl 
                    ]
SELECT ?city ?cityLabel WHERE { 
  ?city wdt:P31 wd:Q515.       # instance of city 
  SERVICE wikibase:label { bd:serviceParam wikibase:language "en" }
}
\end{lstlisting}%
%\footnotetext{The result contains \num{20 800} cities in 2017 year, 
%    \num{9 260} cities in 2020.}



