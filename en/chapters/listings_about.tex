
\section{About listings}
\labsec{section:listing-about}

This book contains \Wikiref{SPARQL} listings.
These scripts extract and process data from Wikidata.%\todo{To add screenshot to this (?) sidenote.}

An example SPARQL script is shown in Listing~\ref{lst:cities}. 
This script gets a list of cities from Wikidata, 
namely: instances of the object \wdqName{city}{515}.
The colored text in the end of the listing title 
(\href{https://w.wiki/ktn}{w.wiki/ktn})\marginnote[]{%
        \label{question:short-link-to-SPARQL}How 
        did we get such a short link (for example, \href{https://w.wiki/ktn}{w.wiki/ktn})
        to the script?
        See the answer in Exercise~\ref{answer:short-link-to-SPARQL}, page~\pageref{answer:short-link-to-SPARQL}.
} 
is the link to the corresponding SPARQL script,  
which is available online in Wikidata Query Service. 
Click the ``Play'' button to execute a script in this Wikidata Query Service. 

% newline in the caption due to: \\\hspace{\textwidth}
% see https://tex.stackexchange.com/a/101624/99685
\begin{lstlisting}[ language=SPARQL, 
                    caption={List of cities\\\hspace{\textwidth}
                        The result contains \num{20 800} cities in 2017, 
                        \num{9 260} cities in 2020.\\\hspace{\textwidth}
                        SPARQL query: \href{https://w.wiki/ktn}{w.wiki/ktn}
                        },
                    label=lst:cities,
                    texcl 
                    ]
SELECT ?city ?cityLabel WHERE { 
  ?city wdt:P31 wd:Q515.       # instance of city 
  SERVICE wikibase:label {bd:serviceParam wikibase:language "en"}
}
\end{lstlisting}%
%\footnotetext{The result contains \num{20 800} cities in 2017 year, 
%    \num{9 260} cities in 2020.}



