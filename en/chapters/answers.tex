\setchapterpreamble[u]{\margintoc}
\chapter{Answers}
\labch{ch-answers}


Tasks and questions are scattered throughout the book, 
and the answers are collected in this chapter.


\section{Introduction}
\labsec{answer-intro}


\begin{exercise}%
    \label{answer:short-link-to-SPARQL}
How to create a short link to a SPARQL script?
\end{exercise}

\begin{marginfigure}[0cm]
    {%
        \setlength{\fboxsep}{0pt}
        \setlength{\fboxrule}{1pt}
        \fcolorbox{gray}{gray}{\includegraphics[width=\linewidth]{chapter/intro/WD_Query_Service_Short_URL_2020.png}}
    }
	\caption{The chain symbol button creates a short link to the SPARQL script, Wikidata Query Service, 2020.}
	\labfig{fig:WDQS-Short-URL-creation}
\end{marginfigure}

The Wikidata Query Service is shown in~\reffig{fig:WDQS-Short-URL-creation}. 
The bottom button with a chain symbol allows you to create a short link to the SPARQL script. 

See question on page~\pageref{question:short-link-to-SPARQL}.



\section{Aircraft from small to large}
\labsec{answer-aircraft}

\begin{exercise}%
    \label{answer:formulate-your-short-label-for-question-here-please}
Question one?
\end{exercise}

Answer text.



\section{From towns to cities with millions of inhabitants}
\labsec{answer-towns}

\begin{exercise}%
    \label{answer:cities_geographic_objects}
Which of these cities were named after toponyms?
\begin{itemize}
\item \href{https://w.wiki/pzi}{Tolyatti}
\item \href{https://w.wiki/pzj}{Tula}
\item \href{https://w.wiki/pzk}{Chernyakhovsk}
\item \href{https://w.wiki/pzm}{Kurilsk}
\item \href{https://w.wiki/pzn}{Vologda}
\item \href{https://w.wiki/pzo}{Obninsk}
\end{itemize}
\end{exercise}

Tula, Kurilsk and Vologda were named after the following toponyms: Tulitsa river, \href{https://w.wiki/qqJ}{Kuril Islands}, \href{https://w.wiki/qqK}{Vologda river}. The answer to the question can also be obtained by running the following SPARQL query (Listing \ref{lst:cities_geographic_objects}). The \href{https://www.wikidata.org/wiki/Property:P138}{named after} property value shows which Wikidata object the city was named after.
    
\begin{lstlisting}[ language=SPARQL, 
                    caption={Cities named after toponyms.\\\hspace{\textwidth}
                        SPARQL query: \href{https://w.wiki/otn}{w.wiki/otn}
                        },
                    label=lst:cities_geographic_objects,
                    texcl 
                    ]
SELECT ?city ?cityLabel ?namedAfterLabel ?whatIsItLabel WHERE {
	?city wdt:P31/wdt:P279* wd:Q7930989. # "city/town" subclasses
	?city wdt:P138 ?namedAfter. # with property "named after"
	?namedAfter wdt:P31 ?whatIsIt. # which is instance of
	FILTER(?city = wd:Q1341 || ?city = wd:Q2770 || ?city = wd:Q5655 
		|| ?city = wd:Q156046 || ?city = wd:Q1957 || ?city = wd:Q175651)
	SERVICE wikibase:label {bd:serviceParam wikibase:language "en"}
}
\end{lstlisting}%

See question on page~\pageref{question:cities_geographic_objects}.

\marginnote[-2.0cm]{
If it doesn't matter what certain type of city the Wikidata object belongs to, you can use a construction with subclasses, specifying the only class relative to which the search will be performed. This construction is discussed in more detail in the section  ``Wikidata completeness and disadvantages'' on page \pageref{section:countries_Wikidata_completeness_and_disadvantages}.
}

\begin{exercise}%
    \label{answer:cities_over_400_age}
Which of these cities were founded more than 400 years ago: \href{https://w.wiki/pzt}{Moscow}, \href{https://w.wiki/pzu}{Sarov}, \href{https://w.wiki/pzx}{Kazan}, \href{https://w.wiki/pzy}{Astrakhan}, \href{https://w.wiki/pzz}{Samara}, \href{https://w.wiki/pz$}{Voronezh}?
\end{exercise}

Moscow (1147 year), Voronezh (1586), Samara (1586), Kazan (1005) and Astrakhan (1558) were founded more than 400 years ago. The answer to the question can also be obtained by running the following SPARQL query (Listing \ref{lst:cities_over_400_age}). The \href{https://www.wikidata.org/wiki/Property:P571}{inception} property value contains the date the city was founded.

\begin{lstlisting}[ language=SPARQL, 
                    caption={Cities founded more than 400 years ago.\\\hspace{\textwidth}
                        SPARQL query: \href{https://w.wiki/oto}{w.wiki/oto}
                        },
                    label=lst:cities_over_400_age,
                    texcl 
                    ]
SELECT ?city ?cityLabel ?inceptionDate WHERE {
	?city wdt:P31/wdt:P279* wd:Q7930989. # "city/town" subclasses
	?city wdt:P17 wd:Q159. # belonging to Russia
	?city wdt:P571 ?inceptionDate. # with property "inception"
	FILTER(BOUND(?inceptionDate) && 
			DATATYPE(?inceptionDate) = xsd:dateTime).
	BIND(NOW() - ?inceptionDate AS ?distance).
	FILTER(0 <= ?distance && ?distance > 146000). # = 400 * 365
	FILTER(?city = wd:Q649 || ?city = wd:Q193522 || ?city = wd:Q900
		|| ?city = wd:Q3927 || ?city = wd:Q894 || ?city = wd:Q3426)
	SERVICE wikibase:label {bd:serviceParam wikibase:language "en"}
}
GROUP BY ?city ?cityLabel ?inceptionDate
\end{lstlisting}%

See question on page~\pageref{question:cities_over_400_age}.

\begin{exercise}%
    \label{answer:cities_flags}
Which city does the flag in \reffig{fig:flag_question_city} belong to?
\end{exercise}

The flag in \reffig{fig:flag_question_city} belongs to \href{https://w.wiki/qqN}{Karabulak}. The answer to the question can also be obtained by running the following SPARQL query (Listing \ref{lst:cities_flags}). The \href{https://www.wikidata.org/wiki/Property:P41}{flag image} property value contains the image of the city flag.

\begin{lstlisting}[ language=SPARQL, 
                    caption={Cities flags.\\\hspace{\textwidth}
                        SPARQL query: \href{https://w.wiki/otp}{w.wiki/otp}
                        },
                    label=lst:cities_flags,
                    texcl 
                    ]
#defaultView:ImageGrid
SELECT ?city ?cityLabel ?flag ?countryLabel WHERE {
	?city wdt:P31/wdt:P279* wd:Q7930989. # "city/town" subclasses
	?city wdt:P17 ?country. # with property "country"
	?city wdt:P41 ?flag. # with property "flag"
	FILTER(?city = wd:Q144969) # for Karabulak only
	SERVICE wikibase:label {bd:serviceParam wikibase:language "en"}
}
\end{lstlisting}%

See question on page~\pageref{question:cities_flags}.

\section{Programming languages and its creators}
\labsec{answer-languages}
\begin{exercise}
    \label{answer:prog_lang_1}
Correlate a programming language and its developer.
\newline
	\begin{tabular}{ll}
		Developer & Language\\
		\hline
		J. Ichbiah & \href{https://www.wikidata.org/wiki/Q154755}{Ada}\\
		C. Moore & \href{https://www.wikidata.org/wiki/Q275472}{Forth}\\
		J. Armstrong & \href{https://www.wikidata.org/wiki/Q334879}{Erlang}\\
	\end{tabular}
\end{exercise}
    The Ada programming language was developed by Jean Ichbiah, Forth was developed by Charles H. Moore, and the creator of Erlang is believed to be Joe Armstrong. The answer to the question can also be obtained by running the following SPARQL query (listing \ref{lst:prog_lang_answer_1}). 
	\begin{lstlisting}[language=SPARQL, caption={{Programming languages developers}\protect\footnotemark}, label=lst:prog_lang_answer_1]
		SELECT ?item_label ?developer_label
		WHERE
		{
		 ?item wdt:P31 wd:Q9143
		 ; rdfs:label ?item_label. 
		 ?item wdt:P178 ?developer.
		 ?developer rdfs:label ?developer_label.
		 
		 FILTER (LANG(?item_label) = "en"). 
		 FILTER (LANG(?developer_label) = "en"). 
		}
		ORDER BY DESC (?item_label)
	\end{lstlisting}
SPARQL query: \href{https://w.wiki/kfZ}{https://w.wiki/kfZ}
\newline
Question from page~\pageref{question:prog_lang_1}.


\begin{exercise}
    \label{answer:prog_lang_2}
Which image is the programming language logo \href{https://www.wikidata.org/wiki/Q513238}{LOLCODE}: \newline
	\begin{tabular}{c c c c}
\includegraphics[width=2cm]{./chapter/programming_language/task_2_logo_1.PNG} & \includegraphics[width=2cm]{./chapter/programming_language/task_2_logo_2.PNG} & \includegraphics[width=2cm]{./chapter/programming_language/task_2_logo_3.PNG} & \includegraphics[width=2cm]{./chapter/programming_language/task_2_logo_4.PNG}
	\end{tabular}
\end{exercise}
    The third picture is the logo of the LOLCODE programming language. The answer to the question can also be obtained by running the following SPARQL query (listing \ref{lst:prog_lang_answer_1}). 
	\begin{lstlisting}[language=SPARQL, caption={{Programmers languages logos}\protect\footnotemark}, label=lst:prog_lang_answer_1]
		#defaultView:ImageGrid
		SELECT ?item_label ?image
		WHERE
		{
		 ?item wdt:P31 wd:Q9143 # instances of programming language
		 ; rdfs:label ?item_label. 
		 ?item wdt:P154 ?image. # image
		 	
		 	FILTER (lang(?item_label) = "en")
}
	\end{lstlisting}
SPARQL query: \href{https://w.wiki/kfd}{https://w.wiki/kfd}
\newline
Question from page~\pageref{question:prog_lang_2}.


\begin{exercise}
    \label{answer:prog_lang_3}
Fill the gaps.\newline
\href{https://www.wikidata.org/wiki/Q83303}{Fortran} ranks first in terms of the number of its dialects. Their number reaches about \underline{\hspace{1cm}}. In second place is \href{https://www.wikidata.org/wiki/Q132874}{Lisp}, it has \underline{\hspace{1cm}} dialects. The third place is shared by\href{https://www.wikidata.org/wiki/Q597330}{Standard ML} and \href{https://www.wikidata.org/wiki/Q633894}{Object Pascal} with \underline{\hspace{1cm}} dialects.\newline
\end{exercise}
 It is believed that Fortran has 8 to 12 dialects, Lisp has 6 dialects, and Standard ML and Object Pascal have 3 dialects.
    
Question from page~\pageref{question:prog_lang_3}.

