\setchapterimage[6cm]{graphics/intro/about_wd/Pile_of_books.jpeg}
\setchapterpreamble[u]{\margintoc}
\chapter{Review Wikidata}
\labch{ch-review-wd}

\footnotetext{Hendrik Conscience Heritage Library in Antwerp, Author: \href{https://commons.wikimedia.org/wiki/File:Pile_of_books.jpg}{Missmarettaphotography / 2016 / Creative Commons Attribution-Share Alike 4.0}.}

\section{What is Wikidata?}
\labsec{section:what-wd}
Wikidata is a structured and collaboratively edited database, which was created by the Wikimedia Foundation\sidenote[][*-1]{ Wikidata is a free, collaborative, multilingual, secondary database that collects structured information to support the operation of Wikipedia, Wikimedia Commons, and other Wikimedia Foundation wiki projects.} \begin{marginfigure}[1.5cm]
{
	\setlength{\fboxsep}{0pt}%
	\setlength{\fboxrule}{1pt}%
	\fcolorbox{gray}{white}{\includegraphics{./graphics/intro/about_wd/Wikidata-logo-en.png}}
}
\caption
{Wikidata logo. \newline
2012 / Planemad / Public domain
}
\label{fig:seyu}
\end{marginfigure}The project was officially launched on the 30th of October in 2012, and is being developed under the direction of Wikimedia Deutschland. The project was created with donations from the Allen Institute for Artificial Intelligence, the Gordon and Betty Moore Foundation and Google. At the moment, Wikidata is a free and open source knowledge base that can be used and edited by humans and machines.

Any Wikidata object has its own unique identifier and properties. This information can be processed using a computer; thus, it is understandable to users. The Wikidata site contains the ‘Wikidata Query’ service, which includes a set of tools for building SPARQL queries and visualizing them in the form of tables, charts, graphs or geographic maps.

Content on Wikidata is licensed under the Creative Commons CC0 license, which allows information to be reused in a variety of ways, allowing users to copy, modify, distribute and manipulate the data for any purpose. Another feature of Wikidata is multilingualism. Anyone can edit Wikidata in over 350 languages.

Wikidata is constantly updated, and new objects are being added. By 2021 there have been over 95 million pages created and over 1.5 billion edits made. Over 800,000 edits were made to Wikidata during the year of 2019, which surpassed the number of edits on the English Wikipedia, making Wikidata the most edited Wikimedia site.

